
\section{Ayudantía 1 (12/08)}

\begin{enumerate}
	\item Sea $X$ un conjunto infinito.
	\begin{enumerate}
		\item Sea $p \in X$ un punto arbitrario en $X$, demuestre que
		\[
		\tau_1 = \{ U \subset X \ : \ U = \varnothing \text{ o } p \in U \}
		\]
		es una topología en $X$. Esta es conocida como \textbf{topología del punto particular}.
		\begin{proof}~
			\begin{itemize}
				\item Claramente $\varnothing, X \in \tau_1$.

				\item $U_{\alpha} \in \tau_1 \implies p \in U_{\alpha} \implies p \in \bigcup_{\alpha \in I} U_{\alpha} \implies \bigcup U_{\alpha} \in \tau_1$.

				\item $U_1,\dots,U_n \in \tau_1 \implies p \in U_i \implies p \in \displaystyle\bigcap_{i=1}^{n} U_i \implies \displaystyle\bigcap_{i=1}^{n} U_i \in \tau_1$.
			\end{itemize}
		\end{proof}

		\item Sea $p \in X$ un punto arbitrario en $X$, demuestre que
		\[
		\tau_2 = \{ U \subset X \ : \ U = X \text{ o } p \not\in U \}
		\]
		es una topología en $X$. Esta es conocida como \textbf{topología del punto excluido}.
		\begin{proof}~
			\begin{itemize}
				\item Claramente $\varnothing, X \in \tau_2 \ (p \not\in \varnothing)$.

				\item $U_{\alpha} \in \tau_2 \implies p \not\in U_{\alpha} \implies p \not\in \bigcup U_{\alpha} \implies \bigcup U_{\alpha} \in \tau_2$.

				\item $U_1,\dots,U_n \in \tau_2 \implies p \not\in U_i \implies p \not\in \displaystyle\bigcap_{i=1}^{n} U_i \implies \displaystyle\bigcap_{i=1}^{n} U_i \in \tau_2$.
			\end{itemize}
		\end{proof}

		\item Determine cuando
		\[
		\tau_3 = \{ U \subset X \ : \ U = X \text{ o } X \setminus U \text{ es infinito} \}
		\]
		es una topología en $X$.
		\begin{proof}~
			Si $p \in X \implies \{ p \}^c$ es infinito $\implies \{ p \}$ es abierto. Si $\tau_3$ es topología y $q \in X$, entonces
			\[
			\bigcup_{p \neq q} \{ p \} = X \setminus \{ q \} \implies (X \setminus \{ q \})^c = \{ q \}
			\]
			es infinito. Contradicción! \textreferencemark \ Es decir, $\tau_3$ no es topología.
		\end{proof}
	\end{enumerate}

	\item Sea $K = \{ \frac{1}{n} \ : \ n \in \N \}$ y $\mathcal{B}_K$ la colección de intervalos abiertos $(a,b) \subset \R$ y de conjuntos de la forma $(a,b) - K$. Es decir,
	\[
	\mathcal{B}_K = \{ (a,b) \ : \ a,b \in \R \} \cup \{ (a,b)-K \}.
	\]
	\begin{enumerate}
		\item Pruebe que $\mathcal{B}_K$ es una base para $X$. Denotamos por $\R_K$ la topología generada.
		\begin{proof}~
			\begin{itemize}
				\item $t \in \R,\ \exists (a,b) \subset \R \ : \ t \in (a,b) \subset \mathcal{B}_K$;

				\item Notar que la intersección de elementos de la base es un elemento de la base
				\begin{align*}
					(a,b) & \cap (c,d)  = (c,b) \\
					(a,b)-K & \cap (c,d)  = (c,b)-K \\
					(a,b)-K & \cap (c,d)-K  = (c,b)-K
				.\end{align*}
			\end{itemize}
		\end{proof}

		\item Considere las siguientes topologías en $\R$:
		\begin{align*}
			\tau_1 & = \text{ topología estándar en } \R \\
			\tau_2 & = \text{ topología } \R_K \\
			\tau_3 & = \text{ topología cofinita en } \R \\
			\tau_4 & = \text{ topología con } (-\infty, a) = \{ x \ : \ x < a \} \text{ como base}  
		.\end{align*}
		Determine, para cada una de estas topologías, cual de las otras contiene.
		%\begin{proof}~
		%	$\tau_{std} = \tau_1 \subsetneq \tau_2 = \R_K$ ya que $(a,b) \in \R_K$ para todo intervalo abierto, y a su vez estas forman una base de $\tau_1$. 
		%\end{proof}
	\end{enumerate}

	\item Sea $X$ conjunto, denotamos por $\tau_{\text{cof}}(X)$ a la topología cofinita en $X$. Demuestre que
	\[
	\tau_{\text{cof}} (X \times Y) \subseteq \tau_{\text{cof}} (X) \times \tau_{\text{cof}} (Y),
	\]
	es decir, la topología producto de las topologías cofinitas es más finita que la topología cofinita en $X \times Y$. De un ejemplo en que estas topologías no son iguales.
	\begin{proof}
		Sea $U \in \tau_{\text{cof}} (X \times Y)$ Entonces,
		\[
		U = X \times Y - \{ (a_1,b_1),\dots,(a_n,b_n) \} = \displaystyle\bigcap_{i=1}^{n} X \times Y - \{ (a_i,b_i) \}.
		\]
		Queremos ver que $X \times Y - \{ (a,b) \}$ es abierto en $\tau_{\text{cof}}(X)\times\tau_{\text{cof}}(Y)$. Notar que $\mathcal{B} = \{ U \times V \ : \ U \in \tau_{\text{cof}}(X), V \in \tau_{\text{cof}}(Y) \}$ es una base para $\tau_{\text{cof}}(X) \times \tau_{\text{cof}}(Y)$. Sea $(c,d) \in X \times Y - \{ (a,b) \}$. Supongamos que $c \neq a$.
		\[
		(c,d) \in \underbrace{X \setminus \{a\}}_{\in \tau_{\text{cof}}(X)} \times \underbrace{Y}_{\in\tau_{\text{cof}}(Y)} \subseteq X \times Y - \{(a,b)\}.  
		\]
		$X$ infinito. Luego,
		\[
		W = X \times Y \setminus \{a\} \in \tau_{\text{cof}}(X) \times \tau_{\text{cof}}(Y).
		\]
		Pero $W^c = X \times \{a\}$ no es finito. Por lo tanto $W \in \tau_{\text{cof}} (X \times Y)$. Entonces $\tau_{\text{cof}} (X \times Y) \subsetneq \tau_{\text{cof}}(X) \times \tau_{\text{cof}}(Y)$.
	\end{proof}

\end{enumerate}
