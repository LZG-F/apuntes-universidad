
\section{Ayudanía 2 (19/08)}

\begin{enumerate}
	\item Considere $W \subseteq Y \subseteq X$, $\tau$ topología en $X$, $\tau'$ topología inducida en $X$. Pruebe que las topologías inducidas por $\tau$ y $\tau'$ sen $W$ son iguales.
	\begin{proof}[Proof ]
		$U$ abierto en $\tau|_{W} \iff U = U_X \cap W$.\\
		$\begin{aligned}
			U \text{ abierto en } \tau'|_{W} & \iff U = \overbrace{(U_X \cap Y)}^{\text{abierto en } \tau'} \cap W \\
			& \iff U_X \cap W
		.\end{aligned}$
	\end{proof}

	\item Considere $\Z$ con la topología profinita y $H \leq \Z$. Pruebe que la topología inducida en $H$ es igual a la topología profinita en $H$.
	\begin{remark}
		Topología profinita: $G,\ \tau (G).\ \mathcal{B} = \{ gH \ : \ H \lhd G, \ [G : H] < \infty \}$. \\
		\noindent Caso de $Z$: si $a,b \in \Z,\ b\neq 0$, entonces $S(a,b) = \{ a+bn \ : \ n \in \Z \}$. Luego, $G = \Z,\ H = (b),\ b\neq 0$.
		\begin{align*}
			\mathcal{B} & = \{ a(b) \ : \ b \neq 0 \} \\
			& = \{ a+bn \ : \ n \neq 0 \} \\
			& = \{ S(a,b) \}  
		.\end{align*}
	\end{remark}
	\begin{proof}[Proof ]
		$\tau (\Z) =$ topología profinita en $\Z$. Si $H \leq \Z,\ H = (m),\ m \neq 0,1$. Notar que se debe demostrar $\tau (H) = \tau (\Z)|_{H}$. (Afirmamos que $\mathcal{B}(H) = \{ S(sm,tm) \ : \ t \neq 0 \}$ es base de $\tau (H)$ y que $\mathcal{B}(Z) \cap H$ es base de $\tau(\Z)|_{H}$) \\
		\noindent \fbox{$\supseteq$} Sea $x \in  B \in \mathcal{B}(\Z) \cap H,\ B = S(a,b) \cap (m)$. Luego, $x = a + bn \in (m)$. $x \in S(x,bm) \subseteq S(a,b) \cap (m)$. Si $y \in S(x,bm)$, entonces
		\begin{align*}
			y & = x + bmk \\
			& = a + bn + bmk \\
			& = a + b(n+mk) \in S(a,b)
		.\end{align*}
		\noindent \fbox{$\subseteq$} Sea $B \in \mathcal{B}(H)$. Luego $B= S(tm,sm),\ s \neq 0$. Notar que $B$ es un abierto de la base de $\tau(\Z)$ y que $B \subseteq H = (m)$. Entonces, $B \in \tau (\Z)|_{H}$.
	\end{proof}

	\item $X$ espacio topológico, $A \subseteq X$. Definimos la frontera de $A$ por
	\[
	\partial A \coloneq \overline{A} \cap \overline{X \setminus A}.
	\]
	\begin{enumerate}
		\item $\partial A = \overline{A} \setminus int(A)$;

		\item Demuestre que $int(A),\ \partial A$ son disjuntos y $\overline{A} = int(A) \cup \partial A$;

		\item $\partial A = \varnothing \iff A$ abierto y cerrado;

		\item $U$ abierto $\iff \partial U = \overline{U} \setminus U$;

		\item $U$ abierto ¿$U = int(\overline{U})$?
	\end{enumerate}
	\begin{proof}[Proof ]
		\begin{enumerate}
			\item Notar que
			\[
			\overline{A} \setminus int(A) = \overline{A} \cap int(A)^c.
			\]
			Basta ver que $int(A)^c = \overline{X \setminus A}$.
			\begin{align*}
				x \in int(A)^c & \iff \forall \ U(X),\ U(X) \not\subseteq A \\
				& \iff \forall \ U(X),\ U(X) \cap (X-A) \neq \varnothing. \\
				x \in \overline{X \setminus A} & \iff \forall \ U(X),\ U(X) \cap (X \setminus A) \neq \varnothing.
			.\end{align*}

			\item $\partial A = \overline{A} \setminus int(A) \implies \partial A \cap int(A) = \varnothing$. Luego, $(\bigcup int(A)),\ partial A \cup int(A) = \overline{A}$.

			\item $\partial A = \varnothing \iff A$ abierto y cerrado.\\
			\Onlyifstep Basta ver que \begin{align*}
				\partial A & = \overline{A} \cap \overline{X \setminus A} \\
				& = A \cap X \setminus A \\
				& = \varnothing.
			\end{align*}

			\Ifstep Basta ver que
			\begin{align*}
				\partial A = \varnothing = \overline{A}\setminus int(A) & \implies int(A) = \overline{A} \quad (int(A) \subseteq A \subseteq \overline{A}) \\
				& \implies A = \overline{A} = int(A) \\
				& \implies A \text{ abierto y cerrado}
			.\end{align*}

			\item $U$ abierto $\iff \partial U = \overline{U} \setminus U$

			\Onlyifstep $\partial U = \overline{U} \setminus int(U) = \overline{U} \setminus U$.

			\Ifstep $\overline{U} =  int(U) \cup \partial U = int(U) \cup (\overline{U} \setminus U)$. $\overline{U} \cap (\overline{U} \cap U^c)^c = \overline{U} \setminus (\overline{U}\setminus U) = int(U)$. (terminar dem)

			\item La igualdad es falsa, pero una de las contenciones es real.
		\end{enumerate}
	\end{proof}
\end{enumerate}
