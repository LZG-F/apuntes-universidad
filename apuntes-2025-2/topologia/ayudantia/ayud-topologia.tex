\documentclass[a4paper]{report}
\usepackage[utf8]{inputenc}
\usepackage[T1]{fontenc}
\usepackage{textcomp}

\usepackage{url}

% \usepackage{hyperref}
% \hypersetup{
%     colorlinks,
%     linkcolor={black},
%     citecolor={black},
%     urlcolor={blue!80!black}
% }

\usepackage{graphicx}
\usepackage{float}
\usepackage[usenames,dvipsnames]{xcolor}

% \usepackage{cmbright}

\usepackage{amsmath, amsfonts, mathtools, amsthm, amssymb}
\usepackage{mathrsfs}
\usepackage{cancel}

\newcommand\N{\ensuremath{\mathbb{N}}}
\newcommand\R{\ensuremath{\mathbb{R}}}
\newcommand\Z{\ensuremath{\mathbb{Z}}}
\renewcommand\O{\ensuremath{\emptyset}}
\newcommand\Q{\ensuremath{\mathbb{Q}}}
\newcommand\C{\ensuremath{\mathbb{C}}}
\let\implies\Rightarrow
\let\impliedby\Leftarrow
\let\iff\Leftrightarrow
\let\epsilon\varepsilon

% demostraciones bidireccionales

\newcommand{\Onlyifstep}{%
	\begingroup
	\fboxsep=1pt
	\raisebox{1.2ex}{\fbox{\raisebox{-1.2ex}{$\Rightarrow$\hspace{-0.05em}}}}%
	\endgroup
	\hspace{0.5em}%
	}
\newcommand{\Ifstep}{%
	\begingroup
	\fboxsep=1pt
	\raisebox{1.2ex}{\fbox{\raisebox{-1.2ex}{\hspace{-0.05ex}$\Leftarrow$}}}%
	\endgroup
	\hspace{0.5em}%
	}

% horizontal rule
\newcommand\hr{
    \noindent\rule[0.5ex]{\linewidth}{0.5pt}
}

\usepackage{tikz}
\usepackage{tikz-cd}

% theorems
\usepackage{thmtools}
\usepackage[framemethod=TikZ]{mdframed}
\mdfsetup{skipabove=1em,skipbelow=0em, innertopmargin=5pt, innerbottommargin=6pt}

\theoremstyle{definition}

\makeatletter

\declaretheoremstyle[headfont=\bfseries\sffamily, bodyfont=\normalfont, mdframed={ nobreak } ]{thmgreenbox}
\declaretheoremstyle[headfont=\bfseries\sffamily, bodyfont=\normalfont, mdframed={ nobreak } ]{thmredbox}
\declaretheoremstyle[headfont=\bfseries\sffamily, bodyfont=\normalfont]{thmbluebox}
\declaretheoremstyle[headfont=\bfseries\sffamily, bodyfont=\normalfont]{thmblueline}
\declaretheoremstyle[headfont=\bfseries\sffamily, bodyfont=\normalfont, numbered=no, mdframed={ rightline=false, topline=false, bottomline=false, }, qed=\qedsymbol ]{thmproofbox}
\declaretheoremstyle[headfont=\bfseries\sffamily, bodyfont=\normalfont, numbered=no, mdframed={ nobreak, rightline=false, topline=false, bottomline=false } ]{thmexplanationbox}


\declaretheorem[numberwithin=chapter, style=thmgreenbox, name=Definition]{definition}
\declaretheorem[sibling=definition, style=thmredbox, name=Corollary]{corollary}
\declaretheorem[sibling=definition, style=thmredbox, name=Proposition]{prop}
\declaretheorem[sibling=definition, style=thmredbox, name=Theorem]{theorem}
\declaretheorem[sibling=definition, style=thmredbox, name=Lemma]{lemma}



\declaretheorem[numbered=no, style=thmexplanationbox, name=Proof]{explanation}
\declaretheorem[numbered=no, style=thmproofbox, name=Proof]{replacementproof}
\declaretheorem[style=thmbluebox,  numbered=no, name=Exercise]{ex}
\declaretheorem[style=thmbluebox,  numbered=no, name=Example]{eg}
\declaretheorem[style=thmblueline, numbered=no, name=Remark]{remark}
\declaretheorem[style=thmblueline, numbered=no, name=Note]{note}

\renewenvironment{proof}[1][\proofname]{\begin{replacementproof}}{\end{replacementproof}}

\AtEndEnvironment{eg}{\null\hfill$\diamond$}%

\newtheorem*{uovt}{UOVT}
\newtheorem*{notation}{Notation}
\newtheorem*{previouslyseen}{As previously seen}
\newtheorem*{problem}{Problem}
\newtheorem*{observe}{Observe}
\newtheorem*{property}{Property}
\newtheorem*{intuition}{Intuition}


\usepackage{etoolbox}
\AtEndEnvironment{vb}{\null\hfill$\diamond$}%
\AtEndEnvironment{intermezzo}{\null\hfill$\diamond$}%




% http://tex.stackexchange.com/questions/22119/how-can-i-change-the-spacing-before-theorems-with-amsthm
% \def\thm@space@setup{%
%   \thm@preskip=\parskip \thm@postskip=0pt
% }

\usepackage{xifthen}

\def\testdateparts#1{\dateparts#1\relax}
\def\dateparts#1 #2 #3 #4 #5\relax{
    \marginpar{\small\textsf{\mbox{#1 #2 #3 #5}}}
}

\def\@lesson{}%
\newcommand{\lesson}[3]{
    \ifthenelse{\isempty{#3}}{%
        \def\@lesson{Lecture #1}%
    }{%
        \def\@lesson{Lecture #1: #3}%
    }%
    \subsection*{\@lesson}
    \testdateparts{#2}
}

% fancy headers
\usepackage{fancyhdr}
\pagestyle{fancy}

% \fancyhead[LE,RO]{Gilles Castel}
\fancyhead[RO,LE]{\@lesson}
\fancyhead[RE,LO]{}
\fancyfoot[LE,RO]{\thepage}
\fancyfoot[C]{\leftmark}
\renewcommand{\headrulewidth}{0pt}

\makeatother

% figure support (https://castel.dev/post/lecture-notes-2)
\usepackage{import}
\usepackage{xifthen}
\pdfminorversion=7
\usepackage{pdfpages}
\usepackage{transparent}
\newcommand{\incfig}[1]{%
    \def\svgwidth{\columnwidth}
    \import{./figures/}{#1.pdf_tex}
}

% %http://tex.stackexchange.com/questions/76273/multiple-pdfs-with-page-group-included-in-a-single-page-warning
\pdfsuppresswarningpagegroup=1

\author{Gilles Castel}

\DeclareMathOperator{\supp}{supp}
\DeclareMathOperator{\spann}{span}
\DeclareMathOperator{\Id}{Id}
\DeclareMathOperator{\Ker}{Ker}
\DeclareMathOperator{\im}{Im}
\DeclareMathOperator{\GL}{GL}
\DeclareMathOperator{\SL}{SL}
\DeclareMathOperator{\Mat}{Mat}
\title{Ayudantía Topología}
\author{}
\date{}

\begin{document}

\maketitle
\tableofcontents
% start lessons

\chapter{Ayudantía 1 (12/08)}

\begin{enumerate}
	\item Sea $X$ un conjunto infinito.
	\begin{enumerate}
		\item Sea $p \in X$ un punto arbitrario en $X$, demuestre que
		\[
		\tau_1 = \{ U \subset X \ : \ U = \varnothing \text{ o } p \in U \}
		\]
		es una topología en $X$. Esta es conocida como \textbf{topología del punto particular}.
		\begin{proof}~
			\begin{itemize}
				\item Claramente $\varnothing, X \in \tau_1$.

				\item $U_{\alpha} \in \tau_1 \implies p \in U_{\alpha} \implies p \in \bigcup_{\alpha \in I} U_{\alpha} \implies \bigcup U_{\alpha} \in \tau_1$.

				\item $U_1,\dots,U_n \in \tau_1 \implies p \in U_i \implies p \in \displaystyle\bigcap_{i=1}^{n} U_i \implies \displaystyle\bigcap_{i=1}^{n} U_i \in \tau_1$.
			\end{itemize}
		\end{proof}

		\item Sea $p \in X$ un punto arbitrario en $X$, demuestre que
		\[
		\tau_2 = \{ U \subset X \ : \ U = X \text{ o } p \not\in U \}
		\]
		es una topología en $X$. Esta es conocida como \textbf{topología del punto excluido}.
		\begin{proof}~
			\begin{itemize}
				\item Claramente $\varnothing, X \in \tau_2 \ (p \not\in \varnothing)$.

				\item $U_{\alpha} \in \tau_2 \implies p \not\in U_{\alpha} \implies p \not\in \bigcup U_{\alpha} \implies \bigcup U_{\alpha} \in \tau_2$.

				\item $U_1,\dots,U_n \in \tau_2 \implies p \not\in U_i \implies p \not\in \displaystyle\bigcap_{i=1}^{n} U_i \implies \displaystyle\bigcap_{i=1}^{n} U_i \in \tau_2$.
			\end{itemize}
		\end{proof}

		\item Determine cuando
		\[
		\tau_3 = \{ U \subset X \ : \ U = X \text{ o } X \setminus U \text{ es infinito} \}
		\]
		es una topología en $X$.
		\begin{proof}~
			Si $p \in X \implies \{ p \}^c$ es infinito $\implies \{ p \}$ es abierto. Si $\tau_3$ es topología y $q \in X$, entonces
			\[
			\bigcup_{p \neq q} \{ p \} = X \setminus \{ q \} \implies (X \setminus \{ q \})^c = \{ q \}
			\]
			es infinito. Contradicción! \textreferencemark \ Es decir, $\tau_3$ no es topología.
		\end{proof}
	\end{enumerate}

	\item Sea $K = \{ \frac{1}{n} \ : \ n \in \N \}$ y $\mathcal{B}_K$ la colección de intervalos abiertos $(a,b) \subset \R$ y de conjuntos de la forma $(a,b) - K$. Es decir,
	\[
	\mathcal{B}_K = \{ (a,b) \ : \ a,b \in \R \} \cup \{ (a,b)-K \}.
	\]
	\begin{enumerate}
		\item Pruebe que $\mathcal{B}_K$ es una base para $X$. Denotamos por $\R_K$ la topología generada.
		\begin{proof}~
			\begin{itemize}
				\item $t \in \R,\ \exists (a,b) \subset \R \ : \ t \in (a,b) \subset \mathcal{B}_K$;

				\item Notar que la intersección de elementos de la base es un elemento de la base
				\begin{align*}
					(a,b) & \cap (c,d)  = (c,b) \\
					(a,b)-K & \cap (c,d)  = (c,b)-K \\
					(a,b)-K & \cap (c,d)-K  = (c,b)-K
				.\end{align*}
			\end{itemize}
		\end{proof}

		\item Considere las siguientes topologías en $\R$:
		\begin{align*}
			\tau_1 & = \text{ topología estándar en } \R \\
			\tau_2 & = \text{ topología } \R_K \\
			\tau_3 & = \text{ topología cofinita en } \R \\
			\tau_4 & = \text{ topología con } (-\infty, a) = \{ x \ : \ x < a \} \text{ como base}  
		.\end{align*}
		Determine, para cada una de estas topologías, cual de las otras contiene.
		%\begin{proof}~
		%	$\tau_{std} = \tau_1 \subsetneq \tau_2 = \R_K$ ya que $(a,b) \in \R_K$ para todo intervalo abierto, y a su vez estas forman una base de $\tau_1$. 
		%\end{proof}
	\end{enumerate}

	\item Sea $X$ conjunto, denotamos por $\tau_{\text{cof}}(X)$ a la topología cofinita en $X$. Demuestre que
	\[
	\tau_{\text{cof}} (X \times Y) \subseteq \tau_{\text{cof}} (X) \times \tau_{\text{cof}} (Y),
	\]
	es decir, la topología producto de las topologías cofinitas es más finita que la topología cofinita en $X \times Y$. De un ejemplo en que estas topologías no son iguales.
	\begin{proof}
		Sea $U \in \tau_{\text{cof}} (X \times Y)$ Entonces,
		\[
		U = X \times Y - \{ (a_1,b_1),\dots,(a_n,b_n) \} = \displaystyle\bigcap_{i=1}^{n} X \times Y - \{ (a_i,b_i) \}.
		\]
		Queremos ver que $X \times Y - \{ (a,b) \}$ es abierto en $\tau_{\text{cof}}(X)\times\tau_{\text{cof}}(Y)$. Notar que $\mathcal{B} = \{ U \times V \ : \ U \in \tau_{\text{cof}}(X), V \in \tau_{\text{cof}}(Y) \}$ es una base para $\tau_{\text{cof}}(X) \times \tau_{\text{cof}}(Y)$. Sea $(c,d) \in X \times Y - \{ (a,b) \}$. Supongamos que $c \neq a$.
		\[
		(c,d) \in \underbrace{X \setminus \{a\}}_{\in \tau_{\text{cof}}(X)} \times \underbrace{Y}_{\in\tau_{\text{cof}}(Y)} \subseteq X \times Y - \{(a,b)\}.  
		\]
		$X$ infinito. Luego,
		\[
		W = X \times Y \setminus \{a\} \in \tau_{\text{cof}}(X) \times \tau_{\text{cof}}(Y).
		\]
		Pero $W^c = X \times \{a\}$ no es finito. Por lo tanto $W \in \tau_{\text{cof}} (X \times Y)$. Entonces $\tau_{\text{cof}} (X \times Y) \subsetneq \tau_{\text{cof}}(X) \times \tau_{\text{cof}}(Y)$.
	\end{proof}











\end{enumerate}

% end lessons
\end{document}
