\clase{43}{21 de Noviembre}{}

\section{$\pi_{1}(S^1 \vee S^1)$}

\begin{note}
	Se puede ver que $\pi_{1}(S^{1} \vee S^{1}) \cong F_{2}$ el grupo libre.
\end{note}

\begin{pregunta}
	¿Cubrimiento simplemente conexo de $X = S^1 \vee S^1$ (ver foto)?
\end{pregunta}

\begin{note}
	$\mcal{X} \approx (-1,1) \times \{0\} \cup \{0\} (-1,1)$ en $\R^2$.  
\end{note}

VER FOTO para entender $T$.

\begin{af}[1]
	Colores + flechas definen un cubrimiento $p : T \to X$
	\begin{enumerate}
		\item Vértices en $X \setminus \{x_{0}\}$: hay vecindades $\approx (0,1)$.

		\item $x_{0}$: hay vecindad $\approx \mcal{X}$ (ver foto)
	\end{enumerate}
\end{af}

\begin{note}
	$p^{-1}(V)$ es unión disjunta de espacios homeomorfos a $\mcal{X}$, con $V$ vecindad del punto $x_{0}$ que define la cuña. Además, $p^{-1}(U) =$ unión disjunta de intervalos, con $U$ vecindad de $x \neq x_{0}$.
\end{note}

\begin{af}[2]
	$T$ es contractible. 
	\begin{itemize}
		\item $T$ es un árbol (grafo sin ciclos cerrados) $\implies$ Hay un camino "canónico" desde cualquier punto de $T$ al vértice $e$.

		\item Si $\gamma_{x}$ es el camino canónico que va de $x$ a $e \implies R \colon T \times [0,1] \to T$ tal que $(x,t) = \gamma_{x}(t)$ es retracto por deformación.
	\end{itemize}
\end{af}

\begin{remark}
	Esto implica que $T$ es simplemente conexo.
\end{remark}

\begin{af}
	$\pi_{1}(X) \not\cong \{1\}$.
\end{af}
\begin{proof}[Proof ]
	Levantar $\alpha$ a $\widetilde{\alpha}$ camino que comienza en $e$. Entonces $\widetilde{\alpha}(1) \neq e$. Por lo tanto, $[\alpha] \neq 1$ en $\pi_{1}(X)$ (Esto también dice que $[\alpha]^{n} \neq 1 \quad \forall n \neq 0$).
\end{proof}

\begin{af}
	$\pi_{1}(X)$ no es abeliano.
\end{af}

\begin{notation}
	$a = [\alpha],\ b = [\beta]$.
\end{notation}

\begin{proof}[Proof Other Information]
	Tomemos levantamientos $\widetilde{\alpha * \beta},\ \widetilde{\beta * \alpha}$ de $\alpha * \beta,\ \beta * \alpha$ respectivamente que comienzan en $e$. \par 
	Al ver $T$ se aprecia que $\widetilde{\alpha * \beta}(1) \neq \widetilde{\beta * \alpha}(1)$. Por lo tanto $ab = [\alpha * \beta] \neq [\beta * \alpha] = ba$ (se está usando que $T$ es simplemente conexo).
\end{proof}

\begin{note}
	Hasta aquí, $\Gamma = \pi_{1}(X)$ es:
	\begin{enumerate}
		\item Infinito;

		\item Contiene a $\Z$;

		\item No abeliano.
	\end{enumerate}
\end{note}

\begin{af}
	$a$ y $b$ generan $\pi_{1}(X)$.
\end{af}

\begin{proof}[Proof Other Information]
	Si $\gamma$ es loop en $X$ con base $x_{0}$, tomamos $\widetilde{\gamma}$ levantamiento que comienza en $e$. Notar que $\widetilde{\gamma}(1)$ es vértice de $T$. \par
	Sea $\widetilde{\gamma}_{c}$ el camino canónico que comienza en $e$ y termina en $\widetilde{\gamma}(1)$. Luego, $\widetilde{\gamma}_{c}$ es concatenación de aristas de $T$. \par
	Entonces, $\gamma_{c} = p \circ \widetilde{\gamma}_{c}$ es concatenación de  $\alpha$'s, $\beta$'s, $\overline{\alpha}$'s y $\overline{\beta}$'s. Como $\widetilde{\gamma}_{c}$ es levantamiento de $\gamma_{c}$ y $\widetilde{\gamma}_{c}(1) = \gamma(1) \implies \gamma_{c} \sim_{p} \gamma_{c}$. \par
	Por lo tanto, $[\gamma] = [\gamma_{c}]$ es producto de muchos $a$'s, $b$'s, $a^{-1}$'s y $b^{-1}$'s.
\end{proof}
