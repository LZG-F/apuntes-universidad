\clase{11}{29 de Agosto}{}

\noindent \textbf{Contexto.} $p:X \to A$ sobreyectiva, $X$ espacio topológico. Uno quiere dar una topología "natural" a $A$ tal que $\rho$ sea continua.

\begin{eg}[estándar]
	Si $\sim$ relación de equivalencia en $X$, con $X \setminus \sim =$ \textit{conjunto de clases de equivalencia}
	\[ \rho : X \to X \setminus \sim,\quad x \mapsto [x]_{\sim} \]
\end{eg}

\begin{eg}[1.] 
	\textbf{Colapsar subespacios.} $Y \subset X$. Luego, $\sim$ en $X$ tal que todos los puntos de $Y$ son equivalentes (y nada más). Entonces, $X \setminus Y = X \setminus \sim$. \newline
\end{eg}

\begin{eg}[1.1]
	$X = [0,1],\ Y = \{0,1\}$ \newline
\end{eg}

\begin{eg}[1.2]
	$X = D^2 = \{x\in\R^2 \ \big| \ |x|\leq 1 \},\ Y = \mathbb{S}^1 = \{x \ \big| \ |x| = 1\}$. \newline
\end{eg}

\begin{eg}[2.]
	\textbf{Acciones de grupo.} $\Gamma$ grupo, $X$ espacio. Acción es $\rho : \Gamma \times X \to X$ (notación $\rho(g,x) = g\cdot x$) tal que
	\begin{enumerate}
		\item $\rho(1_{\Gamma},x) = x \quad \forall x \in X$;

		\item $\rho(gh,x) = \rho(g,\rho(h,x))$.
	\end{enumerate}
\end{eg}

\begin{observe}
	$\rho$ es mismo dato de un homomorfismo
	\[ \Gamma \to Biy(X),\quad g \mapsto (x \mapsto \rho(g,x)) \]
\end{observe}

\begin{eg}
	$\Z^2 \curvearrowright \R^2,\ (m,n)\cdot(x,y) = (x+m,y+n)$.
\end{eg}

Notar que si tenemos $\Gamma \curvearrowright X$ acción, nos da $\sim_{\Gamma}$ tal que $x \sim_{\Gamma} y$ si y sólo si $y = g\cdot x$ para algún $g \in \Gamma$ ($x,y$ en la misma órbita). Además, $X \setminus \Gamma \coloneq X \setminus \sim_{\Gamma}$ espacio de órbitas, o cociente de $X$ por la acción de $\Gamma$. \newline

\noindent \textbf{Contexto.} $p : X \to A$ sobreyectiva, $X$ espacio topológico.

\begin{definition}[topología cociente en $A$]
	\[ \tau = \{ U \subset A \ \big| \ p^{-1}(U) \text{ es abierto en } X \} \]
\end{definition}

\noindent \textbf{Esto es topología:} Viene de que
\begin{align*}
	p^{-1}\left(\bigcup_{\alpha} U_{\alpha}\right) & = \bigcup_{\alpha} p^{-1}(U_{\alpha}) \\
	p^{-1}\left(\bigcap_{\alpha} U_{\alpha}\right) & = \bigcap_{\alpha} p^{-1}(U_{\alpha})
.\end{align*}

\begin{observe}~
	\begin{enumerate}
		\item $p$ es continua si $A$ tiene topología cociente;

		\item Se cumple algo más fuerte
		\[ U \subset A \text{ abierto} \iff p^{-1}(U) \subset X \text{ abierto} \tag{$*$} \]
		[top. cociente es topología más grande tal que $p$ es continua]
	\end{enumerate}
\end{observe}

\begin{definition}[mapa cociente]
	Si $X,A$ son espacios topológicos $p:X \to A$ es \underline{mapa cociente} si es sbore y se cumple $(*)$.
\end{definition}

\begin{observe}
	$X \stackrel{p}{\twoheadrightarrow},\ A$ con topología cociente
	\begin{enumerate}
		\item Si $p$ es continua respecto a $\tau'$ otra topología en $A$, entonces $\tau' \subset \tau_{\text{coc}}$;

		\item $p$ es mapa cociente con respecto a $\tau_{\text{coc}}$. Si $p$ es mapa cociente con respecto a topología $\tau$ en $A$, entonces $\tau = \tau_{\text{coc}}$.
	\end{enumerate}
	$ [ X \stackrel{p}{\to} \text{ mapa cociente } \equiv X \stackrel{p}{\to} A \text{ sobre y } A \text{ tiene top. cociente.} ] $
\end{observe}

\begin{property}
	Suponer que $p: X \to A$ mapa cociente, $Y$ espacio topológico, $f: A \to Y$. Sea $g = f \circ p : X \to Y$. Luego,
	\[ f \text{ es continua} \iff g \text{ es continua} \]
\end{property}
\begin{proof}[Proof ]
	\Ifstep Si $U\subset Y$ abierto (queremos $f^{-1}(U) \subset A$ abierto). Luego, $g$ continua implica que $g^{-1}(U)\subset X$ abierto. Notar que $g^{-1}(U) = (f\circ g)^{-1}(U) = p^{-1}(f^{-1}(U)) \subset X$ abierto. Dado que $p$ es mapa cociente, entonces $f^{-1}(U) \subset A$ abierto.
\end{proof}

\begin{eg}
	$g : [0,2\pi] = X \to \mathbb{S}^1 = \{|z| = 1\},\quad t \mapsto (\cos t,\sin t)$.
	\begin{itemize}
		\item $A = [0,2\pi] \setminus \{0,2\pi\}$;

		\item $p: X \to A$.
	\end{itemize}
	$(g$ cumple $(*)) \implies$ hay una función $f: A \to \mathbb{S}^1$ tal que 

	\[
	\begin{tikzcd}
		{[0,2\pi]} \arrow[d, "p"'] \arrow[rd, "g \text{ (continua sobre } y)"] &              \\
		A \arrow[r, "\text{biyección}"']                                                 & \mathbb{S}^1
	\end{tikzcd}
	\]
\end{eg}

\noindent Propiedad anterior $\implies f$ continua (y biyectiva) \newline

\noindent \textbf{Clave.} $f^{-1}$ es continua! $\leadsto$ [$U\subset A$ abierto $\implies f(U)$ abierto en $\mathbb{S}^1$]

\begin{proof}
	Suponer $U$ que contiene a $p(0) = p(2\pi) \implies p^{-1}(U)$ abierto que contiene a $0$ y a $2\pi$. Entonces, $U$ contiene a $[0,\varepsilon)\cup (2\pi-\varepsilon,2\pi]$ para algún $\varepsilon$ chiquito. Luego $g(U)$ contiene vecindad de $g(p(0))$.
\end{proof}
