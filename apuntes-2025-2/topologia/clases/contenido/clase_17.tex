\section{Clase 17 (12/09): (Arco)conexidad local, Componentes (25)}

\begin{observe}
	Conexidad $\not\implies$ Arcoconexidad.
\end{observe}
\begin{eg}
	$Y = \{(t,\sin(\frac{1}{t}) \ \big| \ t > 0\} \subset \R^2$ arcoconexo. $X = \overline{Y}$ conexo! Pero no es arcoconexo! 
\end{eg}

\begin{lemma}
	$Y \subset A$ espacios topológicos tal que $Y \subset X \subset \overline{Y}$. Si $Y$ es conexo $\implies X$ conexo.
\end{lemma}
\begin{note}
	El $A$ es simplemente porque $Y$ tiene que estar dentro de un espacio para poder tomar su clausura.
\end{note}

\subsection*{Componentes}
\medskip
\begin{definition}[componentes conexa y arcoconexa]
	Sea $X$ espacio topológico, $C \subset X$ es componente conexa (resp. arcoconexa) si:
	\begin{enumerate}
		\item $C$ es conexo (resp. arcoconexo);

		\item $C$ es maximal respecto a $(1)$: Si $C'$ es (arco)conexo y $C \subset C' \implies C = C'$.
	\end{enumerate}
\end{definition}

\begin{observe}~
	\begin{enumerate}
		\item Componentes existen: Si $x \in X$
		\[ C_x \coloneq \bigcup \{C \subset X \ \big| \ C \text{ conexo, } x \in C\} \]
		($C_x$ componente de $x$ en $X$). Esto es conexo (criterio) y \underline{maximal}.

		\item Lo mismo vale para arcoconexidad (Existe versión del criterio).

		\item Componentes conexas forman una partición de $X$. Si $C_x \neq C_y \implies C_x \cap C_y = \varnothing$. En efecto, si $C_x \cap C_y \neq \varnothing,\ C_x \neq C_y \implies C_x \cup C_y$ es conexo aún más grande.

		\item Si $C \subset X$ componente conexa $\implies C$ es cerrado $\implies C = \overline{C}$ ($\overline{C}$ conexo + $C$ conexo maximal) (esto es falso si se reemplaza por componente arcoconexa).
	\end{enumerate}
\end{observe}

\begin{eg}~
	\begin{enumerate}
		\item $X$ es (arco)conexo si $X$ es componente (arco)conexa;

		\item En $X = \Q$ con topología inducida de $\R$, componentes son los singleton. En particular, notar que componentes no son abiertas;

		\item $X = [0,1] \cup (2,3) \cup \{4\}$ (y es claro que $[0,1],\ (2,3)$ y $\{4\}$ son componentes) (aquí componentes son abiertas);

		\item Subconjuntos conexos de $\R$
		\begin{itemize}
			\item $[a,b],\ (a,b],\ [a,b),\ (a,b) \qquad a<b$;

			\item $(-\infty,a),\ (-\infty,a],\ (b,\infty),\ [b,\infty)$;

			\item $\R$;

			\item $\{x\}$.  
		\end{itemize}
		(todos arcoconexos!!!)

		\item $X = \overline{Y} \subset \R^2$. Componentes conexas de $X$: es sólo $X$. Componentes arcoconexas de $X$: $Y,\ \{0\} \times [-1,1]$.
	\end{enumerate}
\end{eg}

\begin{definition}[localmente (arco)conexo]
	$X$ espacio topológico es \uline{localmente} (arco)conexo si $\forall x \in X$, para todo abierto $U \subset X$ tal que $x \in U$, va a existir $V \subset U$ abierto (arco)conexo con $x \in V$.
\end{definition}

\begin{criterio}
$X$ localmente (arco)conexo si y sólo si $\forall U \subset X$ abierto, componentes (arco)conexas de $U$ (respecto a la topología inducida) son abiertos en $X$.
\end{criterio}	

\begin{corollary}~
	\begin{enumerate}
		\item Si $X$ es localmente arcoconexo, componentes conexas son igual a componentes arcoconexas y viceversa;

		\item $X$ localmente arcoconexo y conexo $\implies X$ es arcoconexo.
	\end{enumerate}
\end{corollary}
