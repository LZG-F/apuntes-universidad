\section{Clase 13 (03/09): Acciones Topológicas (Lee p.77)}

\begin{eg}[última propiedad clase pasada, punto 2]
	Sea $\Gamma = \R,\ H = \Q \leadsto \overline{\Gamma} = \R / \Q$. Como $\Q$ no es cerrado en $\R$, entonces $\R / \Q$ no es Hausdorff. Además, veremos que la topología cociente en $\R / \Q$ es la indiscreta. Notar que $U \subset \R / \Q$ es abierto no vacío $\iff p^{-1}(U) \subset \R$ abierto no vacío. Tomar $U \subset \R / \Q$ no vacío, $\exists [x] = p(x) \in U (x \in \R) \implies p^{-1}(U)$ contiene a $x,y$; y de hecho contiene a $x$ tal que $\forall q \in \Q \ (p(x+q) = p(x))$. Por lo tanto, $p^{-1}(U)$ abierto (en $\R$) y $x+\Q \subset p^{-1}(U)$ (denso). $p^{-1}(U)$ es invariante por trasladar por $\Q$: si $y \in p^{-1}(U),\ q \in \Q \implies y+q \in p^{-1}(U)$. Como $x \in p^{-1}(U)$ abierto, entonces $\exists \varepsilon > 0$ tal que $(x-\varepsilon, x + \varepsilon) \subset p^{-1}(U)$. Luego, $(x - \varepsilon + q, x + \varepsilon + q) \subset p^{-1}(U)\ \forall q \in \Q$. En conclusión, $p^{-1}(U) = \R$ y $U = \R / \Q$, $\therefore$ la topología en $\R / \Q $ es $\{ \varnothing, \R / \Q \}$.
\end{eg}

\begin{eg}[Furstenberg]
	Se puede probar que existen infinitos primos de manera puramente topológica (usando topología profinita en $\Z$).
\end{eg}
\begin{proof}[Proof ]
	Base: $\mathcal{B} = \{ \underbrace{\overbrace{a\Z}^{\Gamma'} + b}_{b\Gamma'} \ \big| \ a \neq 0,\ b \in \Z \}$. Observar que, cada $a\Z + b$ es infinito. Esto implica que cada abierto con la topología profinita es o bien vacío, o infinito. En $\Z$, todo número no primo es o bien $1$ o $-1$, o $p\cdot a$ con $p$ primo y $a \in \Z$. Entonces, 
	\[ \Z = \{-1,1\} \ \sqcup \bigcup_{p\text{ primo}} p\Z \tag{$*$} \]
	Notar que cada $p\Z$ es cerrado:
	\[ \Z / p\Z = \bigcup_{1 \leq i \leq p-1}(p\Z + i) \]
	Si hubiera finitos primos, entonces la unión de los $p\Z$ en $(*)$ sería cerrado. Así, $\{-1,1\} \subset \Z$ abierto, lo que es una contradicción! 
\end{proof}

\subsection*{Acciones topológicas}

\textbf{Recuerdo.} $\Gamma \curvearrowright X:\ \Gamma \times X \to X$ tal que $(g,x) \mapsto g\cdot x$ y se cumple $(i)\ 1 \cdot x = x;\ (ii)\ h\cdot(g \cdot x) = (hg) \cdot x$.

\begin{definition}[acción continua]
	Una acción $\Gamma \curvearrowright X$ ($\Gamma $ grupo topológico, $gX$ espacio topológico) es continua si:
	\begin{align*}
		\Gamma \times X & \to X \\
		(g,x) & \mapsto g \cdot x
	\end{align*}
	es continuo.
\end{definition}

\begin{lemma}
	$\Gamma ,X$ grupos topológicos y $\Gamma \curvearrowright X$ acción.
	\begin{enumerate}
		\item Si $\Gamma \curvearrowright X$ continua, entonces $L_g: X \to X$, con $x \mapsto g\cdot x$, es homeomorfismo para cada $g \in \Gamma $,

		\item Si $\Gamma $ es grupo discreto y cada $L_g$ es homeomorfismo, entonces $\Gamma \curvearrowright X$ es continua.
	\end{enumerate}
\end{lemma}
\begin{proof}~
	\begin{enumerate}
		\item Dado $g \in \Gamma$:
		\begin{align*}
			X & \to \{g\} \times X \leftrightarrow \Gamma \times X \to X \\
			x & \mapsto \  (g,x) \quad \mapsto (g,x)  \ \mapsto g \cdot x
		.\end{align*}

		\item Tomar $U \subset X$ abierto. Notar que
		\begin{align*}
			p^{-1} &= \{(g,x) \ \big| \ g \cdot x \in U\} \\
			&= \{(g,x) \ \big| \ L_g(x) \in U \} \\
			&= \{(g,x) \ \big| \ x \in L_g^{-1}(U) \} \\
			&= \{(g,x) \ \big| \ x \in L_{g}^{-1}(U) \} \\
			&= \bigcup_{g\in\Gamma}\{g\} \times L_{g}^{-1}(U) 
		.\end{align*}
		Donde $\{g\}$ es abierto en $\Gamma$ (topología discreta) y $L_{g}^{-1}(U)$ es abierto en $X$ ($L_g$ homeo). Así, la unión es un abierto en $\Gamma \times X$. 
	\end{enumerate}
\end{proof}

\begin{eg}
	$GL_n(\R) \curvearrowright \R^n,\ A \cdot v = A(v)$ (multiplicación usual). Esta acción es continua!
	\begin{align*}
		Mat_{n \times n}(\R) \times \R^n & \to \R^n \quad [(A,v) \mapsto A(v)]\\
		\cup \quad \qquad \qquad & \\
		GL_n(\R) \times \R^n & \to \R^n \\
		(A,v) & \mapsto A(v)
	.\end{align*}
\end{eg}
