\section{Clase 15 (08/09): Conexidad (23, 24)}

\textbf{Recuerdo (TVI).} $f: [a,b] \to \R$ continua. Si $f(a) < 0$ y $f(b) > 0$, entonces $f(c) = 0$ para algún $c \in [a,b]$. \newline

\noindent \textbf{Conexidad.} Es una condición topológica en $X$ tal que $f: X \to \R$ cumple versión esperable del TVI!

\begin{definition}[separación y conexidad]
	$X$ espacio topológico.
	\begin{enumerate}
		\item[i.] Una separación de $X$ es $X = U \cup V$, con $U,V \subset X$ abiertos disjuntos, no vacíos;

		\item[ii.] $X$ es conexo si no tiene separación. Equivalentemente, $X = U \cup V,\ U,V \subset X$ abiertos disjuntos, entonces $\varnothing \in \{ U, V \}$.
	\end{enumerate}
\end{definition}
\smallskip
\begin{eg}[i.]
	$X = [0,1] \cup [2,3] \cup \{5\} \leadsto U = [0,1],\ V = [2,3] \cup \{5\}$ es separación.   
\end{eg}
\smallskip
\begin{eg}[ii.]
	$[0,1]$ es conexo!!! (Magia del axioma del supremo)
\end{eg}
\smallskip
\begin{remark}
	$X = U \cup V$ separación $\longleftrightarrow U \neq \varnothing$ clopen (abierto + cerrado) y $X \setminus U \neq \varnothing$.
\end{remark}

\begin{lemma}
	$X$ espacio topológico. $X$ conexo $\longleftrightarrow \forall f: X \to \R$ tal que $f(x) > 0,\ f(y) < 0$ para algún $x,y \in X \implies f(z) = 0$ para algún $z \in X$.
\end{lemma}
\medskip
\noindent \textbf{Propiedad ganadora.} Si $f: X \to Y$ continua. $X$ conexo $\implies f(X)$ conexo (respecto a la topología inducida).

\begin{corollary}
	Si $p:X \to A$ mapa cociente, $X$ conexo $\implies A$ conexo.
\end{corollary}

\begin{corollary}
	$X,Y$ espacios homeomorfos. $X$ conexo $\longleftrightarrow Y$ conexo.
\end{corollary}

\begin{proof}[Proof ][propiedad ganadora]
	Quremos $f(X)$ conexo (no hay separación). Suponer que $f(X) = U \cup V$ separación ($U,V \subset f(X)$ abiertos, disjuntos y no vacíos). Luego, $X = f^{-1}(f(X)) = f^{-1}(U) \cup f^{-1}(V)$ separación. Pero esto es una contradicción, pues $X$ es conexo!
\end{proof}

\begin{note}
	Se utilizó que la preimagen de abierto es abierto y que siguien siendo disjuntos los abiertos bajo la preimagen.
\end{note}

\begin{lemma}
	$Y \subset X$ espacios topológicos. $Y$ conexo $\longleftrightarrow \forall A,B \subset X$ abiertos tales que:
	\begin{enumerate}
		\item[i.] $Y \subset A \cup B$;

		\item[ii.] $Y \cap A \cap B = \varnothing$;
	\end{enumerate}
	$\implies Y \subset A$ ó $Y \subset B$.
\end{lemma}
\begin{criterio}[Conexidad]
$(Y_{\alpha})_{\alpha \in J}$ familia de subespacioes de $X$ tal que:
\begin{enumerate}
	\item Cada $Y_{\alpha}$ conexo;

	\item $\bigcap_{\alpha \in J} Y_{\alpha} \neq \varnothing$;
\end{enumerate}
$\implies Z = \bigcup_{\alpha \in J} Y_{\alpha}$ conexo.\par
\end{criterio}
\medskip
\begin{remark}
	$\bigcap_{\alpha \in J} Y_{\alpha}$ no necesariamente conexa si cada $Y_{\alpha}$ conexo.
\end{remark}
\smallskip
\begin{eg}~
	\begin{enumerate}
		\item $B = \{x \in \R^n \ \big| \ |x| \leq 1\}$ conexo. En efecto, si $v \in \mathbb{S}^{n-1} \leadsto Y_v = \{tv + (1-t)(-v) \ \big| \ t \in [0,1]\} \approx [0,1]$. Por lo tanto, cada $Y_v$ es conexo. Luego, $0 \in Y_v, \quad \forall v \in \mathbb{S}^{n-1} \implies B = \bigcup_{v \in \mathbb{S}^{n-1}} Y_v$ conexo;

		\item $\R$ es conexo. $\R = \bigcup_{\varepsilon>0} [-\varepsilon,\varepsilon],\ 0 \in [-\varepsilon,\varepsilon]\quad \forall \varepsilon < 0$;

		\item $\mathbb{S}^{n-1}$ conexo si $n\geq 2$ ($\mathbb{S}^{0} = \{-1,1\}$ no conexo (disconexo)). Para $n=2$, recordar que $[0,1] / \sim \to \mathbb{S}^{1}$ homeomorfismo. Por lo tanto, $\mathbb{S}^{1}$ conexo. Para $n$ arbitrario, sean $X = [0,1]^n,\ Y = \partial X \leadsto X / Y \stackrel{\sim}{\longrightarrow} \mathbb{S}^{n}$ homeomorfismo. Otra forma: sea $f: \R^n \setminus \{0\} \to \mathbb{S}^{n-1}$ tal que $v \mapsto \frac{v}{|v|}$ continua y sobre. Luego, es suficiente probar que $\R^n \setminus \{0\}$ conexo si $n \geq 2$.  
	\end{enumerate}
\end{eg}
