
\section{Clase 10 (27/08): Topología producto, Topología cuociente [19, 22]}

\begin{remark}~
	\begin{enumerate}
		\item $\mathcal{B}' \subset \mathcal{B}$;

		\item Si $J$ es finito, topología de cajas $=$ topología producto;

		\item Si $J$ es infinito, en general esto no es cierto.
	\end{enumerate}
\end{remark}

\begin{eg}
	Si $J = \Z^+,\ X_n = \R \ \forall n,\ Z = \prod_{n\geq 1} \R = \R^{\omega},\ f: \R \to \R^{\omega},\ t \mapsto (t,t,t,\dots)$.
\end{eg}

\noindent \textbf{Propiedad.} Si $Z = \prod_{\alpha \in J} X_{\alpha},\ f: Y \to Z \implies f$ está dada por $f(y) = (f_{\alpha}(y))_{\alpha \in J}$ con $f_{\alpha}: Y \to X_{\alpha}$. Con la topología producto, $f$ es continua $\iff$ cada $f_{\alpha}$ es continua. \newline

\noindent Antes de probar la propiedad, veremos que $f:\R \to \R^{\omega}$ no es continua para la topología de cajas: Tomar $B = \prod_{n\geq 1} \left( -\frac{1}{n}, \frac{1}{n}\right)$ es abierto para topología de cajas y $(0,0,0,\dots) = f(0) \in B$. Luego, $f^{-1}(B) = \{0\}$ no es abierto en $\R$. Por lo tanto, $f$ no es continua. \newline

\begin{proof}[Proof][Propiedad]~
	\Onlyifstep Notar que $f_{\alpha} = \pi_{\alpha} \circ f$ (con $\pi_{\alpha}$ la proyección: $Z\to X_{\alpha},\ (x_{\beta})_{\beta} \mapsto x_{\alpha}$) es composición de funciones continuas. Por lo tanto, es continua.\newline

	\noindent \Ifstep Tomar $B = \prod_{\alpha \in J} U_{\alpha}$ en base de topología producto. Luego, notamos
	\begin{align*}
		\prod_{\alpha \in J} U_{\alpha} & = U_{i_1} \times \cdots \times U_{i_n} \times \prod_{\alpha \in J \setminus \{i_1,\dots,i_n\}  } X_{\alpha} \quad \subset Z \\
		& = \bigcap_{j=1}^{n} \pi_{ij}^{-1} (U_{ij})
	\end{align*}
	Por lo tanto, suficiente probar que $f^{-1}(\pi_{\alpha}^{-1}(U_{\alpha}))$ abierto para cada $\alpha,\ \forall U_{\alpha} \subset X_{\alpha}$. Luego, $f^{-1}(\pi_{\alpha}^{-1}(U_{\alpha})) = f_{\alpha}^{-1}(U_{\alpha})$ es abierto porque $f_{\alpha}$ continua.
\end{proof}

\begin{eg}
	$Z = \{0,1\}^{\omega} = \{ \text{sucesiones } (x_1,x_2,\dots) \text{ con } x_i \in \{0,1\}  \}$.    
\end{eg}

\begin{lemma}
	Si $Z = \prod_{\alpha \in J} X_{\alpha}$ donde cada $X_{\alpha}$ tiene topología discreta. Entonces, topología de cajas en $Z$ es la topología discreta.
\end{lemma}

\begin{proof}[Proof ]
	Queremos $\{(x_{\alpha})_{\alpha}\}$ abierto en $Z$. Notar que $\{(x_{\alpha})_{\alpha}\} = \prod_{\alpha} \{x_{\alpha}\}$ es abierto en $Z$ con topología de cajas.   
\end{proof}

\noindent Con topología producto, $Z$ es \underline{homeomorfo} al conjunto de Cantor.\newline

\noindent \textbf{Recuerdo.} En $[0,1]$, $E_n =$ unión de intervalos $B_{i_1 \dots i_n}$ con $i_n \in \{0,1\}$ tal que, inductivamente, si $B_{i_{i_1\dots i_n}} = [a,b]$, entonces 
\[B_{i_1\dots i_n 0} = \left[ a, a+\frac{1}{3^{n+1}} \right],\quad B_{i_1\dots i_n 1} = \left[ b-\frac{1}{3^{n+1}}, b \right] \]
\noindent Luego, $\mathcal{C} = \bigcap_{n\geq 1} E_n$ (Cantor) (cerrado en $\R$, de interior vacío). Construir $f: \{0,1\}^{\Z^+} \to \mathcal{C},\ (x_n)_{n\geq 1} \mapsto \sum_{n\geq 1} \frac{2 x_n}{3^n}$, esto es biyección.\newline

\noindent Veamos que $f$ es continua: Notar que una base del $\mathcal{C}$ es el conjunto
\[ \mathcal{B} = \bigcup_{n \geq 1} \{B_{i_1\dots i_n} \cap \mathcal{C} \ \big| \ i_1,\dots,i_n \in \{0,1\}  \} \] 
Luego, 
\begin{align*}
	f^{-1}(B_{i_1\dots i_n} \cap \mathcal{C}) & = \{ (x_1,\dots,x_n,x_{n+1},\dots) \ \big| \ x_1 = i_1,\ x_2 = i_2, \dots, x_n = i_n \} \\
	& = \underbrace{\{i_1\} \times \{i_2\} \times \cdots \times \{i_n\} \times \{0,1\}^{\Z_{>n}}}_{\text{abierto para topología producto}}
\end{align*}

\noindent \textbf{Propiedades.} $Z = \prod_{\alpha \in J} X_{\alpha}$ espacio topológico.
\begin{enumerate}
	\item Si cada $X_{\alpha}$ es Hausdorff $\implies Z$ Hausdorff ($Z$ con topología producto ó con topología de cajas)

	\item Si $A_{\alpha} \subset X_{\alpha}$, donde $A = \prod_{\alpha \in J} A_{\alpha} \subset \prod_{\alpha \in J} X_{\alpha} = Z$. La topología producto en $A$ es la inducida por la producto en $Z$. Por otro lado, la topología de cajas de $A$ es la inducida por la topologia de cajas de $Z$ (demostrar!).
\end{enumerate}
