\clase{37}{7 de Noviembre}{}

\section{Funtorialidad (52) + $\pi_{1}(S^1)$ (54, 53)}

\subsection{Funciones Continuas y $\pi_{1}$}

\begin{notation}
	$f : (X, x_{0}) \to (Y, y_{0})$ es $f : X \to Y$ continua tal que $f(x_{0}) = y_{0}$ (mapeo punteado).
\end{notation}

\begin{property}
	Si $f : (X, x_{0}) \to (Y, y_{0})$ mapa punteado, entonces induce homomorfismo de grupos $f_{*} : \pi_{1}(X,x_{0}) \to \pi_{1}(Y,y_{0})$ tal que $[\alpha] \mapsto [f \circ \alpha]$, con $\alpha : [0,1] \to X \ (\in \Omega(X,x_{0}))$ y $f \circ \alpha : [0,1] \to Y \ (\in \Omega(Y,y_{0}))$.
\end{property}
\begin{proof}[Proof Other Information][propiedad]~
	\begin{enumerate}
		\item $f_{*}$ bien definido: Queremos $\alpha \sim_{p} \alpha' \implies f \circ \alpha \sim_{p} f \circ \alpha'$ (que viene de $\alpha \sim_{p} \alpha' \implies f \circ \alpha \sim f \circ \alpha'$ [Tarea 6] + se preservan puntos base).

		\item $f_{*}$ es homomorfismo: Queremos $f_{*}([\alpha] * [\beta]) = f_{*}([\alpha]) * f_{*}([\beta])$ (en $\pi_{1}(X,x_{0})$ y en $\pi_{1}(Y,y_{0})$, respectivamente). \par
		Para ello, notar que
		\[ f_{*}([\alpha] * [\beta]) = f_{*}([\alpha * \beta]) = [\underbrace{f \circ ( \alpha * \beta )}_{\gamma_{1}}] \]
		y
		\[ f_{*}([\alpha]) * f_{*}([\beta]) = [f \circ \alpha] * [f \circ \beta] = [\underbrace{(f \circ \alpha) * (f \circ \beta)}_{\gamma_{2}}] \]
		Luego, notar que
		\begin{align*}
			\gamma_{1}(s) = \gamma_{2}(s) = \begin{cases}
				f(\alpha(2s)) \quad & 0 \leq s \leq \frac{1}{2} \\
				f(\beta(2s - 1)) \quad & \frac{1}{2} \leq s \leq 1
			\end{cases} 
		\end{align*}
	\end{enumerate}
\end{proof}

\begin{remark}
	Construcción es "funtorial": $(X,x_{0}) \mapsto \pi_{1}(X,x_{0})$ con $f : (X,x_{0}) \to (Y,y_{0}) \mapsto f_{*} : \pi_{1}(X,x_{0}) \to \pi_{1}(Y,y_{0})$.
	\begin{enumerate}
		\item $\Id_{X} : (X,x_{0}) \to (X,x_{0}) \leadsto (\Id_{X})_{*} = \Id_{\pi_{1}(X,x_{0})} : \pi_{1}(X,x_{0}) \to \pi_{1}(X,x_{0})$.

		\item $f : (X,x_{0}) \to (Y,y_{0}),\ g : (Y,y_{0}) \to (Z,z_{0})$ mapas punteados ($\leadsto g \circ f : (X,x_{0}) \to (Z,z_{0})$) 
		\begin{align*}
			\implies (g \circ f)_{*} = g_{*} \circ f_{*} : \pi_{1}(X,x_{0}) \stackrel{f_{*}}{\to} \pi_{1}(Y,y_{0}) \stackrel{g_{*}}{\to} \pi_{1}(Z,z_{0})
		.\end{align*}
		Luego,
		\begin{align*}
			(g \circ f)_{*}([\alpha]) &= [(g \circ f) \circ \alpha] \\
			&= [g \circ (f \circ \alpha)] \\
			&= g_{*}[f \circ \alpha] \\
			&= g_{*}(f_{*}([\alpha]))
		.\end{align*}
	\end{enumerate}
\end{remark}

\begin{corollary}
	$X,Y$ arcoconexos, $x_{0} \in X,\ f : X \to Y$ homeomorfismo $\implies f_{*} : \pi_{1}(X,x_{0}) \to \pi_{1}(Y,f(x_{0}))$ es isomorfismo de grupos.
\end{corollary}

\begin{note}
	Si $X,Y$ arcoconexos, $X \approx Y \implies \pi_{1}(X) \cong \pi_{1}(Y)$.
\end{note}

\begin{proof}[Proof Other Information]
	$f$ homeomorfismo $\implies f^{-1} : (Y,y_{0}) \to (X,x_{0})$ continua. Afirmamos que $(f^{-1})_{*} : \pi_{1}(Y,y_{0}) \to \pi_{1}(X,x_{0})$ es inversa de $f_{*}$. En efecto, $f^{-1} \circ f = \Id_{X} \implies (f^{-1})_{*} \circ f_{*} = (\Id_{X})_{*} = \Id_{\pi_{1}(X,x_{0})}$.
\end{proof}

\begin{recordar}
	$f : X \to Y$ es biyección ssi $\exists g : Y \to X$ tal que $g \circ f = \Id_{X},\ f \circ g = \Id_{Y}$.
\end{recordar}

\subsection{El Círculo}
Veremos al círculo como
\[ S^1 = \{z \in \R^2 \ \big| \ |z| = 1\},\ x_{0} = 1. \] 
Loops en $S^1$ con base $x_{0}$:
\begin{itemize}
	\item $e_{x_{0}}$;

	\item $\omega_{1}(s) = e^{2 \pi i s}$;

	\item $\omega_{-1}(s) = e^{-2 \pi i s}$;

	\item $\omega_{2}(s) = e^{2 \cdot 2 \pi i s}$;
	\item $\leadsto \omega_{n} = e^{2 n \pi i s}$.
\end{itemize}

\begin{remark}
	$[\omega_{n}] = [\omega_{1}]^{*n}$ (i.e. $\omega_{1}$ genera).
\end{remark}

\begin{theorem}
	$\phi : \Z \to \pi_{1}(S^1,x_{0})$ tal que $n \mapsto [\omega_{n}]$ es isomorfismo de grupos.
\end{theorem}
