
\section{Ayudantía 21/08}

\begin{enumerate}
	\item Sea $f: [a,b] \to \R$ continua. Probar que su gráfico en $\R^2$:
	\[ \mathcal{G} \colon \{ (x,f(x)) \ : \ x \in [a,b] \} \subseteq \R^2 \]
	\noindent tiene medida nula.
	\begin{proof}[Proof ]
		Sea $\varepsilon > 0$, debemos contruir una familia de cuadrados en $\R^2$ que verifique $\mathcal{G} \subseteq  \displaystyle\bigcup_{i \in \N} Q_i$ y $\sum_{i \in \N} |Q_i| < \varepsilon$. Como $f: [a,b] \to \R$ es continua, entonces $f$ es uniformemente continnua. Así. si $\varepsilon,\ \exists \delta > 0 \ : \text{ si } |x-y| < \delta \implies |f(x)-f(y)| < \varepsilon$. Sin pérdida de generalidad, sea $S \leq 1$ y empezamos a particionar $[a,b]$. Sea $n \coloneq \lceil \frac{b-a}{\delta} \rceil$ y consideramos la partición $\{ I_j \}_{j=1}^{n}$ tales que $I_j \coloneq [x_{j},x_{j+1}]$ con $x_0 = a,\ x_n = b$ y $0<x_{j+1}-x_j\leq\delta$. Con esto, tenemos que:
		\begin{enumerate}
			\item Por construcción, $diam (I_j) \leq \delta,\ \forall 0 \leq j \leq n$;

			\item En particular, si $x \in I_j \implies I_j \subseteq B(x,\delta),\ \forall j \in \{1,\dots,n\}$. Luego, en cada $j \in \{1,\dots,n\}$ elegimos $x_j \in I_j$ y cumple que $I_j \subseteq B(x_j,\delta)$. Además, $f(I_j) \subseteq B(f(x_j),\varepsilon)$.
		\end{enumerate}
		\noindent Ahora definimos los cuadrados: Dado $(x,f(x)) \in \mathcal{G} \implies x \in I_j \subseteq B(x_j,\delta),\ \text{ para algún } j$ y $f(x) \in B(f(x_j),\varepsilon)$. Por lo tanto, $(x,f(x)) \in B(x_j, \delta) \times B(f(x),\varepsilon)$. Luego, $\mathcal{G} \subseteq \displaystyle{\bigcup_{j=1}^{n}} B(x_j,\delta) \times B(f(x_j),\varepsilon)$. Además:
		\begin{align*}
			\sum_{j=1}^{n} |B(x_j,\delta) \times B(f(x_j),\varepsilon)| & = n(2\delta)(2\varepsilon) = 4\varepsilon n \delta \\
			& = 4 \varepsilon \lceil \frac{b-a}{\delta} \rceil \delta \\
			& \leq 4 \varepsilon \left( \frac{b-a}{\delta} + 1 \right) \delta \\
			& = 4 \varepsilon ([b-a] + \delta) \\ 
			& \underbrace{\leq}_{\delta\leq 1} 4 \varepsilon (b-1 + 1)
		.\end{align*}
	\end{proof}

	\item Sean $\alpha \in (0,1]$ y $C_0 \coloneq [0,1]$. Para cada $n\geq 1$, defina recursivamente, el conjunto $C_n$ que resulta de retirar el intervalo central de largo $\alpha 3^{-n}$ a $C_{n-1}$. Por ejemplo, $C_1 \coloneq \left[ 0,\frac{3-2}{6} \right] \cup \left[ \frac{3+2}{6},1 \right]$. Defina $C_{\alpha} \coloneq \displaystyle\bigcap_{n\geq 0} C_n$.
	\begin{enumerate}
		\item Pruebe que $C_{\alpha}$ tiene medida nula $\iff \alpha = 1$. [Con esto (y mas resultados) $\chi_{C_{\alpha}}$ es R-integrable $\iff \alpha = 1$.]
	\end{enumerate}
	\begin{proof}[Proof ]
		Primero, estudiemos un poco mas de la construcción de los $C_{\alpha}$. Para construir $C_n$, debemos retirar $2^{n-1}$ intevalos de largo $\alpha 3^{-n}$. Así, si sumamos los largos de los intervalos retirados hasta $n$ obtenemos:
		\begin{align*}
			\sum_{k=1}^{n} 2^{k-1}(\alpha 3^{-k}) = \frac{\alpha}{2} \sum_{n=1}^{k} \left( \frac{2}{3} \right)^k \frac{\alpha}{2} \left( \sum_{k=0}^{n} \left( \frac{2}{3} \right)^k - 1 \right)
		.\end{align*}
		Por lo tanto, el largo neto al sustraer todos los intervalos es 
		\[ \lim_{n \to \infty} \sum_{k=1}^{n} 2^{k-1} (\alpha 3^{-k}) = \frac{\alpha}{2}  \lceil \frac{1}{1-\frac{2}{3}} -1 \rceil = \alpha. \]
		Supongamos que $\alpha < 1$. Supongamos que $\forall \varepsilon > 0,\ \exists \{ I_j \}_{j=1}^{\infty}$ tal que $C_{\alpha} \subseteq \displaystyle\bigcup_{j=1}^{\infty} I_j$ y $\sum_{j=1}^{\infty} |I_j| < \varepsilon$ (i.e., $C_{\alpha}$ tiene medida nula). En particular, si consideramos $\varepsilon \coloneq 1 \alpha > 0$, obtenemos un cubrimiento de $C_{\alpha},\ \{I_k\}_{k\in\N}\ : \ \sum_{k=1}^{\infty} |I_k| < 1 - \alpha$. Si ahora añadimos a esta colección todos los inrevalos sustraidos, entonces puedo cubrir $[0,1]$. Como el largo es numerablemente sub-aditivo, entonces $1 = |[0,1]| = | I_k \cup \{\text{lo que quite}\}|< (1-\alpha) + \alpha = 1$.      
	\end{proof}
\end{enumerate}
