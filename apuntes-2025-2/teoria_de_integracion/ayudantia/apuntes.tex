\documentclass{article}

\usepackage{amsmath, amsthm, amssymb, amsfonts}
\usepackage{thmtools}
\usepackage{graphicx}
\usepackage{setspace}
\usepackage{geometry}
\usepackage{float}
\usepackage{hyperref}
\usepackage[utf8]{inputenc}
\usepackage[english]{babel}
\usepackage{framed}
\usepackage[dvipsnames]{xcolor}
\usepackage{tcolorbox}

\colorlet{LightGray}{White!90!Periwinkle}
\colorlet{LightOrange}{Orange!15}
\colorlet{LightGreen}{Green!15}

\newcommand{\HRule}[1]{\rule{\linewidth}{#1}}

\declaretheoremstyle[name=Theorem,]{thmsty}
\declaretheorem[style=thmsty,numberwithin=section]{theorem}
\tcolorboxenvironment{theorem}{colback=LightGray}

\declaretheoremstyle[name=Proposition,]{prosty}
\declaretheorem[style=prosty,numberlike=theorem]{proposition}
\tcolorboxenvironment{proposition}{colback=LightOrange}

\declaretheoremstyle[name=Dem,]{prcpsty}
\declaretheorem[style=prcpsty,numberlike=theorem]{Dem}
\tcolorboxenvironment{Dem}{colback=LightGreen}

\setstretch{1.2}
\geometry{
    textheight=9in,
    textwidth=5.5in,
    top=1in,
    headheight=12pt,
    headsep=25pt,
    footskip=30pt
}

% ------------------------------------------------------------------------------

\begin{document}

% ------------------------------------------------------------------------------
% Cover Page and ToC
% ------------------------------------------------------------------------------

\title{ \normalsize \textsc{}
		\\ [2.0cm]
		\HRule{1.5pt} \\
		\LARGE \textbf{\uppercase{Template Title}
		\HRule{2.0pt} \\ [0.6cm] \LARGE{Subtitle} \vspace*{10\baselineskip}}
		}
\date{}
\author{\textbf{Author} \\ 
		Who? \\
		Where? \\
		When?}

\maketitle
\newpage

\tableofcontents
\newpage

% ------------------------------------------------------------------------------

\section{Ayudantia 14 de Agosto}

\section{Ejercicio 11 (Guia) (i)}  

\begin{Dem}
	$(A)$ Para ver que $C$ es cerrado, veremos que cada $C_n$ lo se. Notamos que si $f:\R \to \R, g:\R\to\R$ tales que $f(x)\coloneqq frac{1}{3} x$ y $g(x)\coloneqq frac{2}{3} + frac{1}{3}$ son continuas y $C_n=f(C_{n-1})\cup g(C_{n-1})\Rightarrow C_n$ es compacto $\Rightarrow$ es cerrado, $\forall n$ \\

	$(B)$ Para ver que es no numerable, vamos a construir una inyeccion $\Phi : X\to X$ con $X$ no numerable. Sea entonces $X\coloneqq {0,2}^{\N}$ y dado $w\in X$, definimos:

	\[
		C_n(w)\coloneqq\frac{C_0}{3^n} + \sum_{k=1}{n}\frac{w_k}{3^k}
	\]

	Si $n=2$: $C_2(w) = [0,\frac{1}{9}] + \frac{w_1}{3} + \frac{w_2}{9} = \begin{cases}
		[0,\frac{1}{9}] \\
		[\frac{2}{3},\frac{7}{9}] \\
		[\frac{2}{9},\frac{1}{3}] \\
		[\frac{8}{9},1]
	\end{cases}$ \\

	Basicamente, $C_n(w)$ referencia siempre a alguno de los $2^n$ intervalos de $C_n$. Luego, es claro que para $w$ fijo, $C_{n+1}(w)\subseteq C_n(w)\subseteq C_n (*)$ y $diam(C_n(w))\xrightarrow{n\to\infty} 0$. Por el Teorema de interseccion de Cantor: $|\cap_{n\in\N} C_n(w)|=1$. Sea $C(w)$ tal elemento. Luego, por $(*)$, $C(w)\in C$. \\

	Sea entonces $\Phi : {0,2}^{\N}\to C$ tal que $\Phi(w)\coloneqq C(w)$ y $\Phi$ es inyectiva (basta ver que pasa si $w^{(1)},w^{(2)}$ difieren en una coordenada). Como $|{0,2}^{\N}|=C$, se concluye.

	$(C)$ Si suponemos que existe $(a,b)\subset C$. SPG, $a=0$. Consideremos $n\in\N$ suficientemente grande.

	\[
	3^{-n}<b & \Rightarrow (0,b) \nsubseteq [0,\frac{1}{3^n}]\cup[\frac{2}{3^n},\frac{3}{3^n}]\subseteq C_n
	\]

	Luego, $\exist z \in (0,b): z\notin C_n, \forsome n \then z\notin C$ (Contradiccion).
\end{Dem}

\section{Ejercicio 11 (Guia) (ii)}  

\begin{Dem}
	Por $(i)$, sabemos que $C=\overline{C}=\partial C$. El resultado se sigue de lo siguiente: Si $(X,d)$ espacio metrico y $A\subseteq X \then D = \partial A$ (donde $D$ son los puntos de discontinuidad de $X_A$.

\end{Dem}

\section{Ejercicio 11 (Guia) (i)}  

\begin{Dem}
	Consideremos entonces $\mathds{1}_C$. Veamos sus integrales superior e inferior.

	\[
	\underline{\int_{0}^{1}} \mathds{1}_C(x) dx = sup \{L(P,\mathds{1}_C) \quad : \quad P \text{ particion de } [0,1] \}
	\]

	$L(P,\mathds{1}_C) = \sum m_i (x_{i+1}-x_i)=0 \text{ siempre, } \forall P$.\\

	Por lo tanto, $\underline{\int_{0}^{1}} \mathds{1}_C (x) dx =0 $.\\

	Ahora, para la integral superior:

	\[
	\overline{\int_{0}^{1}} \mathds{1}_C (x) dx = \inf \{U(P,mathds{1}_X) \quad : \quad P \text{ particion} \}
	\]
\end{Dem}
%\begin{theorem}
 %   This is a theorem.
%\end{theorem}

%\begin{proposition}
 %   This is a proposition.
%\end{proposition}

%\begin{principle}
 %   This is a principle.
%\end{principle}



% Maybe I need to add one more part: Examples.
% Set style and colour later.

%\subsection{Pictures}

%\begin{figure}[htbp]
 %   \center
 %   \includegraphics[scale=0.06]{img/photo.jpg}
 %   \caption{Sydney, NSW}
%\end{figure}

%\subsection{Citation}

%This is a citation\cite{Eg}.

\newpage

% ------------------------------------------------------------------------------
% Reference and Cited Works
% ------------------------------------------------------------------------------

\bibliographystyle{IEEEtran}
\bibliography{References.bib}

% ------------------------------------------------------------------------------

\end{document}
