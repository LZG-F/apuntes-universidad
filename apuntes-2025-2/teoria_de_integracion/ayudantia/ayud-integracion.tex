\documentclass[a4paper]{report}
\usepackage[utf8]{inputenc}
\usepackage[T1]{fontenc}
\usepackage{textcomp}

\usepackage{url}

% \usepackage{hyperref}
% \hypersetup{
%     colorlinks,
%     linkcolor={black},
%     citecolor={black},
%     urlcolor={blue!80!black}
% }

\usepackage{graphicx}
\usepackage{float}
\usepackage[usenames,dvipsnames]{xcolor}

% \usepackage{cmbright}

\usepackage{amsmath, amsfonts, mathtools, amsthm, amssymb}
\usepackage{mathrsfs}
\usepackage{cancel}

\newcommand\N{\ensuremath{\mathbb{N}}}
\newcommand\R{\ensuremath{\mathbb{R}}}
\newcommand\Z{\ensuremath{\mathbb{Z}}}
\renewcommand\O{\ensuremath{\emptyset}}
\newcommand\Q{\ensuremath{\mathbb{Q}}}
\newcommand\C{\ensuremath{\mathbb{C}}}
\let\implies\Rightarrow
\let\impliedby\Leftarrow
\let\iff\Leftrightarrow
\let\epsilon\varepsilon

% demostraciones bidireccionales

\newcommand{\Onlyifstep}{%
	\begingroup
	\fboxsep=1pt
	\raisebox{1.2ex}{\fbox{\raisebox{-1.2ex}{$\Rightarrow$\hspace{-0.05em}}}}%
	\endgroup
	\hspace{0.5em}%
	}
\newcommand{\Ifstep}{%
	\begingroup
	\fboxsep=1pt
	\raisebox{1.2ex}{\fbox{\raisebox{-1.2ex}{\hspace{-0.05ex}$\Leftarrow$}}}%
	\endgroup
	\hspace{0.5em}%
	}

% horizontal rule
\newcommand\hr{
    \noindent\rule[0.5ex]{\linewidth}{0.5pt}
}

\usepackage{tikz}
\usepackage{tikz-cd}

% theorems
\usepackage{thmtools}
\usepackage[framemethod=TikZ]{mdframed}
\mdfsetup{skipabove=1em,skipbelow=0em, innertopmargin=5pt, innerbottommargin=6pt}

\theoremstyle{definition}

\makeatletter

\declaretheoremstyle[headfont=\bfseries\sffamily, bodyfont=\normalfont, mdframed={ nobreak } ]{thmgreenbox}
\declaretheoremstyle[headfont=\bfseries\sffamily, bodyfont=\normalfont, mdframed={ nobreak } ]{thmredbox}
\declaretheoremstyle[headfont=\bfseries\sffamily, bodyfont=\normalfont]{thmbluebox}
\declaretheoremstyle[headfont=\bfseries\sffamily, bodyfont=\normalfont]{thmblueline}
\declaretheoremstyle[headfont=\bfseries\sffamily, bodyfont=\normalfont, numbered=no, mdframed={ rightline=false, topline=false, bottomline=false, }, qed=\qedsymbol ]{thmproofbox}
\declaretheoremstyle[headfont=\bfseries\sffamily, bodyfont=\normalfont, numbered=no, mdframed={ nobreak, rightline=false, topline=false, bottomline=false } ]{thmexplanationbox}


\declaretheorem[numberwithin=chapter, style=thmgreenbox, name=Definition]{definition}
\declaretheorem[sibling=definition, style=thmredbox, name=Corollary]{corollary}
\declaretheorem[sibling=definition, style=thmredbox, name=Proposition]{prop}
\declaretheorem[sibling=definition, style=thmredbox, name=Theorem]{theorem}
\declaretheorem[sibling=definition, style=thmredbox, name=Lemma]{lemma}



\declaretheorem[numbered=no, style=thmexplanationbox, name=Proof]{explanation}
\declaretheorem[numbered=no, style=thmproofbox, name=Proof]{replacementproof}
\declaretheorem[style=thmbluebox,  numbered=no, name=Exercise]{ex}
\declaretheorem[style=thmbluebox,  numbered=no, name=Example]{eg}
\declaretheorem[style=thmblueline, numbered=no, name=Remark]{remark}
\declaretheorem[style=thmblueline, numbered=no, name=Note]{note}

\renewenvironment{proof}[1][\proofname]{\begin{replacementproof}}{\end{replacementproof}}

\AtEndEnvironment{eg}{\null\hfill$\diamond$}%

\newtheorem*{uovt}{UOVT}
\newtheorem*{notation}{Notation}
\newtheorem*{previouslyseen}{As previously seen}
\newtheorem*{problem}{Problem}
\newtheorem*{observe}{Observe}
\newtheorem*{property}{Property}
\newtheorem*{intuition}{Intuition}


\usepackage{etoolbox}
\AtEndEnvironment{vb}{\null\hfill$\diamond$}%
\AtEndEnvironment{intermezzo}{\null\hfill$\diamond$}%




% http://tex.stackexchange.com/questions/22119/how-can-i-change-the-spacing-before-theorems-with-amsthm
% \def\thm@space@setup{%
%   \thm@preskip=\parskip \thm@postskip=0pt
% }

\usepackage{xifthen}

\def\testdateparts#1{\dateparts#1\relax}
\def\dateparts#1 #2 #3 #4 #5\relax{
    \marginpar{\small\textsf{\mbox{#1 #2 #3 #5}}}
}

\def\@lesson{}%
\newcommand{\lesson}[3]{
    \ifthenelse{\isempty{#3}}{%
        \def\@lesson{Lecture #1}%
    }{%
        \def\@lesson{Lecture #1: #3}%
    }%
    \subsection*{\@lesson}
    \testdateparts{#2}
}

% fancy headers
\usepackage{fancyhdr}
\pagestyle{fancy}

% \fancyhead[LE,RO]{Gilles Castel}
\fancyhead[RO,LE]{\@lesson}
\fancyhead[RE,LO]{}
\fancyfoot[LE,RO]{\thepage}
\fancyfoot[C]{\leftmark}
\renewcommand{\headrulewidth}{0pt}

\makeatother

% figure support (https://castel.dev/post/lecture-notes-2)
\usepackage{import}
\usepackage{xifthen}
\pdfminorversion=7
\usepackage{pdfpages}
\usepackage{transparent}
\newcommand{\incfig}[1]{%
    \def\svgwidth{\columnwidth}
    \import{./figures/}{#1.pdf_tex}
}

% %http://tex.stackexchange.com/questions/76273/multiple-pdfs-with-page-group-included-in-a-single-page-warning
\pdfsuppresswarningpagegroup=1

\author{Gilles Castel}

\DeclareMathOperator{\supp}{supp}
\DeclareMathOperator{\spann}{span}
\DeclareMathOperator{\Id}{Id}
\DeclareMathOperator{\Ker}{Ker}
\DeclareMathOperator{\im}{Im}
\DeclareMathOperator{\GL}{GL}
\DeclareMathOperator{\SL}{SL}
\DeclareMathOperator{\Mat}{Mat}

\title{Ayudantía Teoría de Integración}
\author{}

\begin{document}
\maketitle
\tableofcontents
    % start lessons

\section{Ayudantia 14 de Agosto}

\subsection{Ejercicio 11 (Guia) (i)}  

\begin{Dem}
	$(A)$ Para ver que $C$ es cerrado, veremos que cada $C_n$ lo se. Notamos que si $f:\R \to \R, g:\R\to\R$ tales que $f(x)\coloneqq frac{1}{3} x$ y $g(x)\coloneqq frac{2}{3} + frac{1}{3}$ son continuas y $C_n=f(C_{n-1})\cup g(C_{n-1})\Rightarrow C_n$ es compacto $\Rightarrow$ es cerrado, $\forall n$ \\

	$(B)$ Para ver que es no numerable, vamos a construir una inyeccion $\Phi : X\to X$ con $X$ no numerable. Sea entonces $X\coloneqq {0,2}^{\N}$ y dado $w\in X$, definimos:

	\[
		C_n(w)\coloneqq\frac{C_0}{3^n} + \sum_{k=1}{n}\frac{w_k}{3^k}
	\]

	Si $n=2$: $C_2(w) = [0,\frac{1}{9}] + \frac{w_1}{3} + \frac{w_2}{9} = \begin{cases}
		[0,\frac{1}{9}] \\
		[\frac{2}{3},\frac{7}{9}] \\
		[\frac{2}{9},\frac{1}{3}] \\
		[\frac{8}{9},1]
	\end{cases}$ \\

	Basicamente, $C_n(w)$ referencia siempre a alguno de los $2^n$ intervalos de $C_n$. Luego, es claro que para $w$ fijo, $C_{n+1}(w)\subseteq C_n(w)\subseteq C_n (*)$ y $diam(C_n(w))\xrightarrow{n\to\infty} 0$. Por el Teorema de interseccion de Cantor: $|\cap_{n\in\N} C_n(w)|=1$. Sea $C(w)$ tal elemento. Luego, por $(*)$, $C(w)\in C$. \\

	Sea entonces $\Phi : {0,2}^{\N}\to C$ tal que $\Phi(w)\coloneqq C(w)$ y $\Phi$ es inyectiva (basta ver que pasa si $w^{(1)},w^{(2)}$ difieren en una coordenada). Como $|{0,2}^{\N}|=C$, se concluye.

	$(C)$ Si suponemos que existe $(a,b)\subset C$. SPG, $a=0$. Consideremos $n\in\N$ suficientemente grande.

	\[
	3^{-n}<b & \Rightarrow (0,b) \nsubseteq [0,\frac{1}{3^n}]\cup[\frac{2}{3^n},\frac{3}{3^n}]\subseteq C_n
	\]

	Luego, $\exist z \in (0,b): z\notin C_n, \forsome n \then z\notin C$ (Contradiccion).
\end{Dem}

\subsection{Ejercicio 11 (Guia) (ii)}  

\begin{Dem}
	Por $(i)$, sabemos que $C=\overline{C}=\partial C$. El resultado se sigue de lo siguiente: Si $(X,d)$ espacio metrico y $A\subseteq X \then D = \partial A$ (donde $D$ son los puntos de discontinuidad de $X_A$.

\end{Dem}

\subsection{Ejercicio 11 (Guia) (iii)}  

\begin{Dem}
	Consideremos entonces $\mathds{1}_C$. Veamos sus integrales superior e inferior.

	\[
	\underline{\int_{0}^{1}} \mathds{1}_C(x) dx = sup \{L(P,\mathds{1}_C) \quad : \quad P \text{ particion de } [0,1] \}
	\]

	$L(P,\mathds{1}_C) = \sum m_i (x_{i+1}-x_i)=0 \text{ siempre, } \forall P$.\\

	Por lo tanto, $\underline{\int_{0}^{1}} \mathds{1}_C (x) dx =0 $.\\

	Ahora, para la integral superior:

	\[
	\overline{\int_{0}^{1}} \mathds{1}_C (x) dx = \inf \{U(P,mathds{1}_X) \quad : \quad P \text{ particion} \}
	\]
\end{Dem}

\end{document}
