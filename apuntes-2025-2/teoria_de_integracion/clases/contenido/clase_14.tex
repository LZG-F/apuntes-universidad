\clase{14}{8 de Septiembre}{}

\begin{enumerate}
	\item Construimos un conjunto $V \subseteq [0,1)$ tal que
	\begin{description}
		\item[(V1)] $(V + Q_{1}) \cap (V + Q_{2}) = \varnothing$, tal que $Q_{1}, Q_{2} \in \Q$ son distintos;

		\item[(V2)] $[0,1) \subseteq \bigcup_{Q \in \Q} (V + Q)$.
	\end{description}
	Cualquier conjunto $V \subseteq [0,1)$ que cumpla $V_{1}$ y $V_{2}$ se dice un cojunto de Vitali. Ningún conjunto de Vitali es medible Lebesgue.

	\item La misma demostración se puede adaptar para mostrar que:
	\begin{enumerate}[i.]
		\item Si $|E|_{e} > 0$ entonces existe $\widetilde{E} \subseteq E$ no medible Lebesgue;

		\item Si $\mu$ es una medida en $\R$ invariante por traslaciones definida sobre una $\sigma$-álgebra $\mcal{F}$ tal que $V \in \mcal{F}$ entonces
		\[ \mu([0,1)) = \begin{cases}
			0 \qquad (\implies \mu \equiv 0) \\
			\infty
		\end{cases} \]
		En particular, la noción de longitud no puede extenderse a todo $\mcal{P}(\R)$ (de forma invariante por traslación).

		\item $V \times [0,1]^{d-1} \not\in \mscr{L}(\R^d)$ para ningún $d > 1$.
	\end{enumerate}
\end{enumerate}

\begin{remark}
	La existencia de $V$ nos dice que $| \cdot |_e$ no es ni siquiera finitamente aditiva.
\end{remark}

\noindent \textbf{Paradoja de Banach-Tarski.} Si $A = B(0,1) \subseteq \R^d$, existe una partición finita de $A$,
\[ A = A_{1} \cupd A_{2} \cupd \cdots \cupd A_{k} \quad (\text{basta tomar } k = 6) \]
tal que sólo mediante rotaciones y traslaciones de los $A_{j}$ (operaciones que no cambian medida) se pueden obtener $2$ copias disjuntas de $A$.

\begin{definition}[$\pi$-sistema]
	Una clase de subcontuntos $\mcal{P}$ de un espacio $X$, se dice un $\pi$-sistema si es cerrado por intersecciones finitas, i.e.,
	\[ A,B \in \mcal{P} \Rightarrow{} A \cap B \in \mcal{P} \]
\end{definition}
\smallskip
\begin{eg}~
	\begin{itemize}
		\item Semiálgebra $\implies \pi$-sistema $\not\implies$ semiálgebra;
		
		\item $\mcal{P} = \{(-\infty,x] \ : \ x \in \R\}$ es un $\pi$-sistema pero no semiálgebra;

		\item $\mcal{P} \subseteq \widetilde{I}$ pero $\widetilde{I}$ no es una semiálgebra generada, aunque
		\[ \mcal{A}(\mcal{P}) = \mcal{P}(\widetilde{I}) \implies \sigma(\mcal{P}) = \beta(\R). \]
	\end{itemize}
\end{eg}
\begin{definition}[$\lambda$-sistema]
	Una clase $\mscr{L}$ de subconjuntos de un espacio $X$ se dice un $\lambda$-sistema si:
	\begin{description}
		\item[$(\lambda_{1})$] $X \in \mscr{L}$;

		\item[$(\lambda_{2})$] $A \in \mscr{L} \implies A^c \in \mscr{L}$;

		\item[$(\lambda_{3})$] $(A_n)_{n\in\N} \subseteq \mscr{L}$ disjuntos $\implies \textbigcupd_{n\in\N} A_n \in \mscr{L}$.
	\end{description}
\end{definition}

\begin{note}
	Tenemos que $\phi \in \mscr{L}$ y que, por ende $\mscr{L}$ es también cerrado por uniones disjuntas finitas.
\end{note}

\begin{eg}
	$\sigma$-álgebra $\implies \lambda$-sistema $\not\implies \sigma$-álgebra.
	\[ X = \{1,2,3,4\}, \ \mscr{L} \coloneq \{\varnothing, X, \{1,2\}, \{1,3\}, \{1,4\}, \{2,3\}, \{2,4\}, \{3,4\}\} \]
	$\mscr{L}$ es un $\lambda$-sistema, pero $\{1,2,3\} = \{1,2\} \cup \{2,3\} \not\in \mscr{L}$ y luego, $\mscr{L}$ no es $\sigma$-álgebra.
\end{eg}

\begin{theorem}[$\pi-\lambda$ de Dynkin]
	Si $\mscr{L}$ es un $\lambda$-sistema y $\mcal{P}$ es un $\pi$-sistema tal que $\mcal{P} \subseteq \mscr{L}$, entonces $\sigma(\mcal{P}) \subseteq \mscr{L}$.
\end{theorem}

\begin{lemma}
	Todo $\lambda$-sistema que sea también $\pi$-sistema es, de hecho, una $\sigma$-álgebra.
\end{lemma}

\begin{proof}[Proof ][lema]
	Debemos ver que si $(A_n)_{n\in\N} \subseteq \mscr{L}$, entonces $\bigcup_{n\in\N} A_n \in \mscr{L}$. Para ello, definimos para cada $n\in\N$,
	\[ A_{0} \coloneq \varnothing, \quad B_{n} \coloneq A_{n} \setminus (A_{1} \cup \cdots \cup A_{n-1}) = A_{n} \cap A_{1}^{c} \cap \cdots \cap A_{n-1}^{c} \]
	Notemos que:
	\begin{enumerate}
		\item $(B_n)_{n\in\N}$ son disjuntos y $\textbigcupd_{n\in\N} B_n = \bigcup_{n\in\N} A_n$;

		\item $B_n \in \mscr{L} \quad \forall n \in \N$, pues $\mscr{L}$ es un $\lambda$-sistema y $\pi$-sistema. Pero entonces,
		\[ \bigcup_{n\in\N} = \bigcupd_{n\in\N} B_n \in \mscr{L}. \]
	\end{enumerate}
\end{proof}

\begin{proof}[Proof ][Dynkin]
	Sea 
	\[ \lambda(\mcal{P}) \ceq \bigcap_{\substack{\widetilde{\mscr{L}} \ \lambda \text{-sistema} \\
	\mcal{P} \subseteq \widetilde{\mscr{L}}}} \widetilde{\mscr{L}} \]
	el $\lambda$-sistema generado por $\mcal{P}$. Observar que $\lambda(\mcal{P})$ es el menor $\lambda$-sistema que contiene a $\mcal{P}$. Luego, valen las inclusiones $\mcal{P} \subseteq \lambda(\mcal{P}) \subseteq \mscr{L}$. $(*)$ Si mostramos que $\lambda(\mcal{P})$ es un $\pi$-sistema, entonces, por el lema, $\lambda(\mcal{P})$ resulta una $\sigma$-álgebra (que contiene a $\mcal{P}$) y, por minimalidad, $\sigma(\mcal{P}) \subseteq \lambda(\mcal{P})$.
\end{proof}
