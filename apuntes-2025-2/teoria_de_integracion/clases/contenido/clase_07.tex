\clase{7}{22 de Agosto}{}
\medskip
\begin{definition}[semiálgebra]
	Sea $X$ un espacio y $\mathscr{C}$ una clase de subconjuntos de $X$. Decimos que $\mathscr{C}$ es una semiálgebra (de subconjuntos de $X$) si cumple:
	\begin{enumerate}
		\item $\varnothing \in \mathscr{C}$;

		\item ($\mathscr{C}$ es cerrada por intesecciones finitas) $A,B\in\mathscr{C} \implies A \cap B \in \mathscr{C}$;
		
		\item Si $A \in \mathscr{C}$, existen $C_1,\dots,C_n \in \mathscr{C}$ disjuntos tal que $A^c = \textbigcupd_{i=1}^{n} C_i$.
	\end{enumerate}
\end{definition}
\smallskip	
\begin{eg}~
	\begin{enumerate}
		\item La clase $\mathcal{I}_d$ de intervalos en $\R^d$ es una semiálgebra.

		\item La clase $\widetilde{\mathcal{I}} \coloneq \{ (a,b] \cap \R \ : \ -\infty \leq a \leq b \leq \infty \}$ es una semiálgebra.

		\item Si $X$ e $Y$ son espacios y $\mathscr{C}_X, \mathscr{C}_Y$ son semiálgebras en $X$ e $Y$ respectivamente, entonces
		\[ \mathscr{C}_X \times \mathscr{C}_Y \coloneq \{ F \times G \ : \ F \in \mathscr{C}_X,\ G \in \mathscr{C}_Y \} \]
		es una semiálgebra en $X \times Y$, llamada "semiálgebra producto".
	\end{enumerate}
\end{eg}

\begin{definition}[álgebra]
	Sean $X$ un espacio y $\mathscr{A}$ una clase de subconjuntos de $X$. Decimos que $\mathscr{A}$ es un álgebra (de subconjuntos de $X$) si cumple que:
	\begin{enumerate}
		\item[(i)] $\varnothing \in \mathscr{A}$;
		
		\item[(ii)] $\mathscr{A}$ es cerrado por intersecciones finitas;

		\item[(iii)] ($\mathscr{A}$ es cerrada por complementos) $A \in \mathscr{A} \implies A^c \in \mathscr{A}$. 
	\end{enumerate}	
		\noindent Equivalentemente, en presencia de (iii), (ii) se puede reemplazar por:	
	\begin{enumerate}
		\item[(ii')] ($\mathscr{A}$ es cerrada por uniones finitas) $A,B\in \mathscr{A} \implies A \cup B \in \mathscr{A}$. (\textbf{Dem:} Ejercicio!)
	\end{enumerate}
\end{definition}

\begin{eg}~
	\begin{enumerate}
		\item $X$ espacio, $\mathscr{A}_1 \coloneq \{\varnothing, X\},\ \mathscr{A}_2 \coloneq \mathcal{P}(X)$ son álgebras (donde $\mathscr{A}$ es llamada el álgebra trivial);

		\item Sea $\mathscr{S}$ una semiálgebra de subconjuntos de un espacio $X$. Entonces 
		\[ \mathscr{A} \coloneq \left\{ E \subseteq X \ : \ \exists S_1,\dots,S_n \in \mathscr{S} \text{ disjuntos tal que } E = \displaystyle\bigcupd_{i=1}^{n} S_i \right\} \] 
		es un álgebra, llamada el álgebra generada por $\mathscr{S}$. Notemos que $\mathscr{A}(\mathscr{S})$ es el menor álgebra que contiene a $\mathscr{S}$:
		\begin{enumerate}
			\item[(i)] $\mathscr{A}(\mathscr{S})$ es un álgebra y $\mathscr{S} \subseteq \mathscr{A}(\mathscr{S})$;

			\item[(ii)] Si $\mathscr{A}'$ es un álgebra con $\mathscr{S} \subseteq \mathscr{A}'$ entonces $\mathscr{A}(\mathscr{S}) \subseteq \mathscr{A}'$.
		\end{enumerate}
	\end{enumerate}
\end{eg}
\begin{note}
	Toda álgebra es una semiálgebra.
\end{note}

\begin{definition}[$\sigma$-álgebra]
	Una clase (no vacía) $\mathscr{M}$ de subconjuntos de un espacio $X$ se dice una $\sigma$-álgebra si cumple:
	\begin{enumerate}
		\item $\varnothing \in \mathscr{M}$;

		\item $E \in \mathscr{M} \implies E^c \in \mathscr{M}$;

		\item $(E_n)_{n\in\N} \subseteq \mathscr{M} \implies \bigcup_{n\in\N} E_n \in \mathscr{M}$.
	\end{enumerate}
	\noindent Llamamos al par $(X,\mathscr{M})$ un \underline{espacio medible} y a los elementos de $\mathscr{M}$, \underline{conjuntos medibles}.
\end{definition}

\begin{note}~
	\begin{enumerate}
		\item Todo $\sigma$-álgebra es un álgebra;

		\item Equivalentemente, en presencia de (1), (3) se puede reemplazar por
		\begin{enumerate}
			\item[(3'.)] $(E_n)_{n\in\N} \subseteq \mathscr{M} \implies \bigcap_{n\in\N} E_n \in \mathscr{M}$.
		\end{enumerate}
	\end{enumerate}
\end{note}

\begin{eg}~
	\begin{enumerate}
		\item $\sigma$-álgebra $\implies$ álgebra $\implies$ semiálgebra (no valen las recíprocas);

		\item $\{\varnothing,X\},\mathcal{P}(X)$ son $\sigma$-álgebras;

		\item Si $(\mathscr{M}_{\gamma})_{\gamma \in \Gamma}$ son $\sigma$-álgebras, entonces
		\[ \displaystyle\bigcap_{\gamma\in\Gamma} \mathscr{M}_{\gamma} \coloneq \{ E \subseteq X \ : \ E \in \mathscr{M}_{\gamma},\ \forall \gamma \in \Gamma \} \]
		es una $\sigma$-álgebra.
		
		\item Si $\mathscr{M}$ es una clase de subconjuntos de $X$, entonces
		\[ \sigma(\mathscr{M}) \coloneq \displaystyle\bigcap_{\begin{aligned}
			\mathscr{M}& \ \sigma\text{-álgebra} \\
			& \mathscr{C}\subseteq\mathscr{M}
		\end{aligned}} \mathscr{M} \]
		es la $\sigma$-álgebra generada por $\mathscr{M}$. De hecho, $\sigma(\mathscr{M})$ es la menor $\sigma$-álgebra que contiene a $\mathscr{C}$:
		\begin{enumerate}
			\item $\sigma(\mathscr{C})$ es $\sigma$-álgebra y $\mathscr{C} \subseteq \sigma(\mathscr{C})$;

			\item Si $\mathscr{F}$ es $\sigma$-álgebra y $\mathscr{C} \subset \mathscr{F}$ entonces $\sigma(\mathscr{C}) \subseteq \mathscr{F}$.
		\end{enumerate}

		\item Si $(X,\mathscr{T})$ es un espacio topológico, $\sigma(\mathscr{T})$ se conoce como la \underline{$\sigma$-álgebra} \underline{de Borel}, y sus elementos se llaman \underline{Borelianos}. La notamos $\beta(X)\ (= \sigma(\mathscr{T}))$.
	\end{enumerate}
\end{eg}
	
\begin{eg}
	$\beta(\R)$ contiene a todos los abiertos, cerrados, intervalos, conjuntos de tipo $G_{\delta}$ y $F_{\sigma},\dots$ De hecho, $\beta(\R)=\sigma(\text{cerrados})=\sigma(\text{compactos})=\sigma(\mathcal{I})=\sigma(\widetilde{\mathcal{I}})$.
\end{eg}

\begin{definition}
	Sea $\mathscr{C}$ una clase (no vacía) de subconjuntos de $X$ y $\mu : \mathscr{C} \to [0,\infty]$ una función (la llamamos una \underline{función de conjuntos}). Diremos que:
	\begin{enumerate}
		\item[(i)] \textbf{$\mu$ es monótona} (en $\mathscr{C}$) si $A,B\in\mathscr{C},\ A \subseteq B \implies \mu(A)\leq\mu(B)$;

		\item[(ii)] \textbf{$\mu$ es finitamente aditiva} si $(A_i)_{i=1,\dots,n} \subseteq \mathscr{C}$, entonces
		\[ \mu(\bigcupd_{i=1}^{n} A_i) = \sum_{i=1}^{n} \mu(A_i); \]

		\item[(iii)] \textbf{$\mu$ es $\sigma$-aditiva} si $(A_n)_{n\in\N} \subseteq \mathscr{C}$ disjuntos, entonces
		\[ \mu (\bigcupd_{i=1}^{\infty} A_i ) = \sum_{i=1}^{\infty} \mu(A_i); \]

		\item[(iv)] \textbf{$\mu$ es $\sigma$-subaditiva} si $\mu(A)\leq \sum_{i=1}^{\infty}\mu(A_n)$, para todo $A \in \mathscr{C}$ y $(A_n)_{n\in\N}\subseteq\mathscr{C}$ tal que $A \subseteq \bigcup_{n\in\N} A_n$
	\end{enumerate}
\end{definition}
