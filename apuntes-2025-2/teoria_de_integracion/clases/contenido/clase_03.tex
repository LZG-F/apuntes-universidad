
	\section{Clase 3 (07/08)}

	\begin{note}
		Las integrales en el sentido de Darboux y el de Riemann coinciden.
	\end{note}

	\begin{prop}
		Si $f:[a,b]\to\R$ es monótona, entonces es Riemann integrable.
	\end{prop}

	\begin{remark}
		Una función monótona tiene discontinuidades numerables.
	\end{remark}

	\begin{prop}
		Si $f:[a,b]\to\R$ es continua, entonces es Riemann integrable.
	\end{prop}

	En particular, existen funciones Riemann integrables con numerables discontinuiodades. De hecho, hay ejemplos con $c$ (cardinal del continuo) discontinuidades. No obstante, si $f$ es integral de Riemann, su conjunto de discontinuidades tiene que ser "pequeño".

	\begin{theorem}
		Sea $f:[a,b]\to\R$ acotada. Entonces, $f$ es integral de Riemann si y sólo si su conjunto de discontinuidades tiene medida nula.
	\end{theorem}

	\begin{definition}[intervalo]
		Decimos que un conjunto $I\subseteq\overline{\R}\coloneq\R\cup \{-\infty,\infty\}$ es un intervalo si satisface

		\[
		x,y\in I \Rightarrow z\in I \text{ para todo } \min x,y\leq z\leq\max x,y.
		\]
	\end{definition}

	\begin{eg}
		(y propiedades)
		\begin{itemize}
			\item Dados $a\leq b$ ($a,b\in\R$), los conjuntos $(a,b),(a,b],[a,b],[a,b)$ son intervalos;

			\item El conjunto vacío es un intervalo ($\varnothing = (a,a)$);

			\item Los puntos son intervalos. $I = [\lambda,\lambda]$;

			\item La intersección son intervalos de intervalos.
		\end{itemize}
	\end{eg}

	\begin{definition}[intervalo generalizado]
		Decimos que un conjunto $I\subseteq\R^d$ es un intervalo si puede escribirse como

		\[
		I=\prod_{k=1}^{d} I_k
		\]

		donde cada $I_r$ es un intervalo en $\R$. La medida de un intervalo $I\subseteq\R^d$ se define como
		
		\[
		|I|\coloneq\prod_{k=1}^{d} |I_k|.
		\]
	\end{definition}

	\begin{note}
		Los intervalos en $\R^d$ heredan las mismas pripiedades en $\R$:

		\begin{itemize}
			\item Intersección de intervalos en $\R^d$ es intervalo.

			\item Si $I\subseteq J\subseteq\R^d$ son intervalos, entonces $|I|\leq|J|$.
		\end{itemize}
	\end{note}

	\begin{definition}[medida nula]
		Un conjunto $E\subseteq\R^d$ se dice de medida nula si, dado $\varepsilon>0$, existe una sucesión $(I_n)_{n\in\N}$ de intervalos de $\R^d$ tal que

		\[
		E\subseteq\bigcup_{n\in\N} I_n\quad\text{ y }\quad \sum_{n\in\N}^{}|I_n|<\varepsilon.
		\]
	\end{definition}

	\begin{eg}
		(y propiedades)
		\begin{enumerate}
			\item Todo conjunto unitario $\{x\}, (x\in\R^d)$ tiene medida nula;

			\item Toda unión numerable de conjuntos de medida nula tiene medida nula;

			\item Cualquier conjunto numerable tiene medida nula;
			
			\item Cualquier subconjunto de un conjunto de medida nula tiene medida nula;

			\item Existen conjuntos no numerables de medida nula:

			\begin{itemize}
				\item En $\R^d$ con $d\geq 2$, los ejes $\{x:x_1=0\}, i=1,\dots,d$ tiene medida nula.

				\item En $\R$, el conjunto de cantor tiene medida nula.
			\end{itemize}

			\item $E\subseteq\R^d$ es de medida nula, entonces $\alpha\dot E$ tiene medida nula $\forall\alpha\in\R$.
			
			\item $E\subseteq\R^d$ es de medida nula, entonces $E + v$ tiene medida nula $\forall v\in\R^d$.

			\item Si $E$ contiene un intervalo no unitario, entonces no tiene medida nula. Notar que:

			\begin{itemize}
				\item La vuelta no es válida: $\R\textbackslash\Q$ no contiene untervalos no unitarios pero no puede tener medida nula.

				\item De esto se deduce que si $E\subseteq\R^d$ tiene medida nula. Entonces $E^c$ es denso (no vale la vuelta: $E^c=\Q$).
			\end{itemize}

			\item $E\subseteq\R^d$ tiene medida nula si y sólo si

			\[
			|E|_e\coloneq\inf \{ \sum_{n\in\N} |I_n| : E \subseteq \bigcup_{n\in\N} I_n \} = 0, \quad I_n \text{ intervalo } \forall n \in \N.
			\]
		\end{enumerate}
	\end{eg}
