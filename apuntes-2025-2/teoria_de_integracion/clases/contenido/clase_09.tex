\section{Clase 9 (27/08)}

\begin{remark}
	Si $\tau : \mathscr{S} \to [0,\infty]$ es $\sigma$-aditiva en $\mathscr{S}$ y $\mathscr{S}$ es una semiálgebra, entonces $\tau$ es extendible.
\end{remark}

\begin{remark}
	La extensión puede no ser única si $\tau$ no es $\sigma$-finita.
\end{remark}
\begin{eg}
	$\widetilde{\mathcal{I}}_{\Q} \coloneq \widetilde{\mathcal{I}} \cap \Q = \{ (a,b] \cap \Q \ : \ -\infty \leq a \leq b \leq \infty \}$ 
\end{eg}

\begin{note}~
	\begin{itemize}
		\item $\widetilde{\mathcal{I}}_{\Q}$ es una semiálgebra;

		\item $\sigma (\widetilde{\mathcal{I}}_{\Q}) = \sigma (\widetilde{\mathcal{I}} \cap \Q) \stackrel{\text{Ej!}}{=} \sigma (\widetilde{\mathcal{I}}) \cap \Q = \beta(\R) \cap \Q = \mathcal{P}(\Q)$ (9.52)

	\item $\tau : \widetilde{\mathcal{I}}_{\Q} \to [0,\infty]$, dada por $\tau(A) \coloneq \begin{cases}
			0 \quad A = \varnothing \\
			\infty \quad A \neq \varnothing,\ A \in \widetilde{\mathcal{I}}_{\Q}
		\end{cases}$ (Observar que $\tau$ no es $\sigma$-finita)

		\item Para cada $r > 0,\ \mu_r : \mathcal{P}(\Q) \to [0,\infty]$ dada por $\mu_r(A) \coloneq r(\# A)$ es una extensión de $\tau$ (y es una medida)
	\end{itemize}
\end{note}

\begin{definition}[espacio completo y conjuntos $\mu$-nulos]
	Sea $(X,\mathscr{M},\mu)$ un EdM y definamos 
	\[ \mathscr{N}_{\mu} \coloneq \{ E \subset X \ : \ \exists N \in \mathscr{M} \text{ con } E \subseteq N \text{ y } \mu(N) = 0 \} \] 
	Los elementos de $\mathscr{N}_{\mu}$ se dicen \underline{conjuntos $\mu$-nulos}. Diremos que $(X,\mathscr{M},\mu)$ es \underline{completo} si $\mathscr{N}_{\mu} \subseteq \mathscr{M}$
\end{definition}

\begin{remark}
	$(X,\overline{\sigma(\mathscr{S})},\overline{\mu_{\delta}})$ es \underline{completo}. En efecto, $\mathscr{N}_{\overline{\mu_{\delta}}}$ corresponde al subconjunto de $\overline{\sigma(\mathscr{S})}$ que se obtiene tomando $B=\varnothing$.
\end{remark}

\begin{remark}
	Veremos más adelante que las siguientes premedidas son UE:

	\begin{itemize}
		\item[(i)] Premedidas de Lebesgue-Stieltjes (en particular, la función longitud $\lambda$ (sobre $\widetilde{\mathcal{I}}$) y las premedidas de probabilidad).

		\item[(ii)] Premedidas de Lebesgue en $\R^d$, con $d \in \N$.
	\end{itemize}
\end{remark}

\noindent En particular;

\begin{corollary}
	Para cada función $F$ de Lebesgue-Stieltjes, existe una $\sigma$-álgebra $\mathscr{M}_F$ sobre $\R$ y una única medida $\mu_F$ en $(\R,\mathscr{M}_F)$ tal que
	\[ \mu_F = (I(a,b)) = F(b) - F(a) \quad \forall -\infty \leq a \leq b \leq \infty \]
	Además, $\beta(\R) \subseteq \mathscr{M}_F$. Es decir, $\mu_F$ es una medida que extiende a $\tau_F$, a todo $\mathscr{M}_F$ (y en particular, a todo $\beta(\R)$). Además, $(\R, \mathscr{M}_F, \mu_F)$ es un EdM completo. ($\mathscr{M}_F \coloneq \overline{\sigma(\widetilde{\mathcal{I}})^F},\ \mu_F \coloneq \overline{\mu_{\tau_F}}$). La medida $\mu_F$ se conoce como \underline{medida de L-S asociada a $F$}. En particular, para cualquier función de distribución $F$, existe una única medida de probabilidad $\mathbb{P}_F$ en $(\R,\beta(\R))$ tal que 
	\[ \mathbb{P}_F(I(a,b)) = F(b) - F(a) \quad \forall -\infty \leq a \leq b \leq \infty \]
	(En la guía 3 veremos que $F \to \mathbb{P}_F$ es una biyección)
\end{corollary}

\begin{note}
	Los $\beta$ son los Borelianos y $I(a,b) = (a,b] \cap \R$. (super $F \to$ 10.26).
\end{note}

\begin{eg}[Importante!]
	\textbf{Medida de Lebesgue en $\R$.} Tomando $F = id$ en el Corolario anterior, obtenemos una $\sigma$-álgebra $\mathscr{L}(\R) \coloneq \mathscr{M}_{id}$ con $\beta(\R) \subseteq \mathscr{L}(\R)$ y una medida $\mu_{id}$ en $(\R,\mathscr{L}(\R))$ tal que $\mu_{id}(I(a,b)) = b-a \quad \forall -\infty \leq a \leq b \leq \infty$. En particular, de esto se deduce que $\mu_{id}(I) = |I|\quad \forall I \in \mathcal{I}$. Dicha medida recibe el nombre de \underline{medida de Lebesgue} (en $\R$), y los elementos de $\mathscr{L}(\R)$ se dicen \underline{conjuntos medibles Lebesgue}. Adoptaremos la notación $\mu_{id}(E) \coloneq \lambda(E) \coloneq |E|$. La medida $\mu_{id}$ \underline{es} la extensión de la noción de longitud que buscábamos y $\mathscr{L}(\R)$ son los conjuntos cuya "longitud" podremos medir. Además, los conjuntos de medida nula (de la guía 2), son \underline{exactamente} aquellos $A \in \mathscr{L}(\R)$ tal que $\mu_{id}(A) = 0$ (lo veremos más adelante!).
\end{eg}

\begin{eg}[Medida de Lebesgue en $\R^d$]
	Si $\mathcal{I}_d$ son los intervalos en $\R^d$ y definimos $\tau:\mathcal{I}_d \to [0,\infty]$ como $\tau(I)\coloneq|I|$, entonces $\mathcal{I}_d$ es una semiálgebra y $\tau$ es una premedida $\sigma$-aditiva en $\mathcal{I}_d$ (lo veremos después). Por lo tanto, $\tau$ se puede extender (de manera única, pues $\tau$ es $\sigma$-finita) a una medida $\mu_{\delta}$ sobre la $\sigma$-álgebra $\mathscr{L}(\R^d) = \overline{\sigma(\mathcal{I}_d)^{\tau}}$, llamada \underline{medida de Lebesgue en $\R^d$} y $\mathscr{L}(\R^d)$ es la clase de \underline{conjuntos medibles Lebesgue en $\R^d$}. Al igual que antes, dado $E \in \mathscr{L}(\R^d)$, notamos $|E| \coloneq \mu_{\tau}(E)$.
\end{eg}
