\clase{13}{5 de Septiembre}{}
\medskip
\begin{proof}[Proof ][Continuación clase anterior]
	\text{} \par
	$\boxed{(1) \implies (2)}$ Si $\mu_{\tau}^{*}(A) < \infty$ (LISTO!). Si $\mu_{\tau}^{*}(A) = \infty$, tomamos $(E_n)_{n\in\N} \subseteq \mathscr{S}$ disjuntos tal que $X = \textbigcupd_{n\in\N} E_n$ y $\tau(E_n) < \infty$. Luego, $\mu_{\tau}^{*}(A \cap E_n) \leq \mu_{\tau}^{*} (E_n) = \tau(E_n) < \infty$ (La igualdad es, pues $E_n \in \mathscr{S}, \tau \ \sigma$-sub). Por ende, por lo ya probado, existe $B_n \in \sigma(\mathscr{S})$ tal que $A \cap E_n \subseteq B_n$ y $\mu_{\tau}(B_n) = \mu_{\tau}(A \cap E_n) < \infty$. Luego, $B \coloneq \bigcup_{n\in\N} B_n$ y $N \coloneq B-A$ cumplen lo pedido pues $A = \textbigcupd_{n\in\N} A \cap E_n \subseteq \bigcup_{n\in\N} B_n = B$ y
	\begin{align*}
		\mu_{\tau}(B\setminus A) = \mu_{\tau}\left( \bigcup_{n\in\N} B_n \setminus A \right) & \leq \sum_{n\in\N} \mu_{\tau}(B_n \setminus A) \\
		& \leq \sum_{n\in\N} \mu_{\tau}(B_n \setminus (A \cap E_n)) \\
		&= \sum_{n\in\N} \underbrace{\mu_{\tau}(B_n) - \mu_{\tau}(A \cap E_n)}_{0} = 0 
	.\end{align*}

	$\boxed{(2) \implies (3)}$ Notemos que por $(2) \implies (1),\ A \in \mathscr{M}_{\mu_{\tau}^{*}}$ si vale $(2)$. En particular, $A^c \in \mathscr{M}_{\mu_{\tau}^{*}}$. Luego, por $(1) \implies (2)$ para $A^c,\ \exists \widetilde{B} \in \sigma(\mathscr{S})$ y $N_2 \in \mathscr{M}_{\mu_{\tau}^{*}}$ con $\mu_{\tau}(N_{2}) = 0$ tal que $A^c = \widetilde{B} - N_{2}$. Pero entonces, tomando $C \coloneq \widetilde{B}^c$, vemos que $C \in \sigma(\mathscr{S})$ y $A = (A^c)^c = (\widetilde{B} \cap N_{2}^c)^c = \widetilde{B}^c \cup (N_{2}^c)^c = C \cup N_2$.
\end{proof}
\medskip
\begin{remark}
	$\mathscr{M}_{\mu_{\tau}^{*}} = \overline{\sigma(\mathscr{S})}$ (con resp. a $\mu_{\tau}^{*}|_{\sigma(\mathscr{S})}$). En efecto, si $A \in \mathscr{M}_{\mu_{\tau}^{*}}$ entonces, por $(1) \implies (3)$, existen $C \in \sigma(\mathscr{S})$ y $N \in \mathscr{M}_{\mu_{\tau}^{*}}$ tal que $A = C \cup N$ y $\mu_{\tau}^{*}(N) = 0$. Como $N \in \mathscr{M}_{\mu^{*}}$, por $(1) \implies (2)$ para $N$, existe $\widetilde{N} \in \sigma(\mathscr{S})$ tal que $N \subseteq \widetilde{N}$ y $0 = \mu_{\tau}(N) = \mu_{\tau}(\widetilde{N})$. Luego, $N$ resulta $\mu_{\tau}^{*}\big{|}_{\sigma(\mathscr{S})}$-nulo y, por lo tanto, $N \in \overline{\sigma(\mathscr{S})^{\mu_{\tau}^{*}|_{\sigma(\mathscr{S})}}}$. \par
	
	Por otro lado, si $A \in \overline{\sigma(\mathscr{S})}$ (resp. a $\mu_{\tau}^{*}|_{\sigma(\mathscr{S})}$), entonces $A = B \cup N$ donde $B \in \sigma(\mathscr{S})$ y $\exists \widetilde{N} \in \sigma(\mathscr{S})$ tal que $N \subseteq \widetilde{N}$ y $\mu_{\tau}^{*}(N) = 0$, y entonces $A = B \cup N \in \mathscr{M}_{\mu_{\tau}^{*}}$ (pues $\sigma(\mathscr{S}) \subseteq \mathscr{M}_{\mu_{\tau}^{*}}$).
\end{remark}
\medskip
\begin{remark}
	En particular, hemos probado:
\end{remark}
\begin{prop}
	Si $\tau$ es una premedida UE sobre una semiálgebra $\mathscr{S}$ entonces, dado $A \subseteq X$ (no necesariamente $\mu_{\tau}^{*}$-medible),
	\begin{align*}
		\mu_{\tau}^{*}(A) & \coloneq \min \{ \mu_{\tau}(B) \ \big| \ B \in \sigma(\mathscr{S}),\ A \subseteq B \} \\
		& = \max \{ \mu_{\tau}(C) \ \big| \ C \in \sigma(\mathscr{S}),\ C \subseteq A \}  
	.\end{align*}
\end{prop}

\begin{theorem}
	$\beta(\R^d) \subsetneq \mathcal{L}(\R^d) \subsetneq \mathcal{P}(\R^d)$. De hecho, $\# \mathscr{L}(\R^d) = 2^c,\ \# \mathcal{P}(\R^d) \setminus \mathscr{L}(\R^d) = 2^c,\ \# \beta(\R^d) = c$.
\end{theorem}

\begin{theorem}
	Existe $V \subseteq \R$ no medible Lebesgue.
\end{theorem}

\begin{lemma}
	$|E + x|_{e} = |E|_{e} \quad \forall E \subseteq \R,\ x \in \R$. Además, si $E \in \mathscr{L}(\R)$, entonces $E+x \in \mathscr{L}(\R)$ y $|E| = |E + x| \quad \forall x \in \R$.
\end{lemma}
\medskip
\noindent \textbf{Axioma de Elección.} Si $(A_{\gamma})_{\gamma\in\Gamma}$ es una familia de conjuntos disjuntos, no vacíos, entonces existe un conjunto $A$ tal que $A \cap A_{\gamma}$ tiene exactamente $1$ elemento $\forall \gamma \in \Gamma$.

\begin{proof}[Proof][lema 1.47]
	Definimos una relación de equivalencia $\sim$ en $[0,1)$ decretando que $x \sim y$ si $x-y \in \Q$. Por el Axioma de Elección, existe un conjunto $V \subseteq \R$ que tiene exactamente $1$ elemento de cada clase de equivalencia de $\sim$. Observemos que:
	\begin{enumerate}
		\item[V1)] $(V + Q_1) \cap (V + Q_2) = \varnothing \quad \forall Q_1,Q_2 \in \Q$ distintos. En efecto, si $v_1 + Q_1 = v_2 + Q_2$ con $v_1,v_2 \in V \implies v_1 - v_2 = Q_2 - Q_1 \in \Q \implies v_1 \sim v_2 \implies v_1 = v_2 \implies Q_1 = Q_2$.

		\item[V2)] $[0,1) \subseteq \bigcup_{Q \in \Q} V + Q$. Notar que dado $x\in[0,1)$, existe un único $v \in V$ tal que $x \sim v$, i.e., $x-v = Q \in \Q \implies x = v + Q \in V + Q$.
	\end{enumerate}
	Si $V$ fuera medible, por $(V2)$ y el Lema,
	\[ 1 = = |[0,1)| \leq \sum_{Q \in \Q} |V+Q| = \sum_{Q\in\Q} |V| \implies |V| > 0 \]
	Por otro lado, por $(V1),\ \textbigcupd_{Q \in \Q \cap [0,1)} V+Q \subseteq [0,2)$, y luego, por el Lema y como $|V| > 0$,
	\begin{align*}
		\infty = \sum_{Q \in \Q \cap [0,1)} |V| &= \Big| \bigcupd_{Q\in\Q\cap[0,1)} V+Q \Big| \\
		& \leq |[0,2)| = |[0,1)| + |[1,2)| \\
		&= |[0,1)| + | 1 + [0,1)| = 2|[0,1)| \\
		&= 2 < \infty
	,\end{align*}
	lo cual es una contradicción. Luego $V$ no es medible.
\end{proof}
