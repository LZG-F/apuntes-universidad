\section{Clase 11 (01/09)}

\subsection*{Medida exterior de Hausdorff}

$\mathscr{H}_s =$ medida que "mide" el tamaño de objetos s-dimensionales en $\R^d$. \newline

Si $E$ es un conjunto s-dimensional en $\R^d$, entonces
\[
	\mathscr{H}_s (E) \stackrel{r_1 \ll 1}{\approx} \sum_{i\in\N} \mathscr{H}_s(E \cap B(x_i,r_i)) \approx \sum_{i\in\N} (\text{diam}(E \cap B(x_i,r_i)))^s
.\] 
Teniendo esto en cuenta, dados $d \in \N,\ s \in [0,d],\ \delta > 0$, definimos:
\begin{itemize}
	\item $C_{\delta} \coloneq \{ A \subseteq \R^d \ : \ \text{diam}A < \delta \}$;

	\item $\mathscr{H}_{s}^{(\delta)} (E) \coloneq \inf \{ \sum_{n\in\N} (\text{diam}A_n)^s \ : \ (A_n)_{n\in\N} \subseteq C_{\delta},\ E \subseteq \bigcup_{n\in\N} A_n \}$. Donde $\mathscr{H}_{s}^{(\delta)} (E)$ es la medida exterior inducida por $\tau_{s}^{(\delta)}$ y $\tau_{s}^{(\delta)}(A) \coloneq (\text{diam}A)^s$ la $\delta$-premedida de Hausdorff s-dimensional en $\R^d$ con $\tau_{s}^{(\delta)} : C_{\delta} \to [0,\infty]$.
\end{itemize}

\begin{observe}
	Si $\delta' < \delta$ entonces $\mathscr{H}_{s}^{(\delta')}(E) \geq \mathscr{H}_{s}^{(\delta)} (E)$.
\end{observe}

\noindent Luego, podemos definir
\[
	\mathscr{H}_s(E) \coloneq \sup_{\delta>0} \mathscr{H}_{s}^{(\delta)}(E) = \lim_{\delta \to 0^{+}} \mathscr{H}_{s}^{(\delta)} (E) 
,\] 
donde $\mathscr{H}_s$ es la medida exterior de Hausdorff s-dimensional en $\R^d$.

\begin{definition}[conjunto $\mu^{*}$-medible]
	Sea $X$ un espacio y $\mu^{*} : \mathcal{P}(X) \to [0,\infty]$ medida exterior. Decimos que $E \subseteq X$ es un \underline{conjunto $\mu^{*}$-medible} si
	\[
		\mu^{*}(A) = \mu^{*}(A \cap E) + \mu^{*}(A \cap E^c) \quad \forall A \subseteq X
	.\] 
\end{definition}

\begin{observe}
	$\mu^{*}(A) \leq \mu^{*}(A \cap E) + \mu^{*}(A \cap E^c)$ vale siempre (por $\sigma$-subaditividad de $\mu^{*}$. Luego, para ver que $R$ es $\mu^{*}$-medible, basta ver que $\mu^{*}(A) \geq \mu^{*}(A \cap E) + \mu^{*}(A \cap E^c)$.
\end{observe}

\begin{theorem}
	Sea $\mu^{*}$ una medida exterior sobre un espacio $X$. Entonces:
	\begin{enumerate}
		\item $\mu^{*}(E) = 0 \implies E$ es $\mu^{*}$-medible;

		\item La clase $\mathscr{M}_{\mu^{*}}$ de conjuntos $\mu^{*}$-medibles es una $\sigma$-álgebra;

		\item La restricción $\mu$ de $\mu^{*}$ a $\mathscr{M}_{\mu^{*}}$ es una medida.
	\end{enumerate}
	En particular, $(X,\mathscr{M}_{\mu^{*}},\mu)$ es un espacio de medida completo.
\end{theorem}
\begin{proof}~
	\begin{enumerate}
		\item Si $A \subseteq X,\ \mu^{*}(A \cap E) \leq \mu^{*}(E) = 0$. Además, por monotonía, $\mu^{*}(A \cap E^c) \leq \mu^{*}(A)$. Luego, $\mu^{*}(A \cap E) + \mu^{*}(A \cap E^c) = 0 + \mu^{*}(A \cap E^c) \leq \mu^{*}(A)$.

		\item \underline{$\varnothing \in \mathscr{M}_{\mu^{*}}$:} Se sigue de (1), pues $\mu^{*}(\varnothing) = 0$, por definición.

		\underline{$E \in \mathscr{M}_{\mu^{*}}$:} Directo de la definición de $\mathscr{M}_{\mu^{*}}$, puesto que es simétrica en $E$ y $E^c$.

		\underline{$(E_n)_{n\in\N} \subseteq \mathscr{M}_{\mu^{*}} \implies \bigcup_{n\in\N} E_n \in \mathscr{M}_{\mu^{*}}$:} Esto lo demostramos en tres pasos. En primer lugar, demostramos que si $E_1,E_2 \in \mathscr{M}_{\mu^{*}}$, entonces $E_1 \cap E_2, E_1 \cup E_2 \in \mathscr{M}_{\mu^{*}}$.
		\begin{proof}
			Si $A \subseteq X$, entonces
			\begin{align*}
				\mu^{*}(A) & = \mu^{*}(A \cap E_1) + \mu^{*}(A \cap E_1^c) \\
				& = \mu^{*}(A \cap E_1) + \mu^{*}(A \cap \overbrace{E_1^c \cap E_2}^{E_2 \setminus E_1}) + \mu^{*}(A \cap \underbrace{E_1^c \cap E_2^c}_{(E_1 \cup E_2)^c}) \\
				& \geq \mu^{*}(A \cap (E_1 \cup E_2)) + \mu^{*}(A \cap (E_1 \cup E_2)^c) 
			.\end{align*}
			Notar que la primera igualdad se tiene por $E_1 \in \mathscr{M}_{\mu^{*}}$ y la segunda por $E_2 \in \mathscr{M}_{\mu^{*}}$. Esto implica que $E_1 \cap E_2 \in \mathscr{M}_{\mu^{*}}$. Pero entonces $E_1 \cap E_2 = ((E_1 \cap E_2)^c)^c = (\underbrace{\underbrace{E_{1}^c}_{\in \mathscr{M}_{\mu^{*}}} \cup \underbrace{E_{2}^c}_{\in \mathscr{M}_{\mu^{*}}}}_{\in \mathscr{M}_{\mu^{*}}})^c \in \mathscr{M}_{\mu^{*}}$
		\end{proof}
		Para el segundo paso, demostramos que si $E_1,\dots,E_n \in \mathscr{M}_{\mu^{*}}$ disjuntos, entonces 
		\[ \mu^{*}\left(A \cap \left(\displaystyle\bigcupd_{i=1}^{n} E_i \right)\right) = \displaystyle\sum_{i=1}^{n} \mu^{*}(A \cap E_i). \]
		\begin{proof}
			La idea es probarlo por inducción. Basta ver el caso $n=2$ (los otros casos salen iterando éste)
			\begin{align*}
				& \mu^{*}(A \cap (E_1 \cupd E_2)) \\
				= \ & \mu^{*}(\underbrace{A \cap (E_1 \cupd E_2) \cap E_1}_{A \cap E_1}) + \mu^{*}(\underbrace{A \cap (E_1 \cupd E_2) \cap E_1^c}_{A \cap E_2}) 
			.\end{align*}
			pues $E_2 \subseteq E_1^c$ por ser disjuntos.
		\end{proof}
		Por último, vemos que si $(E_n)_{n\in\N} \subseteq \mathscr{M}_{\mu^{*}}$, entonces $\bigcup_{n\in\N} E_n \in \mathscr{M}_{\mu^{*}}$.
		\begin{proof}
			Podemos suponer que los $E_n$ son disjuntos. Si no, los cambiamos por
			\begin{align*}
				& E_{1}' \coloneq E_{1} \in \mathscr{M}_{\mu^{*}} \\
				& E_{2}' \coloneq E_{2} \setminus E_{1} = E_2 \cap E_1^c \in \mathscr{M}_{\mu^{*}} \\
				& \ \vdots \\
				& E_{n+1}' \coloneq E_{n+1} \setminus \bigcup_{i=1}^{n} E_i \in \mathscr{M}_{\mu^{*}}
			,\end{align*}
			y 
			\[ \displaystyle\bigcup_{n\in\N} E_n = \displaystyle\bigcupd_{n=1}^{\infty} E_{n}'. \]
			Sea 
			\[ F_n \coloneq \displaystyle\bigcupd_{i=1}^{n} E_i \longrightarrow E \coloneq \displaystyle\bigcupd_{n\in\N} E_n. \]
			Notar que si $F_n \subseteq E$, entonces $E^c \subseteq F_{n}^{c}$. Luego, dado $A \subseteq X$, como $F_n \in \mathscr{M}_{\mu^{*}}$, se tiene
			\begin{align*}
				\mu^{*}(A) &= \underbrace{\mu^{*}(A \cap F_n)}_{= \sum_{i=1}^{n} \mu^{*}(A \cap E_i)} + \mu^{*}(\underbrace{A \cap F_{n}^{c}}_{\subseteq A \cap E^c}) \\
				& \geq \sum_{i=1}^{n} \mu^{*}(A \cap E_i) + \mu^{*}(A \cap E^c)
			.\end{align*}
			Tomando $n \to \infty$,
			\begin{align*}
				\mu^{*}(A) & \geq \sum_{i=1}^{n} \mu^{*}(A \cap E_i) + \mu^{*}(A \cap E^c) \\
				& \geq \mu^{*}(A \cap E) + \mu^{*}(A \cap E^c) \tag{$\mu^{*}\ \sigma$-subad.} \\
				& \quad A \cap E = \bigcup_{i=1}^{\infty} A \cap E_i
			.\end{align*}
		\end{proof}
	\end{enumerate}
\end{proof}
