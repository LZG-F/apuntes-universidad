\section{Clase 10 (29/08)}

\subsection*{Demostración del teorema de extensión de Carathéodory}

\textbf{Paso 1:} Medidas Exteriores

\begin{prop}
	Si $I \subseteq \R$ es un intervalo,
	\[ |E|_{e} = \inf \{ \sum_{n=1}^{\infty} |I_n| \ : \  (I_n)_{n\in\N} \text{ intervalos, } E \subseteq \bigcup_{n=1}^{\infty} I_n \} \]
\end{prop}
\begin{proof}[Proof ]
	$\geq)$ Tomando $I_1 = I,\ I_{n+1} = \varnothing \quad \forall n \in \N$ \newline

	$\leq)$ Por la $\sigma$-subaditividad de $\lambda$ en $\mathcal{I}$: si $I\subseteq \bigcup_{n=1}^{\infty}$ entonces $\lambda(I) \leq \sum_{i=1}^{\infty} \lambda(I_i)$.
\end{proof}

\begin{definition}[Medida exterior inducida por una premedida]
	Sea $X$ un espacio, $\mathscr{C}$ una clase de subconjuntos de $X$ y $\mathcal{T} : \mathscr{C} \to [0,\infty]$ una premedida. Definimos la \underline{medida exterior inducida por $\mathcal{T}$} como la aplicación $\mu_{\mathcal{T}}^{*} : \mathscr{P}(X) \to [0,\infty]$ dada por
	\[ \mu_{\mathcal{T}}^{*}(A) \coloneq \inf \{ \sum_{n=1}^{\infty} \mathcal{T}(C_i) \ : \ (C_i)_{i\in\N} \subseteq \mathscr{C} \text{ y } A \subseteq \bigcup_{i=1}^{\infty} C_i \} \]
	con la convención de que $\inf \varnothing \coloneq \infty$.
\end{definition}

\begin{eg}
	$\mu_{\lambda}^{*} =$ \textit{medida exterior de Lebesgue} y la notamos $|E|_e \coloneq \mu_{\lambda}^{*}(E)$.
\end{eg}

Idealmente, nos gustaría que $\mu_{\mathcal{T}}^{*}$ cumpla
\[ \begin{cases}
	(C1) \ \mu_{\mathcal{T}^{*}}(C) = \mathcal{T}(C) \quad \forall C \in \mathscr{C} \\
	(C2) \ \mu_{\mathcal{T}}^{*} \text{ es } \sigma\text{-subaditiva en } \mathscr{P}(X)
\end{cases} \]
no tienen por qué cumplirse ninguna de la 2:
\begin{enumerate}
	\item[(C1)] $X = \{a,b\},\ \mathscr{C} = \{\varnothing, \{a\}, X\},\ \mathcal{T}(A) = \begin{cases}
		0 \quad A = \varnothing \\
		2 \quad A = \{a\} \\
		1 \quad A = X
	\end{cases} \mathcal{T}(\{a\}) = 2,\ \mu_{\mathcal{T}}^{*}(\{a\}) = 1 \neq \mathcal{T}(\{a\})$.

	\item[(C2)] Medida exterior de Lebesgue no es $\sigma$-aditiva (lo vemos mas adelante!)
\end{enumerate}

\begin{prop}
	Si $\mathcal{T}$ es una premedida sobre una semiálgebra $\mathscr{S}$ que satisface
	\begin{enumerate}
		\item[(E2)] $\mathcal{T}$ es $\sigma$-subaditiva en $\mathscr{S}$,
	\end{enumerate}
	entonces $\mu_{\mathcal{T}}^{*}(A) = \mathcal{T}(A)\quad \forall A \in \mathscr{S}$ (i.e. $\mu_{\mathcal{T}}^{*}$ cumple (C1)).
\end{prop}
\begin{proof}[Proof ]
	$\underline{\mu_{\mathcal{T}}^{*}(A) \leq \mathcal{T}(A).}$ Tomando $C_1 = A \in \mathscr{S},\ C_{n+1} = \varnothing \in \mathscr{S}$. Luego $(C_n)_{n\in\N}$ es cubrimiento de $A$ por elementos de $\mathscr{S}$ y luego
	\[ \mu_{\mathcal{T}}^{*}(A) \leq \sum_{n\in\N} \mathcal{T}(C_n) = \mathcal{T}(A) \]
	$\underline{\mathcal{T}(A) \leq \mu_{\mathcal{T}}^{*}(A).}$ Si $(C_n)_{n\in\N}\subseteq \mathscr{S}$ es un cubrimiento de $A \in \mathscr{S}$ entonces por (E2), tenemos que $\mathcal{T}(A) \leq \sum_{n\in\N} \mathcal{T}(C_n)$. Tomando $\inf$ sobre tales cubrimientos, resulta $\mathcal{T}(A) \leq \mu_{\mathcal{T}}^{*}(A)$.
\end{proof}

\begin{theorem}
	Sean $X$ un espacio, $\mathscr{C}$ una clase de subconjuntos de $X$ y $\mathcal{T}:\mathscr{C} \to [0,\infty]$ una premedida. Entonces,
	\begin{enumerate}
		\item $\mu_{\mathcal{T}}^{*}(\varnothing)$;

		\item $\mu_{\mathcal{T}}^{*}$ es monótona $(A \subseteq B \implies \mu_{\mathcal{T}}^{*}(A)\leq\mu_{\mathcal{T}}^{*}(B))$;

		\item $\mu_{\mathcal{T}}^{*}$ es $\sigma$-subaditiva $(A\subseteq\bigcup_{n\in\N}A_n \implies \mu_{\mathcal{T}}^{*}(A) \leq \sum_{n=1}^{\infty} \mu_{\mathcal{T}}^{*}(A_n)$.
	\end{enumerate}
\end{theorem}
\begin{proof}[Proof ]
	\begin{enumerate}
		\item $\mu_{\mathcal{T}}^{*}(\varnothing)\geq 0$ es por definición. Para ver que $\mu_{\mathcal{T}}^{*}(\varnothing)\leq 0$, tomamos el cubrimiento $C_n = \varnothing$ y repetimos el argumento de la Proposición anterior.
		
		\item Si $\mu_{\mathcal{T}}^{*}(B)=\infty$, la desigualdad es inmediata. Si $\mu_{\mathcal{T}}^{*}(B)<\infty$, entonces existen cubrimientos de $B$ por elementos de $\mathscr{S}$. Sea $(C_n)_{n\in\N}\subseteq\mathscr{S}$ un cubrimiento de $B$. Entonces, $(C_n)_{n\in\N}$ es también cubrimiento de $A$ y, luego, $\mu_{\mathcal{T}}^{*}(A)\leq \sum_{n\in\N} \mathcal{T}(C_n)$. Como esto es cierto para todo cubrimiento $(C_n)_{n\in\N}$ de $B$, tomando ínfimo en la desigualdad anterior sobre tales cubrimientos resulta $\mu_{\mathcal{T}}^{*}(A) \leq \mu_{\mathcal{T}}^{*}(B)$.

		\item Dado $\varepsilon>0$, sea $(C_{i}^{(n)})_{i\in\N}$ un cubrimiento de $A_n$ tal que $\sum_{i=1}^{\infty} \mathcal{T}(C_{i}^{(n)}) \leq \mu_{\mathcal{T}}^{*}(A_n) + \frac{\varepsilon}{2^n}$. Luego, notando que $(C_{i}^{(n)}\ : \ i\in\N,\ n\in\N)$ es un cubrimiento de $A$, obtenemos que
		\begin{align*}
			\mu_{\mathcal{T}}^{*}(A) \leq \sum_{n=1}^{\infty} \sum_{i=1}^{\infty} \mathcal{T}(C_{i}^{(n)}) & \leq \sum_{n=1}^{\infty} \left(\mu_{\mathcal{T}}^{*}(A_n) + \frac{\varepsilon}{2^n} \right) \\
			& \leq \sum_{n=1}^{\infty} \mu_{\mathcal{T}}^{*}(A_n) + \varepsilon \underbrace{\sum_{n=1}^{\infty} \frac{1}{2^m}}_{1}
		\end{align*}
		Luego, $\mu_{\mathcal{T}}^{*}(A) \leq \sum_{n=1}^{\infty} \mu_{\mathcal{T}}^{*}(A_n) + \varepsilon \quad \forall \varepsilon>0$. Tomando $\varepsilon \to 0^+$, obtenemos la $\sigma$-subaditividad de $\mu_{\mathcal{T}}^{*}$.
	\end{enumerate}
\end{proof}

\begin{definition}[medida exterior]
	Sea $X$ un espacio. Decimos que $\mu^{*} : \mathscr{P}(X)\to[0,\infty]$ es una \underline{medida exterior} si:
	\begin{enumerate}
		\item $\mu^{*}(\varnothing) = 0$;

		\item $A \subseteq B \implies \mu^{*}(A)\leq \mu^{*}(B)$;

		\item $A \subseteq \bigcup_{n\in\N} A_n \implies \mu^{*}(A) \leq \sum_{n=1}^{\infty} \mu^{*}(A_n)$.
	\end{enumerate}
\end{definition}

\begin{eg}~
	\begin{enumerate}
		\item Medidas exteriores generadas por una premedida;

		\item Si $(\mu_{\gamma}^{*})_{\gamma\in\Gamma}$ son medidas exteriores sobre $X$, entonces
		\[ \mu^{*}(A) \coloneq \sup_{\gamma\in\Gamma} \mu_{\gamma}^{*}(A) \]
		es una medida exterior (Ej. Guía 3).

		\item \textbf{Medida exterior $s$-dimensional de Hausdorff en $\R^d$.}
		\begin{itemize}
			\item Si $I$ es un intervalo en $\R^d$, entonces $|rI| = r^d|I|$;

			\item Si $E \subseteq \R^d$ es medible Lebesgue, entonces $|rE| = r^d|E|$;

			\item En particular, si $E = B(x,r)$, entonces
			\[ |E| = |B(0,r)| = |rB(0,1)| = r^d|B(0,1)| = C_d (diam E)^d,\quad C_d \coloneq \frac{|B(0,1)|}{2^d} \]

			\item Si $E \subseteq \R^d$ es "s-dimensional" y $\mathscr{H}_s$ es la mdeida que queremos, entonces debería valer que...
		\end{itemize}
	\end{enumerate}
\end{eg}
