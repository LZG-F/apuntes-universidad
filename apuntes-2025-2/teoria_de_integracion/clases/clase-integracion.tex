\documentclass[a4paper]{report}
\usepackage[utf8]{inputenc}
\usepackage[T1]{fontenc}
\usepackage{textcomp}

\usepackage{url}

% \usepackage{hyperref}
% \hypersetup{
%     colorlinks,
%     linkcolor={black},
%     citecolor={black},
%     urlcolor={blue!80!black}
% }

\usepackage{graphicx}
\usepackage{float}
\usepackage[usenames,dvipsnames]{xcolor}

% \usepackage{cmbright}

\usepackage{amsmath, amsfonts, mathtools, amsthm, amssymb}
\usepackage{mathrsfs}
\usepackage{cancel}

\newcommand\N{\ensuremath{\mathbb{N}}}
\newcommand\R{\ensuremath{\mathbb{R}}}
\newcommand\Z{\ensuremath{\mathbb{Z}}}
\renewcommand\O{\ensuremath{\emptyset}}
\newcommand\Q{\ensuremath{\mathbb{Q}}}
\newcommand\C{\ensuremath{\mathbb{C}}}
\let\implies\Rightarrow
\let\impliedby\Leftarrow
\let\iff\Leftrightarrow
\let\epsilon\varepsilon

% demostraciones bidireccionales

\newcommand{\Onlyifstep}{%
	\begingroup
	\fboxsep=1pt
	\raisebox{1.2ex}{\fbox{\raisebox{-1.2ex}{$\Rightarrow$\hspace{-0.05em}}}}%
	\endgroup
	\hspace{0.5em}%
	}
\newcommand{\Ifstep}{%
	\begingroup
	\fboxsep=1pt
	\raisebox{1.2ex}{\fbox{\raisebox{-1.2ex}{\hspace{-0.05ex}$\Leftarrow$}}}%
	\endgroup
	\hspace{0.5em}%
	}

% horizontal rule
\newcommand\hr{
    \noindent\rule[0.5ex]{\linewidth}{0.5pt}
}

\usepackage{tikz}
\usepackage{tikz-cd}

% theorems
\usepackage{thmtools}
\usepackage[framemethod=TikZ]{mdframed}
\mdfsetup{skipabove=1em,skipbelow=0em, innertopmargin=5pt, innerbottommargin=6pt}

\theoremstyle{definition}

\makeatletter

\declaretheoremstyle[headfont=\bfseries\sffamily, bodyfont=\normalfont, mdframed={ nobreak } ]{thmgreenbox}
\declaretheoremstyle[headfont=\bfseries\sffamily, bodyfont=\normalfont, mdframed={ nobreak } ]{thmredbox}
\declaretheoremstyle[headfont=\bfseries\sffamily, bodyfont=\normalfont]{thmbluebox}
\declaretheoremstyle[headfont=\bfseries\sffamily, bodyfont=\normalfont]{thmblueline}
\declaretheoremstyle[headfont=\bfseries\sffamily, bodyfont=\normalfont, numbered=no, mdframed={ rightline=false, topline=false, bottomline=false, }, qed=\qedsymbol ]{thmproofbox}
\declaretheoremstyle[headfont=\bfseries\sffamily, bodyfont=\normalfont, numbered=no, mdframed={ nobreak, rightline=false, topline=false, bottomline=false } ]{thmexplanationbox}


\declaretheorem[numberwithin=chapter, style=thmgreenbox, name=Definition]{definition}
\declaretheorem[sibling=definition, style=thmredbox, name=Corollary]{corollary}
\declaretheorem[sibling=definition, style=thmredbox, name=Proposition]{prop}
\declaretheorem[sibling=definition, style=thmredbox, name=Theorem]{theorem}
\declaretheorem[sibling=definition, style=thmredbox, name=Lemma]{lemma}



\declaretheorem[numbered=no, style=thmexplanationbox, name=Proof]{explanation}
\declaretheorem[numbered=no, style=thmproofbox, name=Proof]{replacementproof}
\declaretheorem[style=thmbluebox,  numbered=no, name=Exercise]{ex}
\declaretheorem[style=thmbluebox,  numbered=no, name=Example]{eg}
\declaretheorem[style=thmblueline, numbered=no, name=Remark]{remark}
\declaretheorem[style=thmblueline, numbered=no, name=Note]{note}

\renewenvironment{proof}[1][\proofname]{\begin{replacementproof}}{\end{replacementproof}}

\AtEndEnvironment{eg}{\null\hfill$\diamond$}%

\newtheorem*{uovt}{UOVT}
\newtheorem*{notation}{Notation}
\newtheorem*{previouslyseen}{As previously seen}
\newtheorem*{problem}{Problem}
\newtheorem*{observe}{Observe}
\newtheorem*{property}{Property}
\newtheorem*{intuition}{Intuition}


\usepackage{etoolbox}
\AtEndEnvironment{vb}{\null\hfill$\diamond$}%
\AtEndEnvironment{intermezzo}{\null\hfill$\diamond$}%




% http://tex.stackexchange.com/questions/22119/how-can-i-change-the-spacing-before-theorems-with-amsthm
% \def\thm@space@setup{%
%   \thm@preskip=\parskip \thm@postskip=0pt
% }

\usepackage{xifthen}

\def\testdateparts#1{\dateparts#1\relax}
\def\dateparts#1 #2 #3 #4 #5\relax{
    \marginpar{\small\textsf{\mbox{#1 #2 #3 #5}}}
}

\def\@lesson{}%
\newcommand{\lesson}[3]{
    \ifthenelse{\isempty{#3}}{%
        \def\@lesson{Lecture #1}%
    }{%
        \def\@lesson{Lecture #1: #3}%
    }%
    \subsection*{\@lesson}
    \testdateparts{#2}
}

% fancy headers
\usepackage{fancyhdr}
\pagestyle{fancy}

% \fancyhead[LE,RO]{Gilles Castel}
\fancyhead[RO,LE]{\@lesson}
\fancyhead[RE,LO]{}
\fancyfoot[LE,RO]{\thepage}
\fancyfoot[C]{\leftmark}
\renewcommand{\headrulewidth}{0pt}

\makeatother

% figure support (https://castel.dev/post/lecture-notes-2)
\usepackage{import}
\usepackage{xifthen}
\pdfminorversion=7
\usepackage{pdfpages}
\usepackage{transparent}
\newcommand{\incfig}[1]{%
    \def\svgwidth{\columnwidth}
    \import{./figures/}{#1.pdf_tex}
}

% %http://tex.stackexchange.com/questions/76273/multiple-pdfs-with-page-group-included-in-a-single-page-warning
\pdfsuppresswarningpagegroup=1

\author{Gilles Castel}

\DeclareMathOperator{\supp}{supp}
\DeclareMathOperator{\spann}{span}
\DeclareMathOperator{\Id}{Id}
\DeclareMathOperator{\Ker}{Ker}
\DeclareMathOperator{\im}{Im}
\DeclareMathOperator{\GL}{GL}
\DeclareMathOperator{\SL}{SL}
\DeclareMathOperator{\Mat}{Mat}

%MIS PAQUETES

\newcommand{\bigcupd}{\mathop{\ensurestackMath{\stackinset{l}{}{c}{+.25ex}{\hspace{3pt}{\text{\tiny{D}}}}{\displaystyle{\bigcup}}}}}

\usepackage{stackengine,scalerel}

%\newcommand{\bigcupcolon}{\mathop{\ThisStyle{%
%  \ensurestackMath{\stackinset{c}{-0.4pt}{c}{+.25\LMex}{\tiny{D}}{\SavedStyle\bigcup}}}}}

%MIS PAQUETES

\title{Teoría de Integración}
\author{}
\date{Basado en las clases impartidas por Santiago Saglietti en el segundo semeste del 2025}
\begin{document}
    \maketitle
    \tableofcontents
    % start lessons

    \chapter{Integral de Riemann}
    \setcounter{section}{0}
	\section{Clase 1 (04/08)}

	\begin{definition}[partición + intervalos]
		Una partición de un intervalo $[a,b]\subseteq\R$ es un subconjunto finito $\Pi\subseteq[a,b]$ tal que $a,b\in\Pi$. Denotaremos a las particiones como $\Pi=\{x_0,\dots,x_n\}$, donde $a=x_0<x_1<\cdots<x_n=b$. Los intervalos $I_i=[x_{i-1},x_i]$, $i=1,\dots,n$ serán llamados intervalos de la partición.
	\end{definition}

	\begin{remark}
		A veces, identificaremos la partición $\Pi$ con $(I_i)_{i=1,\dots,n}$. En tal caso, abusando de la notación, escribiremos $I_i\in\Pi$ cuando queramos hablar de los intervalos de $\Pi$.
	\end{remark}

	\begin{definition}[norma de particiones]
		La norma de una partición $\Pi$ como $\|\Pi\|\coloneqq \max_{i=1,\dots,n}(x_i-x_{i-1})=\max_{I_i\in\Pi}|I_i|$.
	\end{definition}

	\begin{definition}[partición marcada]
		Una partición marcada de $[a,b]$ es un par $\Pi^*\coloneqq(\Pi,\varepsilon)$ donde:
		\begin{itemize}
			\item $\Pi = \{x_0,\dots,x_n\}$ es una partición de $[a,b]$;
			\item $\varepsilon = \{x_1^*,\dots,x_n^*\}$ es una colección de puntos tal que $x_i^*\in I_i$ para cada $i=1,\dots,n$.
		\end{itemize}
	\end{definition}

	\begin{remark}
		Dada una partición marcada $\Pi^*=(\Pi,\varepsilon)$, definimos $\|\Pi^*\|\coloneqq \|\Pi\|$.
	\end{remark}

	\begin{definition}[Suma de Riemann]
		Sean $f:[a,b]\to\R$ acotada y $\Pi^*=(\Pi,\varepsilon)$ una partición marcada. Definimos la suma de Riemann de $f$ asociada a $\Pi^*$ como:
		\[
		S_R(f;\Pi^*)\coloneqq \sum_{n=1}^{n} f(x_i^*)(x_i-x_{i-1})= \sum_{I_i\in\Pi}^{} f(x_i^*)|I_i|.
		\]
	\end{definition}

	\section{Clase 2 (06/08)}

	\begin{definition}[Riemann integrable]
		Dada $f:[a,b]\to\R$ acotada, decimos que es Riemann integrable si existe el límite $\lim_{\|\Pi^*\|\to 0} S_R(f;\Pi^*)$. \\
		Equivalentemente, $\exists L\in\R$, tal que dado cualquier $\varepsilon>0$, existe $\delta=\delta(\varepsilon)>0$ tal que $\|\Pi^*\|<\delta\Rightarrow|S_R(f;\Pi^*)-L|<\varepsilon$.
	\end{definition}

	\begin{remark}
		Cuando el límite existe, lo llamamos la integral de Riemann de $f$ en $[a,b]$ y lo notamos $\int_{a}^{b} f(x) dx$.
	\end{remark}

	\begin{definition}[Sumas superior e inferior de Darboux]
		Dadas $f:[a,b]\to\R$ acotada y $\Pi=(I_i)_{i=1,\dots,n}$ una partición de $[a,b]$, definimos 
	
		\begin{align*}
			m_{I_i}\coloneqq\inf_{x\in I_i} f(x) &,\quad M_{I_i}\coloneqq\sup_{x\in I_i} f(x) \quad \text{y} \\
			\underline{S}(f;\Pi)\coloneqq\sum_{I_i\in\Pi}^{} m_{I_i}|I_i|&,\quad \overline{S}(f;\Pi)\coloneqq\sum_{I_i\in\Pi}^{} M_{I_i}|I_i|
		.\end{align*}

		Llamamos a $\underline{S}(f;\Pi)$ y $\overline{S}(f;\Pi)$ las sumas inferior y superior de Darboux de $f$ con respecto a $\Pi$, respectivamente.
	\end{definition}

	\begin{note}
		Como $m_{I_i}\leq f(x)\leq M_{I_i}$, $\forall x\in I_i$ para toda partición marcada $\Pi^*=(\Pi;\varepsilon)$, tenemos $\underline{S}(f;\Pi)\leq S_R(f;\Pi^*)\leq \overline{S}(f;\Pi)$.
	\end{note}

	\begin{definition}[refinamiento]
		Diremos que una partición $\Pi'$ de $[a,b]$ es un refinamiento de otra partición de $[a,b]$, $\Pi$, si $\Pi\subseteq\Pi'$. \\
		Equivalentemente, si para todo $J_i\in\Pi'$ existe $I_i\in\Pi$ tal que $J_i\subseteq I_i$.
	\end{definition}

	\begin{prop}
		Sea $f:[a,b]\to\R$ acotada. Entonces,
		\begin{itemize}
			\item Si $\Pi\subseteq\Pi'$ son particiones de $[a,b]$,

			\[
			\underline{S}(f;\Pi)\leq\underline{S}(f;\Pi'),\quad\overline{S}(f;\Pi)\geq\overline{S}(f;\Pi').
			\]

			\item Si $\Pi_1,\Pi_2$ son particiones de $[a,b]$ cualesquiera,

			\[
			\underline{S}(f;\Pi_1)\leq\overline{S}(f;\Pi_2)
			\]
		\end{itemize}
	\end{prop}

	\begin{definition}
		Sea $f:[a,b]\to\R$ acotada. Definimos:
		\begin{itemize}
			\item La integral superior (de Darboux) de $f$ como $\overline{\int_{a}^{b}} f(x) dx \coloneq \displaystyle{\inf_{\Pi}} \overline{S}(f;\Pi)$.
			
			\item La integral inferior (de Darboux) de $f$ como $\underline{\int_{a}^{b}} f(x) dx \coloneq \displaystyle{\sup_{\Pi}} \underline{S}(f;\Pi)$.
		\end{itemize}
	\end{definition}

	\begin{theorem}
		Sea $f:[a,b]\to\R$ acotada. Entonces,
		\[
		\underline{\int_{a}^{b}} f(x) dx = \displaystyle{\lim_{\|\Pi\|\to 0}} \underline{S}(f;\Pi)\quad \text{y} \quad \overline{\int_{a}^{b}} f(x) dx = \displaystyle{\lim_{\|\Pi\|\to 0}}\overline{S}(f;\Pi).
		\]
	\end{theorem}

	\begin{remark}
		Equivalentemente, para cualquier sucesión $(\Pi_n)_{n\in\N}$ de partición de $[a,b]$ tal que $\|\Pi_n\|\xrightarrow{n\to\infty} 0$, se tiene que 

		\[
		\underline{\int_{a}^{b}} f(x) dx = \lim_{n\to\infty} \underline{S}(f;\Pi_n)\quad\text{y}\quad \overline{\int_{a}^{b}} f(x) dx = \lim_{n\to \infty} \overline{S}(f;\Pi_n).
		\]
	\end{remark}

	\begin{theorem}
		Dada $f:[a,b]\to\R$ acotada, son equivalentes:

		\begin{enumerate}
			\item $\underline{\int_{a}^{b}} f(x) dx = \overline{\int_{a}^{b}} f(x) dx$ (i.e., $f$ es Darboux integrable).

			\item $f$ es Riemann integrable.

			\item $\lim_{\|\Pi\|\to 0} \overline{S}(f;\Pi)-\underline{S}(f;\Pi)=0$.

			\item $\forall (\Pi_n)_{n\in\N}$ sucesión de particiones de $[a,b]$ tal que $\|\Pi_n\|\to 0$, 

			\[
			\lim_{n \to \infty} \overline{S}(f;\Pi_n)-\underline{S}(f;\Pi_n)=0.
			\]

			\item $\exists (\Pi_n)_{n\in\N}$ sucesión de particiones de $[a,b]$ tal que 

			\[
			\lim_{n \to \infty}\overline{S}(f;\Pi_n)-\underline{S}(f;\Pi_n)=0.
			\]
		\end{enumerate}
	\end{theorem}

	\section{Clase 3 (07/08)}

	\begin{note}
		Las integrales en el sentido de Darboux y el de Riemann coinciden.
	\end{note}

	\begin{prop}
		Si $f:[a,b]\to\R$ es monótona, entonces es Riemann integrable.
	\end{prop}

	\begin{remark}
		Una función monótona tiene discontinuidades numerables.
	\end{remark}

	\begin{prop}
		Si $f:[a,b]\to\R$ es continua, entonces es Riemann integrable.
	\end{prop}

	En particular, existen funciones Riemann integrables con numerables discontinuiodades. De hecho, hay ejemplos con $c$ (cardinal del continuo) discontinuidades. No obstante, si $f$ es integral de Riemann, su conjunto de discontinuidades tiene que ser "pequeño".

	\begin{theorem}
		Sea $f:[a,b]\to\R$ acotada. Entonces, $f$ es integral de Riemann si y sólo si su conjunto de discontinuidades tiene medida nula.
	\end{theorem}

	\begin{definition}[intervalo]
		Decimos que un conjunto $I\subseteq\overline{\R}\coloneq\R\cup \{-\infty,\infty\}$ es un intervalo si satisface

		\[
		x,y\in I \Rightarrow z\in I \text{ para todo } \min x,y\leq z\leq\max x,y.
		\]
	\end{definition}

	\begin{eg}
		(y propiedades)
		\begin{itemize}
			\item Dados $a\leq b$ ($a,b\in\R$), los conjuntos $(a,b),(a,b],[a,b],[a,b)$ son intervalos;

			\item El conjunto vacío es un intervalo ($\varnothing = (a,a)$);

			\item Los puntos son intervalos. $I = [\lambda,\lambda]$;

			\item La intersección son intervalos de intervalos.
		\end{itemize}
	\end{eg}

	\begin{definition}[intervalo generalizado]
		Decimos que un conjunto $I\subseteq\R^d$ es un intervalo si puede escribirse como

		\[
		I=\prod_{k=1}^{d} I_k
		\]

		donde cada $I_r$ es un intervalo en $\R$. La medida de un intervalo $I\subseteq\R^d$ se define como
		
		\[
		|I|\coloneq\prod_{k=1}^{d} |I_k|.
		\]
	\end{definition}

	\begin{note}
		Los intervalos en $\R^d$ heredan las mismas pripiedades en $\R$:

		\begin{itemize}
			\item Intersección de intervalos en $\R^d$ es intervalo.

			\item Si $I\subseteq J\subseteq\R^d$ son intervalos, entonces $|I|\leq|J|$.
		\end{itemize}
	\end{note}

	\begin{definition}[medida nula]
		Un conjunto $E\subseteq\R^d$ se dice de medida nula si, dado $\varepsilon>0$, existe una sucesión $(I_n)_{n\in\N}$ de intervalos de $\R^d$ tal que

		\[
		E\subseteq\bigcup_{n\in\N} I_n\quad\text{ y }\quad \sum_{n\in\N}^{}|I_n|<\varepsilon.
		\]
	\end{definition}

	\begin{eg}
		(y propiedades)
		\begin{enumerate}
			\item Todo conjunto unitario $\{x\}, (x\in\R^d)$ tiene medida nula;

			\item Toda unión numerable de conjuntos de medida nula tiene medida nula;

			\item Cualquier conjunto numerable tiene medida nula;
			
			\item Cualquier subconjunto de un conjunto de medida nula tiene medida nula;

			\item Existen conjuntos no numerables de medida nula:

			\begin{itemize}
				\item En $\R^d$ con $d\geq 2$, los ejes $\{x:x_1=0\}, i=1,\dots,d$ tiene medida nula.

				\item En $\R$, el conjunto de cantor tiene medida nula.
			\end{itemize}

			\item $E\subseteq\R^d$ es de medida nula, entonces $\alpha\dot E$ tiene medida nula $\forall\alpha\in\R$.
			
			\item $E\subseteq\R^d$ es de medida nula, entonces $E + v$ tiene medida nula $\forall v\in\R^d$.

			\item Si $E$ contiene un intervalo no unitario, entonces no tiene medida nula. Notar que:

			\begin{itemize}
				\item La vuelta no es válida: $\R\textbackslash\Q$ no contiene untervalos no unitarios pero no puede tener medida nula.

				\item De esto se deduce que si $E\subseteq\R^d$ tiene medida nula. Entonces $E^c$ es denso (no vale la vuelta: $E^c=\Q$).
			\end{itemize}

			\item $E\subseteq\R^d$ tiene medida nula si y sólo si

			\[
			|E|_e\coloneq\inf \{ \sum_{n\in\N} |I_n| : E \subseteq \bigcup_{n\in\N} I_n \} = 0, \quad I_n \text{ intervalo } \forall n \in \N.
			\]
		\end{enumerate}
	\end{eg}

	\section{Clase 4 (08/08)}

	\begin{theorem}
		Sea $f:[a,b]\to\R$ acotada. Entonces
		\begin{align*}
			f \text{ Riemann integrable}\Longleftrightarrow & D_f=\{x\in[a,b]:f\text{ discontinua en } x\} \\ & \text{ tiene medida nula.}
		\end{align*}
	\end{theorem}

	\subsection{Limitaciones de la integral de Riemann}

	\begin{enumerate}
		\item Sólo está definida para $f$ acotada y sobre intervalos $[a,b]$ acotados. La teoría de integrales impropias resuelve esto.

		\item Propiedades del espacio $\mathcal{R}([a,b])=\{f:[a,b]\to\R:f\text{ Riemann integrable}\}$: Nos gustaría poder definir una noción de convergencia en $\mathcal{R}([a,b])$ tal que
		\[
		f_n\to f \text { en } \mathcal{R}([a,b]) \Rightarrow \int_{a}^{b} f_n\to \int_{a}^{b} f \quad \left( \lim \int_{a}^{b} f_n = \int_{a}^{b} \lim f_n \right).
		\]
	\end{enumerate}

	\begin{remark}
		La convergencia puntal NO cumple esto (punto 2).
	\end{remark}

	\begin{eg}[1]~
		\begin{itemize}
			\item $f_n \coloneq n \chi_{(0,\frac{1}{n}]}$ es Riemann integrable en $[0,1],\ \forall n \in \N$;

			\item $f_n \to f \cong 0$ puntualmente en $[0,1]$;

			\item $\int_{0}^{1} f_n = 1 \not\to 0 = \int_{0}^{1} f$.
		\end{itemize}
	\end{eg}

	\begin{eg}[2]~
		\begin{itemize}
			\item Sea $(Q_n)_{n \in \N}$ una enumeración de $\Q \cap [0,1]$;

			\item $f_n \coloneq \chi_{\{ Q_1,\dots,Q_n \}}$ es Riemann integrable en $[0,1],\ \forall n \in \N$;

			\item $f_n \to f \coloneq \chi_{\Q \cap [0,1]}$ puntualmente en $[0,1]$;

			\item $f$ no es Riemann integrable. $\underline{\int_{0}^{1}} f = 0 \neq 1 = \overline{\int_{0}^{1}} f$.
		\end{itemize}
	\end{eg}

	\begin{remark}
		La convergencia uniforme SÍ cumple esto, pero es demasiado fuerte.
	\end{remark}

	\begin{ex}[Guía 1]
		Sean $(f_n)_{n \in \N} \subset \mathcal{R}([a,b])$ tales que $f_n \to f$ uniformemente en $[a,b]$. Entonces, $f \in \mathcal{R}([a,b])$ y $\lim_{n \to \infty} \int_{a}^{b} f_n = \int_{a}^{b} f$.
	\end{ex}

	\begin{eg}[3]~
		\begin{itemize}
			\item $f_n (x) \coloneq x^n$ en $[0,1],\ f_n \in \mathcal{R}([a,b]),\ \forall n \in \N,\ f_n \to \chi = f$ puntualmente;

			\item $f \in \mathcal{R}([a,b])$ y $\int_{0}^{1} f_n (x) dx = \frac{1}{n+1} \to 0 = \int_{0}^{1}$;

			\item $f_n$ no converge uniformemente a $f$.
		\end{itemize}
	\end{eg}

	Resulta que la noción de convergencia "óptima" (la más "débil" que cumple lo que queremos) es la de convergencia en $L'$:

	\[
	f_n \xrightarrow{L'} f \text{ si } \lim_{n \to \infty} \int_{a}^{b} | f_n - f | = 0.
	\]

	Esta noción de convergencia viene dada por una "norma":

	\begin{itemize}
		\item $\| f \|_{L'} \coloneq \int_{a}^{b} |f|$ (recordar que $f \in \mathcal{R}([a,b]) \implies |f| \in \mathcal{R}([a,b])$);

		\item $d_{L'} (f,g) \coloneq \| f - g \|_{L'} = \int_{a}^{b} |f-g|$.
	\end{itemize}

	\begin{remark}
		$\| \cdot \|_{L'}$ no es una norma porque $\| f \|_{L'} = 0 \nRightarrow f = 0$. Decimos que es una \textit{pseudo-norma} y $d$ una \textit{pseudo-métrica}.
	\end{remark}

	Para arreglar esto, dadas $f,g : [a,b] \to \R$, decimos que son \textit{equivalentes} y lo notamos $f \sim g$ si $\{ x \in [a,b] \ : \ f(x) \neq g(x) \}$ tiene medida nula. Resulta que $\sim$ es una relación de equivalencia y, además,

	\[
	f,g \in \mathcal{R} ([a,b]),\ f \sim g \implies \int_{a}^{b} f = \int_{a}^{b} g.
	\]

	Sea $\overline{\mathcal{R}}([a,b])$ el conjunto de clases de equivalencia de $\mathcal{R}([a,b])$, y denotamos por $\overline{f}$ a la clase de equivalencia de $f \in \mathcal{R}([a,b])$. Con esto, $\| \overline{f} \|_{L'} \coloneq \int_{a}^{b} |f| dx$ define una norma en $\overline{\mathcal{R}} ([a,b])$ que se llama la \textbf{norma $L'$}.

	\begin{remark}
		Hay un problema: $(\overline{\mathcal{R}} ([a,b]), \| \cdot \|_{L'})$ NO ES COMPLETO!
	\end{remark}

	\begin{enumerate}
		\item[3.] \textbf{TFC:} Si $f \in \mathcal{R} ([a,b])$ es continua en $x_0 \in [a,b]$, entonces $F(x) \coloneq \int_{a}^{x} f(t) dt$ es derivable en $x_0$ y $F'(x_0) = f(x_0)$. En particular, $F$ es derivable en $x$ y $F'(x)=f(x)$ para todo $x$ salvo un conjunto de medida nula.
	\end{enumerate}


	\subsection{Clase 5 (18/08)}

	\noindent \textbf{Teorema Fundamental del Cálculo: } Si $f \in \mathcal{R}([a,b])$ es continua en $x_0 \in [a,b]$, entonces $F: [a,b] \to \R$ dada por $F(x) \coloneq \int_{a}^{x} f(t) dt$ es derivable en $x=x_0$ y vale $F'(x_0) = f(x_0)$. En particular, $F'(x) = f(x)$ salvo quizás por un conjunto de $x \in [a,b]$ de medida nula. O sea, podemos integrar y luego derivar y esto es "casi" como no hacer nada. Pero, tenemos problemas:

	\begin{enumerate}
		\item \textbf{Este "casi" no puede removerse}
		\begin{theorem}[Hankel, 1871]
			Dado $[a,b] \subseteq \R$, existe $f\in\mathcal{R}([a,b])$ tal que $F(x) \coloneq \int_{a}^{x} f(t) dt$ no es derivable para ningún $x$ en un subconjunto denso en $[a,b]$ (y, en particular, infinito).
		\end{theorem}

		\item \textbf{A veces no podemos componer en el orden inverso}
		\begin{theorem}[Volterra, 1881]
			Dado $[a,b]\subseteq\R$, existe $f: [a,b]\to\R$ derivable en $[a,b]$, tal que $f'$ es acotada en $[a,b]$ pero $f' \not\in \mathcal{R} ([a,b])$.
		\end{theorem}
	\end{enumerate}

	\noindent \textbf{Extendiendo la integral de Riemann}

	Sean $f:[a,b]\to\R$ acotada y $\Pi = \{ x_0,\dots,x_n \}$ una partición de $[a,b]$. Definimos:
	\begin{align*}
		\Phi_{f,\Pi}(x) & \coloneq m_{I_1} \chi_{[x_0,x_i]} (x) + \sum_{i=2}^{n} m_{I_i} \chi_{(x_{i-1},x_i]}(x), \quad m_{I_i} = \inf_{t \in I_i} f(t) \\
		& = m_{I_1} \chi_{\{x_0\}}(x) + \sum_{i=1}^{n} m_{I_i} \chi_{(x_{i-1},x_i]}(x) \\
		\psi_{f,\Pi} & \coloneq M_{I_1} \chi_{\{x_0\}}(x) + \sum_{i=1}^{n} M_{I_i} \chi_{(x_{i-1},x_i]}(x), \quad M_{I_i} = \sup_{t\in I_i} f(t)
	.\end{align*}

	Observemos que $\Phi_{f,\Pi}(x) \leq f(x) \leq \psi_{f,\Pi}(x) \quad \forall x \in [a,b]$. Además, 
	\[
	\int_{a}^{b} \Phi_{f,\Pi}(x) dx = \underline{S}(f,\Pi) \\
	\int_{a}^{b} \psi_{f,\Pi}(x) dx = \overline{S}(f,\Pi).
	\]
	\noindent En particular, si $f$ es Riemann integrable,
	\begin{align*}
		\int_{a}^{b} f(x) dx & = \overline{\int_{a}^{b}} f(x) dx = \inf \left\{ \int_{a}^{b} \psi_{f,\Pi} \ : \ \Pi \text{ partición} \right\} \\
		& = \underline{ \int_{a}^{b} } f(x) dx = \sup \left\{ \int_{a}^{b} \Phi_{f,\Pi} \ : \ \Pi \text{ partición} \right\}
	.\end{align*}

	\begin{definition}[función escalonada]
		Una función $\Phi : [a,b] \to \R$ se dice escalonada si existen $\Pi = \{ x_0,\dots,x_n \}$ partición de $[a,b]$ y $c_1,\dots,c_n \in \R$ tales que
		\[
		\Phi |_{(x_{i-1},x_i)} \equiv c_i \quad \forall i = 1,\dots,n
		\]
	\end{definition}

	Notemos que podemos escribir a cualquier función $\Phi$ escalonada como
	\begin{align*}
		\Phi (x) & \coloneq \sum_{i=1}^{n} c_i \cdot \chi_{(x_{i-1},x_i}(x) + \sum_{i=0}^{n} \Phi(x_i) \cdot \chi_{\{x_i\}}(x) \\
		& = \sum_{i=1}^{k} c_j \cdot \chi_{A_j}(x).
	.\end{align*}
	\noindent donde los $A_j$ son intervalos disjuntos tales que $\displaystyle\bigcupd_{j=1}^{k} A_j = [a,b]$ (se pone una "D" dentro de la unión para denotar que estamos haciendo una unión disjunta).

	Si tomamos $\Phi$ de la forma $\Phi = \sum_{j=1}^{k} c_j \cdot \chi_{A_j}$ con $(A_j)_{j=1,\dots,k}$ disjuntos, $\displaystyle\bigcupd_{j=1}^{k} A_j = [a,b]$ pero $A_j$  no son necesariamente intervalos, diremos que $\Phi$ es una función escalonada generalizada. Como para funciones escalonadas "normales", tenemos
	\[
	\int_{a}^{b} \Phi (x) dx = \sum_{j=1}^{k} c_j \cdot |A_j| \left( = \sum_{i=1}^{n} c_i \cdot |I_i| \right)
	\]

	\noindent \textbf{La función longitud}
	Sea $\mathcal{I}$ la colección de los intervalos en $\R$. Definimos la función longitud $\lambda : \mathcal{I} \to [0,\infty]$ como $\lambda (I) \coloneq |I|$.
	
	\noindent \textbf{Propiedades:}
	\begin{enumerate}
		\item $\lambda (\varnothing) = 0$;

		\item $I_1,I_2 \in \mathcal{I},\ I_1\subseteq I_2 \implies \lambda (I_1) \leq \lambda (I_2) \ (\text{Monotonía de } \lambda)$;

		\item (Aditividad finita de $\lambda$) Si $I \in \mathcal{I}$ es tal que $I = \displaystyle\bigcupd_{i=1}^{n} J_i$ con $J_i \in \mathcal{I},\ \forall i = 1,\dots,n,\ J_i \cap J_j = \varnothing$ sin $i\neq j$, entonces
		\[
		\lambda (I) = \sum_{i=1}^{n} \lambda (J_i);
		\]

		\item ($\sigma$-aditividad de $\lambda$) Si $I \in \mathcal{I}$ es tal que $I = \displaystyle\bigcup_{i=1}^{\infty} I_i $, con $(I_i)_{i} \in \N \subseteq \mathcal{I}$ disjuntos, entonces
		\[
		\lambda(I) = \sum_{i=1}^{\infty} \lambda (I_i)
		;\]

		\item ($\sigma$-subaditividad de $\lambda$) Si $I \in \mathcal{I}$ verifica $I \subseteq \displaystyle\bigcup_{i=1}^{\infty} I_i, \ (I_1)_{i \in \N})$ intervalos (no necesariamente disjuntos), entonces $\lambda (I) \leq \sum_{i=1}^{\infty} \lambda (I_i)$;

		\item $\lambda (I + x) = \lambda (I), \ \forall x \in \R, \ I+x \coloneq \{a + x \ : \ a \in I \} $;

		\item $\lambda(\{x\}) = 0 \ \forall \ x \in \R$.  
	\end{enumerate}


	\section{Clase 6 (20/08)}

	Nos gustaría extender $\lambda$ a una clase más grande que $\mathcal{I}$. Más precisamente, nos gustaría definir una aplicación $m : \mathcal{M} \to [0,\infty]$, donde $\mathcal{M}$ es una coleccción de subconjuntos de $\R$ tal que $\mathcal{I} \subseteq \mathcal{M}$, de manera tal que, dado $E \in \mathcal{M},\ m(E)$ represente la "longitud" de $E$. Idealmente, nos gustaría que $m$ cumpla lo siguiente:

	\begin{enumerate}
		\item $\mathcal{M} = \mathcal{P}(\R)$;

		\item Si $I \in \mathcal{I}$, entonces $m(I) = |I|$;

		\item $m$ es $\sigma$-aditiva ($E,\ (E_n)_{n\in\N}\in \mathcal{M},\ E = \displaystyle\bigcupd_{n=1}^{\infty} E_n \implies m(E) = \sum_{n=1}^{\infty} m(E_n)$);
	\end{enumerate}

	\begin{ex}
		$(1)+(2)+(3) \implies m$ es monóton, $\sigma$-subaditiva y finitamente aditiva. 
	\end{ex}

	\begin{enumerate}
		\item[4] Si $E \in \mathcal{M}$, entonces $E + x \in \mathcal{M}$ y $m(E + x) = m(E)\ \forall x \in \R$.
	\end{enumerate}

	\noindent El problema es que, si asumimos el Axioma de Elección, uno puede mostrar que no existe una tal $m$ que cumpla $(1)-(2)-(3)-(4)$ y, de hecho, no se sabe si existe $m$ que cumpla $(1)-(2)-(3)$. (Si asumimos la hipótesis del continuo, entonces no existe $m$ que cumpla $(1)-(2)-(3)$). \\

	\noindent Luego, para construir $m$ debemos debilitar alguna de las propiedades:
	\begin{itemize}
		\item Si debilitamos $(1) \implies$ TEORÍA DE LA MEDIDA;
		
		\item Si debilitamos $(3)$ pidiento solo (hay dos opciones):

		\begin{itemize}
			\item[$\rightarrow$] aditividad finita $ \implies$ "medidas finitamente aditivas";
		
			\item[$\rightarrow$] $\sigma$-subaditividad $\implies$ "medidas exteriores".
		\end{itemize}
	\end{itemize}

	\noindent Vamos a optar por debilitar $(1)$. \\\\
	\noindent Una manera de extender $\lambda$ es la siguiente:
	\begin{enumerate}
		\item[i.] Si $E = \displaystyle\bigcupd_{i=1}^{n} I_i$ entonces defninimos $\lambda(E) \coloneq \sum_{i=1}^{n} \lambda (I_i)$;

		\item[ii.] Si $E = \displaystyle\bigcupd_{i=1}^{\infty} I_i$ entonces definimos $\lambda(E) \coloneq \sum_{i=1}^{\infty} \lambda (I_i)$;

		\item[iii.] La fórmula anterior nos permite definir $\lambda (6)$ para todo $6$ abierto en $\R$;

		\item[iv.] Para conjuntos mas generales, "aproximar" por abiertos. 
	\end{enumerate}

	\begin{definition}[premedida]
		Sea $X$ un conjunto no vacío y $\mathscr{C}$ una colección de subconjuntos de $X$ tal que $\varnothing \in \mathscr{C}$. Diremos que una aplicación $\mathcal{T} : \mathscr{C} \to [0,\infty]$ es una premedida si $\mathcal{T} (\varnothing)=0$.
	\end{definition}

	\begin{remark}
		El conjunto no vacío $X$ será llamado un espacio y la colección $\mathscr{C}$ será llamada una clase (de subconjuntos de $X$).
	\end{remark}

	\noindent Intuitivamente, $\mathscr{C}$ representa la colección de subconjuntos cuyo "tamaño" sabemos medir y $\mathcal{T}$ nos da su medida.

	\begin{eg}~
		\begin{enumerate}
			\item \textbf{Premedida de Lebesgue:} $\mathscr{C} \coloneq \mathcal{I} \coloneq \{ I \subseteq \R \ : \ I \text{ intervalo}\}, \ \mathcal{T}(I) \coloneq |I|$.

			\item \textbf{Premedidas de Lebesgue-Stieltjes:} Sea $F:\R \to \R$ monótona creciente y continua a derecha ($\lim_{x \to x_0}^{+} F(x) = F(x_0)$). Una función tal se dice una función de Lebesgue-Stieltjes. 
		\end{enumerate}
	\end{eg}

	\noindent Observemos que, por monotonía, existen límites \[ \left\{ \begin{aligned}
		F(\infty) & \coloneq \lim_{x \to \infty} F(x) \\ 
		F(-\infty) & \coloneq \lim_{x \to -\infty} F(x) 
	\end{aligned} \right\} \in \R \]

	\noindent Sea además la clase $\widetilde{\mathcal{I}}$ de intervalos de $\R$ dada por
	\begin{align*}
		\widetilde{\mathcal{I}} & \coloneq \{ I(a,b) \ : \ \} \text{ donde } I(a,b) \coloneq (a,b] \cap \R \\
		& = \{ (a,b] \ : \ -\infty \leq a \leq b \} \cup \{ (a,\infty) \ : \ -\infty \leq a < \infty \}. 
	.\end{align*}

	\noindent Definimos la premedida $\mathcal{T}_F$ de Lebesgue-Stieltjes asociada a $F$ como la aplicación $\mathcal{T}_F: \widetilde{\mathcal{I}} \to [0, \infty]$, dada por
	\[ \mathcal{T}_F (I(a,b)) = F(b) - F(a) .\]

	\begin{note}
		Observar que si $F(x)=x$ entonces $\mathcal{T}_F$ es la premedida de Lebesgue (sobre $\widetilde{\mathcal{I}}$.
	\end{note}

	\begin{enumerate}
		\item[3.] \textbf{Premedidas de Probabilidad:} Si $F$ es una función de L-S tal que $F(\infty) = 1$ y $F(-\infty) = 0$, decimos que $F$ es una función de distribución (acumulada). En tal caso, la premedida $\mathcal{T}_F$ se conoce como premedida de probabilidad o predistribución (en $\R$).
	\end{enumerate}

	\begin{remark}
		$\mathcal{T}_F (\R) = \mathcal{T}_F (I(-\infty,\infty)) = F(\infty) - F(-\infty) = 1 - 0 = 1$.
	\end{remark}

	\begin{enumerate}
		\item[4.] \textbf{Premedida...}
	\end{enumerate}

%##################################################################%
%##################################################################%
%##################################################################%

	\section{Clase 7 (22/08)}

	\begin{definition}[semiálgebra]
		Sea $X$ un espacio y $\mathscr{C}$ una clase de subconjuntos de $X$. Decimos que $\mathscr{C}$ es una semiálgebra (de subconjuntos de $X$) si cumple:
		\begin{enumerate}
			\item $\varnothing \in \mathscr{C}$;

			\item ($\mathscr{C}$ es cerrada por intesecciones finitas) $A,B\in\mathscr{C} \implies A \cap B \in \mathscr{C}$;
			
			\item Si $A \in \mathscr{C}$, existen $C_1,\dots,C_n \in \mathscr{C}$ disjuntos tal que $A^c = \displaystyle\bigcupd_{i=1}^{n} C_i$.
		\end{enumerate}
	\end{definition}
	
	\begin{eg}~
		\begin{enumerate}
			\item La clase $\mathcal{I}_d$ de intervalos en $\R^d$ es una semiálgebra.

			\item La clase $\widetilde{\mathcal{I}} \coloneq \{ (a,b] \cap \R \ : \ -\infty \leq a \leq b \leq \infty \}$ es una semiálgebra.

			\item Si $X$ e $Y$ son espacios y $\mathscr{C}_X, \mathscr{C}_Y$ son semiálgebras en $X$ e $Y$ respectivamente, entonces
			\[ \mathscr{C}_X \times \mathscr{C}_Y \coloneq \{ F \times G \ : \ F \in \mathscr{C}_X,\ G \in \mathscr{C}_Y \} \]
			es una semiálgebra en $X \times Y$, llamada "semiálgebra producto".
		\end{enumerate}
	\end{eg}

	\begin{definition}[álgebra]
		Sean $X$ un espacio y $\mathscr{A}$ una clase de subconjuntos de $X$. Decimos que $\mathscr{A}$ es un álgebra (de subconjuntos de $X$) si cumple que:
		\begin{enumerate}
			\item[(i)] $\varnothing \in \mathscr{A}$;
			
			\item[(ii)] $\mathscr{A}$ es cerrado por intersecciones finitas;

			\item[(iii)] ($\mathscr{A}$ es cerrada por complementos) $A \in \mathscr{A} \implies A^c \in \mathscr{A}$. 
		\end{enumerate}	
			\noindent Equivalentemente, en presencia de (iii), (ii) se puede reemplazar por:	
		\begin{enumerate}
			\item[(ii')] ($\mathscr{A}$ es cerrada por uniones finitas) $A,B\in \mathscr{A} \implies A \cup B \in \mathscr{A}$. (\textbf{Dem:} Ejercicio!)
		\end{enumerate}
	\end{definition}

	\begin{eg}~
		\begin{enumerate}
			\item $X$ espacio, $\mathscr{A}_1 \coloneq \{\varnothing, X\},\ \mathscr{A}_2 \coloneq \mathcal{P}(X)$ son álgebras (donde $\mathscr{A}$ es llamada el álgebra trivial);

			\item Sea $\mathscr{S}$ una semiálgebra de subconjuntos de un espacio $X$. Entonces 
			\[ \mathscr{A} \coloneq \{ E \subseteq X \ : \ \exists S_1,\dots,S_n \in \mathscr{S} \text{ disjuntos tal que } E = \displaystyle\bigcupd_{i=1}^{n} S_i \} \] 
			es un álgebra, llamada el álgebra generada por $\mathscr{S}$. Notemos que $\mathscr{A}(\mathscr{S}$ es el menor álgebra que contiene a $\mathscr{S}$:
			\begin{enumerate}
				\item[(i)] $\mathscr{A}(\mathscr{S})$ es un álgebra y $\mathscr{S} \subseteq \mathscr{A}(\mathscr{S})$;

				\item[(ii)] Si $\mathscr{A}'$ es un álgebra con $\mathscr{S} \subseteq \mathscr{A}'$ entonces $\mathscr{A}(\mathscr{S} \subseteq \mathscr{A}'$.
			\end{enumerate}
		\end{enumerate}
	\end{eg}

	\begin{note}
		Toda álgebr es una semiálgebra.
	\end{note}

	\begin{definition}[$\sigma$-álgebra]
		Una clase (no vacía) $\mathscr{M}$ de subconjuntos de un espacio $X$ se dice una $\sigma$-álgebra si cumple:
		\begin{enumerate}
			\item $\varnothing \in \mathscr{M}$;

			\item $E \in \mathscr{M} \implies E^c \in \mathscr{M}$;

			\item $(E_n)_{n\in\N} \subseteq \mathscr{M} \implies \displaystyle\bigcup_{n\in\N} E_n \in \mathscr{M}$.
		\end{enumerate}
		\noindent Llamamos al par $(X,\mathscr{M})$ un \underline{espacio medible} y a los elementos de $\mathscr{M}$, \underline{conjuntos medibles}.
	\end{definition}

	\begin{note}~
		\begin{enumerate}
			\item Todo $\sigma$-álgebra es un álgebra;

			\item Equivalentemente, en presencia de (1), (3) se puede reemplazar por
			\begin{enumerate}
				\item[(iii')] $(E_n)_{n\in\N} \subseteq \mathscr{M} \implies \displaystyle\bigcap_{n\in\N} E_n \in \mathscr{M}$.
			\end{enumerate}
		\end{enumerate}
	\end{note}

	\begin{eg}~
		\begin{enumerate}
			\item $\sigma$-álgebra $\implies$ álgebra $\implies$ semiálgebra (no valen las recíprocas);

			\item $\{\varnothing,X\},\mathcal{P}(X)$ son $\sigma$-álgebras;

			\item Si $(\mathscr{M}_{\gamma})_{\gamma \in \Gamma}$ son $\sigma$-álgebras, entonces
			\[ \displaystyle\bigcap_{\gamma\in\Gamma} \mathscr{M}_{\gamma} \coloneq \{ E \subseteq X \ : \ E \in \mathscr{M}_{\gamma},\ \forall \gamma \in \Gamma \} \]
			es una $\sigma$-álgebra.
			
			\item Si $\mathscr{M}$ es una clase de subconjuntos de $X$, entonces
			\[ \sigma(\mathscr{M}) \coloneq \displaystyle\bigcap_{\begin{aligned}
				\mathscr{M}& \ \sigma\text{-álgebra} \\
				& \mathscr{C}\subseteq\mathscr{M}
			\end{aligned}} \mathscr{M} \]
			es la $\sigma$-álgebra generada por $\mathscr{C}$. De hecho, $\sigma(\mathscr{M})$ es la menor $\sigma$-álgebra que contiene a $\mathscr{C}$:
			\begin{enumerate}
				\item $\sigma(\mathscr{C})$ es $\sigma$-álgebra y $\mathscr{C} \subseteq \sigma(\mathscr{C})$;

				\item Si $\mathscr{F}$ es $\sigma$-álgebra y $\mathscr{C} \subset \mathscr{F}$ entonces $\sigma(\mathscr{C}) \subseteq \mathscr{F}$.
			\end{enumerate}

			\item Si $(X,\mathscr{T})$ es un espacio topológico, $\sigma(\mathscr{T})$ se conoce como la \underline{$\sigma$-álgebra} \underline{de Borel}, y sus elementos se llaman \underline{Borelianos}. La notamos $\beta(X)\ (= \sigma(\mathscr{T}))$.
		\end{enumerate}
	\end{eg}
	
	\begin{eg}
		$\beta(\R)$ contiene a tods los abiertos, cerrados, intervalos, conjuntos de tipo $G_{\delta}$ y $F_{\sigma},\dots$ De hecho, $\beta(\R)=\sigma(\text{cerrados})=\sigma(\text{compactos})=\sigma(\mathcal{I})=\sigma(\widetilde{\mathcal{I}})$.
	\end{eg}

	\begin{definition}
		Sea $\mathscr{C}$ una clase (no vacía) de subconjuntos de $X$ y $\mu : \mathscr{C} \to [0,\infty]$ una función (la llamamos una \underline{función de conjuntos}). Diremos que:
		\begin{enumerate}
			\item[(i)] \textbf{$\mu$ es monótona} (en $\mathscr{M}$) si $A,B\in\mathscr{C},\ A \subseteq B \implies \mu(A)\leq\mu(B)$;

			\item[(ii)] \textbf{$\mu$ es finitamente aditiva} si $(A_i)_{i=1,\dots,n} \subseteq \mathscr{C}$ disjuntos $\implies \mu(\bigcupd_{i=1}^{n} A_i) = \sum_{i=1}^{n} \mu(A_i)$;

			\item[(iii)] \textbf{$\mu$ es $\sigma$-aditiva} si $(A_n)_{n\in\N} \subseteq \mathscr{C}$ disjuntos $\implies \mu (\bigcupd_{i=1}^{\infty} A_i ) = \sum_{i=1}^{\infty} \mu(A_i)$;

			\item[(iv)] \textbf{$\mu$ es $\sigma$-subaditiva} si $\mu(A)\leq \sum_{i=1}^{\infty}\mu(A_n)$, para todo $A \in \mathscr{C}$ y $(A_n)_{n\in\N}\subseteq\mathscr{C}$ tal que $A \subseteq \bigcup_{n\in\N} A_n$
		\end{enumerate}
	\end{definition}

%##################################################################%
%##################################################################%
%##################################################################%

\section{Clase 8 (25/08)}

\begin{remark}
	Rana da una definición más débil de (4):
	\[ A \in \mathscr{C},\ A = \bigcup_{i=1}^{\infty} A_i,\ A_i \in \mathscr{C}\ \forall i \implies \mu(A) \leq \sum_{i=1}^{\infty}\mu(A_i) \]
	\noindent Ambas definiciones son equivalentes si $\mathscr{C}$ es una semiálgebra y $\mu$ es monótona (siempre será el caso para nosotros).
\end{remark}

\begin{definition}[premedida finita y $\sigma$-finita]
	Una premedida $\mathcal{T} : \mathscr{C} \to [0,\infty]$ se dice:
	\begin{enumerate}
		\item \textbf{finita} si $X \in \mathscr{C}$ y $\mathcal{T} < \infty$;

		\item \textbf{$\sigma$-finita} si existen $(C_n)_{n\in\N} \subseteq \mathscr{C}$ \underline{disjuntos} tales que $\bigcupd_{n=1}^{\infty} C_n = X$ y $\mathcal{T} (C_n) < \infty\ \forall n \in \N$.
	\end{enumerate}
\end{definition}

\begin{eg}~
	\begin{enumerate}
		\item finita $\implies \sigma$-finita;

		\item La función longitud $\lambda : \mathcal{I} \to [0,\infty]$ es $\sigma$-finita pero no finita;

		\item Si $F$ es una función de L-S, entonces $\mathcal{T}_F : \widetilde{\mathcal{I}} \to [0,\infty]$ es siempre $\sigma$-finita $(\mathcal{T}_F ((n,n+1]) = F(n+1) - F(n) < \infty\ \forall n \in \Z)$ y es finita si y sólo si $\mathcal{T}_F (\R) = \mathcal{T}_F ((-\infty,\infty]\cap\R) = F(\infty)-F(-\infty) < \infty$.
	\end{enumerate}
\end{eg}

\begin{definition}[medida]
	Sea $(X,\mathscr{M})$ es un espacio medible. Diremos que $\mu : \mathscr{M} \to [0,\infty]$ es una \underline{medida} (en $(X,\mathscr{M})$) si:
	\begin{enumerate}
		\item $\mu (\varnothing) = 0$;
		
		\item $\mu$ es $\sigma$-subaditiva en $\mathscr{M}\ \left( \mu \left(\bigcupd_{i=1}^{\infty} A_i \right) = \sum_{i=1}^{\infty} \mu(A_i) \right)$.
	\end{enumerate}
	\noindent Llamamos a la terna $(X,\mathscr{M},\mu)$ un \underline{epacio de medida}.
\end{definition}

\noindent \textbf{Objetivo.} Construir un espacio de medida $(\R,\mathscr{M},\mu)$ tal que $\mathcal{I} \subseteq \mathscr{M}$ y
\[ \begin{cases}
	\mu(I) = |I|\ \forall I \in \mathcal{I}, \\
	\mu(E+x) = \mu(E) \ \forall E \in \mathscr{M}.
\end{cases} \]

\begin{eg}[Espacios de Probabilidad]
	Si $(X,\mathscr{M},\mu)$ es un EdM tal que $\mu(X)=1,\ (X,\mathscr{M},\mu)$ recibe el nombre de \underline{espacios de probabilidad}.
\end{eg}

\begin{itemize}
	\item $X$ recibe el nombre de \underline{espacio muestral}, y se lo nota $\Omega$ (en lugar de $X$);

	\item $\mathscr{M}$ se suele notar como $\mathscr{F}$ (ó $\mathscr{Y}$). Sus elementos se dicen \underline{eventos};

	\item $\mu$ recibe el nombre de \underline{medida de probabilidad} ó \underline{distribución} y se la nota $\mathbb{P}$.
\end{itemize}

\noindent En probabilidad, típicamente se estudian $2$ tipos de distribuciones en $\R$ (o en $\R^d$).

\begin{enumerate}
	\item \textbf{Distribuciones discretas:} $\exists S \subseteq \R$ numerable y $(p_x)_{x \in S} \subseteq [0,1]$ tal que $\mathbb{P}(A) = \sum_{x \in A \cap S} p_x$.
	\begin{eg}
		Binomial, Geométrica, Poisson,...
	\end{eg}

	\item \textbf{Distribuciones (absolutamente) continuas:} $\exists f : \R \to \R_{\geq 0}$ "integrable" tal que $\mathbb{P}(A) = \int_{A} f(x) dx$.
	\begin{eg}
		Uniforme, Exponencial, Normal,...
	\end{eg}
\end{enumerate}

\noindent \textbf{Propiedades generales de una medida.} Si $\mu$ es una medida sobre $(X,\mathscr{M})$, entonces:
\begin{enumerate}
	\item $\mu$ es monótona (en $\mathscr{M}$);

	\item $\mu$ es $\sigma$-subaditiva;

	\item $\mu$ es \textbf{continua por debajo}: si $(A_n)_{n\in\N} \subseteq \mathscr{M}$ es \underline{creciente} $(A_n \subseteq A_{n+1}\ \forall n)$ entonces
	\[ \mu \left( \bigcup_{n\in\N} A_n \right) = \lim_{n \to \infty} \mu(A_n). \]

	\item $\mu$ es \textbf{continua por arriba}: si $(A_n)_{n\in\N} \subseteq \mathscr{M}$ es \underline{decreciente} $(A_{n+1} \subseteq A_n\ \forall n)$ y $\mu(A_{n_0})<\infty$ para algún $n_0\ (\implies \mu(A_n)<\infty\ \forall n\geq n_0)$, entonces
	\[ \mu \left( \bigcap_{n\in\N}A_n \right) = \lim_{n \to \infty} \mu(A_n). \]

	\noindent (\textbf{Cuidado!} (4) puede no valer si $\mu (A_n) = \infty \ \forall n \in \N$)
\end{enumerate}

\begin{definition}[premedida extendible y unívocamente extendible]
	Una premedida $\mathcal{T} : \mathscr{S} \to [0,\infty]$ definida sobre una semiálgebra de subconjunto de $X$, se dice:
	\begin{enumerate}
		\item \textbf{Extendible} si es
		\begin{enumerate}
			\item[(E1)] finitamente aditiva en $\mathscr{S}$;

			\item[(E2)] $\sigma$-subaditiva en $\mathscr{S}$.
		\end{enumerate}

		\item \textbf{Unívocamente extendible} si es extendible y se cumple
		\begin{enumerate}
			\item[(E3)] $\sigma$-finita
		\end{enumerate}
	\end{enumerate}
\end{definition}

\begin{remark}
	Los nombres de extendible y unívocamente extendible no se encontrarán en el Rana (los puso el profe).
\end{remark}

\begin{theorem}[Extensión de Carathéodory]
	Dados un espacio $X$ y una premedida $\mathcal{T}$ sobre una semiálgebra $\mathscr{S}$ de subconjuntos de $X$ tal que $\mathcal{T}$ es extendible, existe una extensión de $\mathcal{T}$ a una medida $\mu_{\mathcal{T}}$ definida sobre $\sigma(\mathscr{S})$ la $\sigma$-álgebra generada por $\mathscr{S}$. Más aún, si $\mathcal{T}$ es unívocamente extendible, entonces la extensión $\mu_{\mathcal{T}}$ a $\sigma(\mathscr{S})$ es \underline{única}. \\
	Por último, si $\mathcal{T}$ es unívocamente extendible, entonces se puede extender de manera única a una medida $\overline{\mu_{\mathcal{T}}}$ sobre la $\mu_{\mathcal{T}}$-completación de $\sigma(\mathscr{S})$, i.e. la $\sigma$-álgebra $\overline{\sigma(\mathscr{S})}$ dada por
	\[ \overline{\sigma(\mathscr{S})} \coloneq \{ B \cup N : B \in \sigma(\mathscr{S}), \exists \widetilde{N} \in \sigma(\mathscr{S}) \text{ con } N \subseteq \widetilde{N} \text{ y } \mu_{\mathcal{T}} (\widetilde{N}) = 0 \} \]
	\noindent mediante la fórmula $\overline{\mu_{\mathcal{T}}}(B\cap N) \coloneq \mu_{\mathcal{T}}(B)$.
\end{theorem}

%###################################################################################%
%###################################################################################%
%###################################################################################%

\section{Clase 9 (27/08)}

\begin{remark}
	Si $\mathcal{T} : \mathscr{S} \to [0,\infty]$ es $\sigma$-aditiva en $\mathscr{S}$ y $\mathscr{S}$ es una semiálgebra, entonces $\mathcal{T}$ es extendible.
\end{remark}

\begin{remark}
	La extensión puede no ser única si $\mathcal{T}$ no es $\sigma$-finita.
\end{remark}
\begin{eg}
	$\widetilde{\mathcal{I}}_{\Q} \coloneq \widetilde{\mathcal{I}} \cap \Q = \{ (a,b] \cap \Q \ : \ -\infty \leq a \leq b \leq \infty \}$ 
\end{eg}

\begin{note}~
	\begin{itemize}
		\item $\widetilde{\mathcal{I}}_{\Q}$ es una semiálgebra;

		\item $\sigma (\widetilde{\mathcal{I}}_{\Q}) = \sigma (\widetilde{\mathcal{I}} \cap \Q) \stackrel{\text{Ej!}}{=} \sigma (\widetilde{\mathcal{I}}) \cap \Q = \beta(\R) \cap \Q = \mathcal{P}(\Q)$ (9.52)

		\item $\mathcal{T} : \widetilde{\mathcal{I}}_{\Q} \to [0,\infty]$, dada por $\mathcal{T}(A) \coloneq \begin{cases}
			0 \quad A = \varnothing \\
			\infty \quad A \neq \varnothing,\ A \in \widetilde{\mathcal{I}}_{\Q}
		\end{cases}$ (Observar que $\mathcal{T}$ no es $\sigma$-finita)

		\item Para cada $r > 0,\ \mu_r : \mathcal{P}(\Q) \to [0,\infty]$ dada por $\mu_r(A) \coloneq r(\# A)$ es una extensión de $\mathcal{T}$ (y es una medida)
	\end{itemize}
\end{note}

\begin{definition}[espacio completo y conjuntos $\mu$-nulos]
	Sea $(X,\mathscr{M},\mu)$ un EdM y definamos 
	\[ \mathscr{N}_{\mu} \coloneq \{ E \subset X \ : \ \exists N \in \mathscr{M} \text{ con } E \subseteq N \text{ y } \mu(N) = 0 \} \] 
	Los elementos de $\mathscr{N}_{\mu}$ se dicen \underline{conjuntos $\mu$-nulos}. Diremos que $(X,\mathscr{M},\mu)$ es \underline{completo} si $\mathscr{N}_{\mu} \subseteq \mathscr{M}$
\end{definition}

\begin{remark}
	$(X,\overline{\sigma(\mathscr{S})},\overline{\mu_{\delta}})$ es \underline{completo}. En efecto, $\mathscr{N}_{\overline{\mu_{\delta}}}$ corresponde al subconjunto de $\overline{\sigma(\mathscr{S})}$ que se obtiene tomando $B=\varnothing$.
\end{remark}

\begin{remark}
	Veremos más adelante que las siguientes premedidas son UE:

	\begin{itemize}
		\item[(i)] Premedidas de Lebesgue-Stieltjes (en particular, la función longitud $\lambda$ (sobre $\widetilde{\mathcal{I}}$) y las premedidas de probabilidad).

		\item[(ii)] Premedidas de Lebesgue en $\R^d$, con $d \in \N$.
	\end{itemize}
\end{remark}

\noindent En particular;

\begin{corollary}
	Para cada función $F$ de Lebesgue-Stieltjes, existe una $\sigma$-álgebra $\mathscr{M}_F$ sobre $\R$ y una única medida $\mu_F$ en $(\R,\mathscr{M}_F)$ tal que
	\[ \mu_F = (I(a,b)) = F(b) - F(a) \quad \forall -\infty \leq a \leq b \leq \infty \]
	Además, $\beta(\R) \subseteq \mathscr{M}_F$. Es decir, $\mu_F$ es una medida que extiende a $\mathcal{T}_F$, a todo $\mathscr{M}_F$ (y en particular, a todo $\beta(\R)$). Además, $(\R, \mathscr{M}_F, \mu_F)$ es un EdM completo. ($\mathscr{M}_F \coloneq \overline{\sigma(\widetilde{\mathcal{I}})^F},\ \mu_F \coloneq \overline{\mu_{\mathcal{T}_F}}$). La medida $\mu_F$ se conoce como \underline{medida de L-S asociada a $F$}. En particular, para cualquier función de distribución $F$, existe una única medida de probabilidad $\mathbb{P}_F$ en $(\R,\beta(\R))$ tal que 
	\[ \mathbb{P}_F(I(a,b)) = F(b) - F(a) \quad \forall -\infty \leq a \leq b \leq \infty \]
	(En la guía 3 veremos que $F \to \mathbb{P}_F$ es una biyección)
\end{corollary}

\begin{note}
	Los $\beta$ son los Borelianos y $I(a,b) = (a,b] \cap \R$. (super $F \to$ 10.26).
\end{note}

\begin{eg}[Importante!]
	\textbf{Medida de Lebesgue en $\R$.} Tomando $F = id$ en el Corolario anterior, obtenemos una $\sigma$-álgebra $\mathscr{L}(\R) \coloneq \mathscr{M}_{id}$ con $\beta(\R) \subseteq \mathscr{L}(\R)$ y una medida $\mu_{id}$ en $(\R,\mathscr{L}(\R))$ tal que $\mu_{id}(I(a,b)) = b-a \quad \forall -\infty \leq a \leq b \leq \infty$. En particular, de esto se deduce que $\mu_{id}(I) = |I|\quad \forall I \in \mathcal{I}$. Dicha medida recibe el nombre de \underline{medida de Lebesgue} (en $\R$), y los elementos de $\mathscr{L}(\R)$ se dicen \underline{conjuntos medibles Lebesgue}. Adoptaremos la notación $\mu_{id}(E) \coloneq \lambda(E) \coloneq |E|$. La medida $\mu_{id}$ \underline{es} la extensión de la noción de longitud que buscábamos y $\mathscr{L}(\R)$ son los conjuntos cuya "longitud" podremos medir. Además, los conjuntos de medida nula (de la guía 2), son \underline{exactamente} aquellos $A \in \mathscr{L}(\R)$ tal que $\mu_{id}(A) = 0$ (lo veremos más adelante!).
\end{eg}

\begin{eg}[Medida de Lebesgue en $\R^d$]
	Si $\mathcal{I}_d$ son los intervalos en $\R^d$ y definimos $\mathcal{T}:\mathcal{I}_d \to [0,\infty]$ como $\mathcal{T}(I)\coloneq|I|$, entonces $\mathcal{I}_d$ es una semiálgebra y $\mathcal{T}$ es una premedida $\sigma$-aditiva en $\mathcal{I}_d$ (lo veremos después). Por lo tanto, $\mathcal{T}$ se puede extender (de manera única, pues $\mathcal{T}$ es $\sigma$-finita) a una medida $\mu_{\delta}$ sobre la $\sigma$-álgebra $\mathscr{L}(\R^d) = \overline{\sigma(\mathcal{I}_d)^{\mathcal{T}}}$, llamada \underline{medida de Lebesgue en $\R^d$} y $\mathscr{L}(\R^d)$ es la clase de \underline{conjuntos medibles Lebesgue en $\R^d$}. Al igual que antes, dado $E \in \mathscr{L}(\R^d)$, notamos $|E| \coloneq \mu_{\mathcal{T}}(E)$.
\end{eg}




% end lessons
\end{document}
