\documentclass[a4paper]{report}
\usepackage[utf8]{inputenc}
\usepackage[T1]{fontenc}
\usepackage{textcomp}

\usepackage{url}

% \usepackage{hyperref}
% \hypersetup{
%     colorlinks,
%     linkcolor={black},
%     citecolor={black},
%     urlcolor={blue!80!black}
% }

\usepackage{graphicx}
\usepackage{float}
\usepackage[usenames,dvipsnames]{xcolor}

% \usepackage{cmbright}

\usepackage{amsmath, amsfonts, mathtools, amsthm, amssymb}
\usepackage{mathrsfs}
\usepackage{cancel}

\newcommand\N{\ensuremath{\mathbb{N}}}
\newcommand\R{\ensuremath{\mathbb{R}}}
\newcommand\Z{\ensuremath{\mathbb{Z}}}
\renewcommand\O{\ensuremath{\emptyset}}
\newcommand\Q{\ensuremath{\mathbb{Q}}}
\newcommand\C{\ensuremath{\mathbb{C}}}
\let\implies\Rightarrow
\let\impliedby\Leftarrow
\let\iff\Leftrightarrow
\let\epsilon\varepsilon

% demostraciones bidireccionales

\newcommand{\Onlyifstep}{%
	\begingroup
	\fboxsep=1pt
	\raisebox{1.2ex}{\fbox{\raisebox{-1.2ex}{$\Rightarrow$\hspace{-0.05em}}}}%
	\endgroup
	\hspace{0.5em}%
	}
\newcommand{\Ifstep}{%
	\begingroup
	\fboxsep=1pt
	\raisebox{1.2ex}{\fbox{\raisebox{-1.2ex}{\hspace{-0.05ex}$\Leftarrow$}}}%
	\endgroup
	\hspace{0.5em}%
	}

% horizontal rule
\newcommand\hr{
    \noindent\rule[0.5ex]{\linewidth}{0.5pt}
}

\usepackage{tikz}
\usepackage{tikz-cd}

% theorems
\usepackage{thmtools}
\usepackage[framemethod=TikZ]{mdframed}
\mdfsetup{skipabove=1em,skipbelow=0em, innertopmargin=5pt, innerbottommargin=6pt}

\theoremstyle{definition}

\makeatletter

\declaretheoremstyle[headfont=\bfseries\sffamily, bodyfont=\normalfont, mdframed={ nobreak } ]{thmgreenbox}
\declaretheoremstyle[headfont=\bfseries\sffamily, bodyfont=\normalfont, mdframed={ nobreak } ]{thmredbox}
\declaretheoremstyle[headfont=\bfseries\sffamily, bodyfont=\normalfont]{thmbluebox}
\declaretheoremstyle[headfont=\bfseries\sffamily, bodyfont=\normalfont]{thmblueline}
\declaretheoremstyle[headfont=\bfseries\sffamily, bodyfont=\normalfont, numbered=no, mdframed={ rightline=false, topline=false, bottomline=false, }, qed=\qedsymbol ]{thmproofbox}
\declaretheoremstyle[headfont=\bfseries\sffamily, bodyfont=\normalfont, numbered=no, mdframed={ nobreak, rightline=false, topline=false, bottomline=false } ]{thmexplanationbox}


\declaretheorem[numberwithin=chapter, style=thmgreenbox, name=Definition]{definition}
\declaretheorem[sibling=definition, style=thmredbox, name=Corollary]{corollary}
\declaretheorem[sibling=definition, style=thmredbox, name=Proposition]{prop}
\declaretheorem[sibling=definition, style=thmredbox, name=Theorem]{theorem}
\declaretheorem[sibling=definition, style=thmredbox, name=Lemma]{lemma}



\declaretheorem[numbered=no, style=thmexplanationbox, name=Proof]{explanation}
\declaretheorem[numbered=no, style=thmproofbox, name=Proof]{replacementproof}
\declaretheorem[style=thmbluebox,  numbered=no, name=Exercise]{ex}
\declaretheorem[style=thmbluebox,  numbered=no, name=Example]{eg}
\declaretheorem[style=thmblueline, numbered=no, name=Remark]{remark}
\declaretheorem[style=thmblueline, numbered=no, name=Note]{note}

\renewenvironment{proof}[1][\proofname]{\begin{replacementproof}}{\end{replacementproof}}

\AtEndEnvironment{eg}{\null\hfill$\diamond$}%

\newtheorem*{uovt}{UOVT}
\newtheorem*{notation}{Notation}
\newtheorem*{previouslyseen}{As previously seen}
\newtheorem*{problem}{Problem}
\newtheorem*{observe}{Observe}
\newtheorem*{property}{Property}
\newtheorem*{intuition}{Intuition}


\usepackage{etoolbox}
\AtEndEnvironment{vb}{\null\hfill$\diamond$}%
\AtEndEnvironment{intermezzo}{\null\hfill$\diamond$}%




% http://tex.stackexchange.com/questions/22119/how-can-i-change-the-spacing-before-theorems-with-amsthm
% \def\thm@space@setup{%
%   \thm@preskip=\parskip \thm@postskip=0pt
% }

\usepackage{xifthen}

\def\testdateparts#1{\dateparts#1\relax}
\def\dateparts#1 #2 #3 #4 #5\relax{
    \marginpar{\small\textsf{\mbox{#1 #2 #3 #5}}}
}

\def\@lesson{}%
\newcommand{\lesson}[3]{
    \ifthenelse{\isempty{#3}}{%
        \def\@lesson{Lecture #1}%
    }{%
        \def\@lesson{Lecture #1: #3}%
    }%
    \subsection*{\@lesson}
    \testdateparts{#2}
}

% fancy headers
\usepackage{fancyhdr}
\pagestyle{fancy}

% \fancyhead[LE,RO]{Gilles Castel}
\fancyhead[RO,LE]{\@lesson}
\fancyhead[RE,LO]{}
\fancyfoot[LE,RO]{\thepage}
\fancyfoot[C]{\leftmark}
\renewcommand{\headrulewidth}{0pt}

\makeatother

% figure support (https://castel.dev/post/lecture-notes-2)
\usepackage{import}
\usepackage{xifthen}
\pdfminorversion=7
\usepackage{pdfpages}
\usepackage{transparent}
\newcommand{\incfig}[1]{%
    \def\svgwidth{\columnwidth}
    \import{./figures/}{#1.pdf_tex}
}

% %http://tex.stackexchange.com/questions/76273/multiple-pdfs-with-page-group-included-in-a-single-page-warning
\pdfsuppresswarningpagegroup=1

\author{Gilles Castel}

\DeclareMathOperator{\supp}{supp}
\DeclareMathOperator{\spann}{span}
\DeclareMathOperator{\Id}{Id}
\DeclareMathOperator{\Ker}{Ker}
\DeclareMathOperator{\im}{Im}
\DeclareMathOperator{\GL}{GL}
\DeclareMathOperator{\SL}{SL}
\DeclareMathOperator{\Mat}{Mat}

\usepackage{stackengine,scalerel}

\newcommand{\bigcupcolon}{\mathop{\ThisStyle{%
  \ensurestackMath{\stackinset{c}{-0.4pt}{c}{+.25\LMex}{\tiny{D}}{\SavedStyle\bigcup}}}}}

\title{Ayudantía de Estadística para Matemáticas (no programación)}
\author{}
\date{Basado en las ayudantías impartidas por - en el segundo semeste del 2025}
\begin{document}
%title
\maketitle
\tableofcontents
% start lessons

\chapter{I1}
\setcounter{section}{0}
\section{Ayudantía 2}
\subsection{Probabilidad Condicional}
\begin{enumerate}
	\item \textbf{Monty Hall.}

	\begin{proof}
		$p\in \{1,\dots,n\},\ A = \{ p \text{ fuese auto}\},\ C = \{ \text{ganar al cambiar} \}$. Queremos comparar: $\mathbb{P}(C)$ y $\mathbb{P}(A)$. Notemos
		\[ \mathbb{P}(C) = \mathbb{P}(C|A)\mathbb{P}(A) + \mathbb{P}(C|A^c)\mathbb{P}(A^c) \]
		\[ \mathbb{P}(A) = \frac{t}{n};\quad \mathbb{P}(A^c) = 1 - \frac{t}{n} = \frac{n-t}{n} \]
		\[ \mathbb{P}(C|A) = \frac{t-1}{n-k-1},\quad \mathbb{P}(C|A^c) = \frac{t}{n-k-1} \]
		Entonces,
		\begin{align*}
			\mathbb{P}(C) & = \left( \frac{t-1}{n-k-1} \right) \left( \frac{t}{n} \right) \left( \frac{t}{n-k-1} \right) \left( \frac{n-t}{n} \right) \\
			& = \frac{t}{n} \left( \frac{(t-1) + n - t}{n-k-1} \right) \\
			& = \left( \frac{t}{n} \right) \left(\frac{n-1}{n-k-1} \right)
		\end{align*}
		Luego,
		\begin{align*}
			\mathcal{P}(C) \geq \mathcal{P}(A) & \iff \frac{t}{n} \left( \frac{n-1}{n-k-1} \right) \geq \frac{t}{n} \\
			& \iff n-1 \geq n-k-1 \\
			& \iff k \geq 0
		.\end{align*}
		Hay que cambiarse!
	\end{proof}

	\item \textbf{Bella Durmiente.}
	\begin{proof}
		Eventos:
		\begin{align*}
			L & = \{\text{despertó el Lunes}\} \\
			M & = \{\text{despertó el martes}\} \\
			D & = L \sqcup M = \{\text{despertó}\} \\
			C & = \{\text{haya salido cara}\}  
		.\end{align*}
		Notar que, como $\mathbb{P}(M)\mathbb{P}(C|M) = \mathbb{P}(C\cap M) = \mathbb{P}(\underbrace{M|C}_{0})\mathbb{P}(C)$, entonces
		\begin{align*}
			\mathbb{P}(C|D) & = \mathbb{P}(C|L)\mathbb{P}(L) + \mathbb{P}(\underbrace{C|M}_{0})\mathbb{P}(M) \\
			& = \frac{1}{2} \cdot \frac{2}{3} \\
			& = \frac{1}{3}
		.\end{align*}
	\end{proof}
\end{enumerate}

\subsection{Independencia de Eventos}

\begin{enumerate}
	\item \textbf{Paradoja del cumpleaños.}
	\begin{proof}[Proof ]
		$E_i = \{ \text{Persona } i \text{ cumple años en distinta fecha que los } i-1 \text{ anteriores} \} $. Vamos a calcular $\mathbb{P}\left( \bigcap_{i=1}^{n} E_i \right)$ (la intesección es que cumplen en distinta fecha). Notemos que
		\begin{align*}
			\mathbb{P}(E_1\cap E_2) & = \mathbb{P}(E_2|E_1)\mathbb{P}(E_1) \\
			\mathbb{P}((E_1\cap E_2)\cap E_3) & = \mathbb{P}(E_3|E_1\cap E_2) \mathbb{P}(E_1\cap E_2) \\
			& = \mathbb{P}(E_3|E_1\cap E_2) \mathbb{P}(E_2|E_1) \mathbb{P}(E_1)
		.\end{align*}
		esto se puede seguir por induccción. Luego,
		\begin{align*}
			\mathbb{P}\left(\bigcap_{i=1}^{n} E_i \right) & = \mathbb{P}(E_1)\mathbb{P}(E_2|E_1)\cdots \mathbb{P}(E_n|E_1 \cap \cdots \cap E_{n-1}) \\
			& = \prod_{k=1}^{k} \frac{N-(k-1)}{N} \\
			& = \frac{1}{N^n}\cdot \frac{N!}{(N-n)!} \\
			\implies p & = 1- \frac{1}{N^n}\cdot \frac{N!}{(N-n)!}
		.\end{align*}
	\end{proof}

	\item \textbf{Complemento de eventos independientes.} $A_1\dots A_n$ independientes ssi $A_n^c \dots A_n^c$ independientes
	\begin{proof}[Proof ]
		$\Rightarrow$ Notar que si $n=2$, entonces
		\begin{align*}
			\mathbb{P}(A_1^c \cap A_2^c) & = \mathbb{P}((A_1\cup A_2)^c) \\
			& = 1 - \mathbb{P}(A_1 \cup A_2) \\
			&= 1 - (\mathbb{P}(A_1) + \mathbb{P}(A_{2}) - \mathbb{P}(A_{1}\cap A_{2})) \\
			&= 1 - \mathbb{P}(A_{1}) - \mathbb{P}(A_{2}) + \mathbb{P}(A_{1})\mathbb{P}(A_{2}) \\
			&= (1-\mathbb{P}(A_{1}))(1 - \mathbb{P}(A_{2})) \\
			&= \mathbb{P}(A_1^c)\mathbb{P}(A_2^c)
		.\end{align*}
		Supongamos que se cumplepara $k<n$: $i_1\dots i_m$ indices:
		\begin{align*}
			\mathbb{P}(A_{i1}\cap \cdots \cap A_{im}) & = \prod_{j=1}^{m} \mathbb{P}(A_{ij}) \quad \text{trivial para } m<n \\
			\mathbb{P}(A_{1}\cap\cdots\cap A_{n}) &= \mathbb{P}(A_{1}\cap (A_{2}\cap\cdots\cap A_n)) \\
			&= \mathbb{P} (A_1)\mathbb{P}(A_{2}\cap\cdots\cap A_n) \\
			&= \prod_{i=1}^{n} \mathbb{P}(A_i)
		.\end{align*}
	\end{proof}
\end{enumerate}

\subsection{Un problema de Dardos}

\begin{proof}[Proof ]
	Sean
	\begin{align*}
		D_1 &= \{\text{acierto el dardo } 1\}   \\
		\mathbb{P}(D_1) & = \frac{1}{3} \\
		D_i &= \{\text{acierto el dardo } i\} \quad : \quad i \in \{2,3\}  \\
		\mathbb{P}(D_i|D_{i-1}) &= \frac{1}{2} \\
		\mathbb{P}(D_i|D_{i-1}^c) &= \frac{1}{4}
	.\end{align*}
	Por ejemplo,
	\begin{align*}
		\mathbb{P}(6) = & \mathbb{P}(D_1\cap D_2 \cap D_{3}) + \mathbb{P}(D_1^c \cap D_2 \cap D_3) \\
		& + \mathbb{P}(D_1\cap D_2^c \cap D_{3}) + \mathbb{P}(D_1\cap D_{2} \cap d_{3}^c) \\
		= & \mathbb{P}(D_{3}|D_{2})\mathbb(D_{2}|D_{1}) \mathbb{P}(D_{1}) + \mathbb(D_{3}|D_{2}) \mathbb{P}(D_{2}|D_{1}^c)\mathbb(D_{1}^c) \\
		& + \mathbb{P}(D_{3}|D_{2}^c)\mathbb{P}(D_{2}^c|D_{1}) \mathbb{P}(D_{1}) + \mathbb{P}(D_{3}^c|D_{2}) \mathbb{P}(D_{2}|D_{1}) \mathbb{P}(D_{1}) \\
		= & \frac{1}{2}\cdot\frac{1}{2}\cdot\frac{1}{2} + \frac{1}{2}\cdot\frac{1}{4}\cdot\frac{2}{3} \\
		& + \frac{1}{4}\cdot\frac{1}{2}\cdot\frac{1}{3}  + \frac{1}{2}\cdot\frac{1}{2}\cdot\frac{1}{3} \\
		= & \frac{7}{24}
	.\end{align*}
	Luego, $\mathbb{P}(6_{\text{elementos}} = 1 - \mathbb{P}^5($
\end{proof}





% end lessons
\end{document}
