\documentclass[a4paper]{report}
\usepackage[utf8]{inputenc}
\usepackage[T1]{fontenc}
\usepackage{textcomp}

\usepackage{url}

% \usepackage{hyperref}
% \hypersetup{
%     colorlinks,
%     linkcolor={black},
%     citecolor={black},
%     urlcolor={blue!80!black}
% }

\usepackage{graphicx}
\usepackage{float}
\usepackage[usenames,dvipsnames]{xcolor}

% \usepackage{cmbright}

\usepackage{amsmath, amsfonts, mathtools, amsthm, amssymb}
\usepackage{mathrsfs}
\usepackage{cancel}

\newcommand\N{\ensuremath{\mathbb{N}}}
\newcommand\R{\ensuremath{\mathbb{R}}}
\newcommand\Z{\ensuremath{\mathbb{Z}}}
\renewcommand\O{\ensuremath{\emptyset}}
\newcommand\Q{\ensuremath{\mathbb{Q}}}
\newcommand\C{\ensuremath{\mathbb{C}}}
\let\implies\Rightarrow
\let\impliedby\Leftarrow
\let\iff\Leftrightarrow
\let\epsilon\varepsilon

% demostraciones bidireccionales

\newcommand{\Onlyifstep}{%
	\begingroup
	\fboxsep=1pt
	\raisebox{1.2ex}{\fbox{\raisebox{-1.2ex}{$\Rightarrow$\hspace{-0.05em}}}}%
	\endgroup
	\hspace{0.5em}%
	}
\newcommand{\Ifstep}{%
	\begingroup
	\fboxsep=1pt
	\raisebox{1.2ex}{\fbox{\raisebox{-1.2ex}{\hspace{-0.05ex}$\Leftarrow$}}}%
	\endgroup
	\hspace{0.5em}%
	}

% horizontal rule
\newcommand\hr{
    \noindent\rule[0.5ex]{\linewidth}{0.5pt}
}

\usepackage{tikz}
\usepackage{tikz-cd}

% theorems
\usepackage{thmtools}
\usepackage[framemethod=TikZ]{mdframed}
\mdfsetup{skipabove=1em,skipbelow=0em, innertopmargin=5pt, innerbottommargin=6pt}

\theoremstyle{definition}

\makeatletter

\declaretheoremstyle[headfont=\bfseries\sffamily, bodyfont=\normalfont, mdframed={ nobreak } ]{thmgreenbox}
\declaretheoremstyle[headfont=\bfseries\sffamily, bodyfont=\normalfont, mdframed={ nobreak } ]{thmredbox}
\declaretheoremstyle[headfont=\bfseries\sffamily, bodyfont=\normalfont]{thmbluebox}
\declaretheoremstyle[headfont=\bfseries\sffamily, bodyfont=\normalfont]{thmblueline}
\declaretheoremstyle[headfont=\bfseries\sffamily, bodyfont=\normalfont, numbered=no, mdframed={ rightline=false, topline=false, bottomline=false, }, qed=\qedsymbol ]{thmproofbox}
\declaretheoremstyle[headfont=\bfseries\sffamily, bodyfont=\normalfont, numbered=no, mdframed={ nobreak, rightline=false, topline=false, bottomline=false } ]{thmexplanationbox}


\declaretheorem[numberwithin=chapter, style=thmgreenbox, name=Definition]{definition}
\declaretheorem[sibling=definition, style=thmredbox, name=Corollary]{corollary}
\declaretheorem[sibling=definition, style=thmredbox, name=Proposition]{prop}
\declaretheorem[sibling=definition, style=thmredbox, name=Theorem]{theorem}
\declaretheorem[sibling=definition, style=thmredbox, name=Lemma]{lemma}



\declaretheorem[numbered=no, style=thmexplanationbox, name=Proof]{explanation}
\declaretheorem[numbered=no, style=thmproofbox, name=Proof]{replacementproof}
\declaretheorem[style=thmbluebox,  numbered=no, name=Exercise]{ex}
\declaretheorem[style=thmbluebox,  numbered=no, name=Example]{eg}
\declaretheorem[style=thmblueline, numbered=no, name=Remark]{remark}
\declaretheorem[style=thmblueline, numbered=no, name=Note]{note}

\renewenvironment{proof}[1][\proofname]{\begin{replacementproof}}{\end{replacementproof}}

\AtEndEnvironment{eg}{\null\hfill$\diamond$}%

\newtheorem*{uovt}{UOVT}
\newtheorem*{notation}{Notation}
\newtheorem*{previouslyseen}{As previously seen}
\newtheorem*{problem}{Problem}
\newtheorem*{observe}{Observe}
\newtheorem*{property}{Property}
\newtheorem*{intuition}{Intuition}


\usepackage{etoolbox}
\AtEndEnvironment{vb}{\null\hfill$\diamond$}%
\AtEndEnvironment{intermezzo}{\null\hfill$\diamond$}%




% http://tex.stackexchange.com/questions/22119/how-can-i-change-the-spacing-before-theorems-with-amsthm
% \def\thm@space@setup{%
%   \thm@preskip=\parskip \thm@postskip=0pt
% }

\usepackage{xifthen}

\def\testdateparts#1{\dateparts#1\relax}
\def\dateparts#1 #2 #3 #4 #5\relax{
    \marginpar{\small\textsf{\mbox{#1 #2 #3 #5}}}
}

\def\@lesson{}%
\newcommand{\lesson}[3]{
    \ifthenelse{\isempty{#3}}{%
        \def\@lesson{Lecture #1}%
    }{%
        \def\@lesson{Lecture #1: #3}%
    }%
    \subsection*{\@lesson}
    \testdateparts{#2}
}

% fancy headers
\usepackage{fancyhdr}
\pagestyle{fancy}

% \fancyhead[LE,RO]{Gilles Castel}
\fancyhead[RO,LE]{\@lesson}
\fancyhead[RE,LO]{}
\fancyfoot[LE,RO]{\thepage}
\fancyfoot[C]{\leftmark}
\renewcommand{\headrulewidth}{0pt}

\makeatother

% figure support (https://castel.dev/post/lecture-notes-2)
\usepackage{import}
\usepackage{xifthen}
\pdfminorversion=7
\usepackage{pdfpages}
\usepackage{transparent}
\newcommand{\incfig}[1]{%
    \def\svgwidth{\columnwidth}
    \import{./figures/}{#1.pdf_tex}
}

% %http://tex.stackexchange.com/questions/76273/multiple-pdfs-with-page-group-included-in-a-single-page-warning
\pdfsuppresswarningpagegroup=1

\author{Gilles Castel}

\DeclareMathOperator{\supp}{supp}
\DeclareMathOperator{\spann}{span}
\DeclareMathOperator{\Id}{Id}
\DeclareMathOperator{\Ker}{Ker}
\DeclareMathOperator{\im}{Im}
\DeclareMathOperator{\GL}{GL}
\DeclareMathOperator{\SL}{SL}
\DeclareMathOperator{\Mat}{Mat}

\usepackage{stackengine,scalerel}

\newcommand{\bigcupcolon}{\mathop{\ThisStyle{%
  \ensurestackMath{\stackinset{c}{-0.4pt}{c}{+.25\LMex}{\tiny{D}}{\SavedStyle\bigcup}}}}}

\title{Teoría de Integración}
\author{}
\date{Basado en las clases impartidas por Santiago Saglietti en el segundo semeste del 2025}
\begin{document}
    \maketitle
    \tableofcontents
    % start lessons

    \chapter{Integral de Riemann}
    \setcounter{section}{0}
	\section{Clase 1 (04/08)}

	\begin{definition}[partición + intervalos]
		Una partición de un intervalo $[a,b]\subseteq\R$ es un subconjunto finito $\Pi\subseteq[a,b]$ tal que $a,b\in\Pi$. Denotaremos a las particiones como $\Pi=\{x_0,\dots,x_n\}$, donde $a=x_0<x_1<\cdots<x_n=b$. Los intervalos $I_i=[x_{i-1},x_i]$, $i=1,\dots,n$ serán llamados intervalos de la partición.
	\end{definition}

	\begin{remark}
		A veces, identificaremos la partición $\Pi$ con $(I_i)_{i=1,\dots,n}$. En tal caso, abusando de la notación, escribiremos $I_i\in\Pi$ cuando queramos hablar de los intervalos de $\Pi$.
	\end{remark}

	\begin{definition}[norma de particiones]
		La norma de una partición $\Pi$ como $\|\Pi\|\coloneqq \max_{i=1,\dots,n}(x_i-x_{i-1})=\max_{I_i\in\Pi}|I_i|$.
	\end{definition}

	\begin{definition}[partición marcada]
		Una partición marcada de $[a,b]$ es un par $\Pi^*\coloneqq(\Pi,\varepsilon)$ donde:
		\begin{itemize}
			\item $\Pi = \{x_0,\dots,x_n\}$ es una partición de $[a,b]$;
			\item $\varepsilon = \{x_1^*,\dots,x_n^*\}$ es una colección de puntos tal que $x_i^*\in I_i$ para cada $i=1,\dots,n$.
		\end{itemize}
	\end{definition}

	\begin{remark}
		Dada una partición marcada $\Pi^*=(\Pi,\varepsilon)$, definimos $\|\Pi^*\|\coloneqq \|\Pi\|$.
	\end{remark}

	\begin{definition}[Suma de Riemann]
		Sean $f:[a,b]\to\R$ acotada y $\Pi^*=(\Pi,\varepsilon)$ una partición marcada. Definimos la suma de Riemann de $f$ asociada a $\Pi^*$ como:
		\[
		S_R(f;\Pi^*)\coloneqq \sum_{n=1}^{n} f(x_i^*)(x_i-x_{i-1})= \sum_{I_i\in\Pi}^{} f(x_i^*)|I_i|.
		\]
	\end{definition}

	\section{Clase 2 (06/08)}

	\begin{definition}[Riemann integrable]
		Dada $f:[a,b]\to\R$ acotada, decimos que es Riemann integrable si existe el límite $\lim_{\|\Pi^*\|\to 0} S_R(f;\Pi^*)$. \\
		Equivalentemente, $\exists L\in\R$, tal que dado cualquier $\varepsilon>0$, existe $\delta=\delta(\varepsilon)>0$ tal que $\|\Pi^*\|<\delta\Rightarrow|S_R(f;\Pi^*)-L|<\varepsilon$.
	\end{definition}

	\begin{remark}
		Cuando el límite existe, lo llamamos la integral de Riemann de $f$ en $[a,b]$ y lo notamos $\int_{a}^{b} f(x) dx$.
	\end{remark}

	\begin{definition}[Sumas superior e inferior de Darboux]
		Dadas $f:[a,b]\to\R$ acotada y $\Pi=(I_i)_{i=1,\dots,n}$ una partición de $[a,b]$, definimos 
	
		\begin{align*}
			m_{I_i}\coloneqq\inf_{x\in I_i} f(x) &,\quad M_{I_i}\coloneqq\sup_{x\in I_i} f(x) \quad \text{y} \\
			\underline{S}(f;\Pi)\coloneqq\sum_{I_i\in\Pi}^{} m_{I_i}|I_i|&,\quad \overline{S}(f;\Pi)\coloneqq\sum_{I_i\in\Pi}^{} M_{I_i}|I_i|
		.\end{align*}

		Llamamos a $\underline{S}(f;\Pi)$ y $\overline{S}(f;\Pi)$ las sumas inferior y superior de Darboux de $f$ con respecto a $\Pi$, respectivamente.
	\end{definition}

	\begin{note}
		Como $m_{I_i}\leq f(x)\leq M_{I_i}$, $\forall x\in I_i$ para toda partición marcada $\Pi^*=(\Pi;\varepsilon)$, tenemos $\underline{S}(f;\Pi)\leq S_R(f;\Pi^*)\leq \overline{S}(f;\Pi)$.
	\end{note}

	\begin{definition}[refinamiento]
		Diremos que una partición $\Pi'$ de $[a,b]$ es un refinamiento de otra partición de $[a,b]$, $\Pi$, si $\Pi\subseteq\Pi'$. \\
		Equivalentemente, si para todo $J_i\in\Pi'$ existe $I_i\in\Pi$ tal que $J_i\subseteq I_i$.
	\end{definition}

	\begin{prop}
		Sea $f:[a,b]\to\R$ acotada. Entonces,
		\begin{itemize}
			\item Si $\Pi\subseteq\Pi'$ son particiones de $[a,b]$,

			\[
			\underline{S}(f;\Pi)\leq\underline{S}(f;\Pi'),\quad\overline{S}(f;\Pi)\geq\overline{S}(f;\Pi').
			\]

			\item Si $\Pi_1,\Pi_2$ son particiones de $[a,b]$ cualesquiera,

			\[
			\underline{S}(f;\Pi_1)\leq\overline{S}(f;\Pi_2)
			\]
		\end{itemize}
	\end{prop}

	\begin{definition}
		Sea $f:[a,b]\to\R$ acotada. Definimos:
		\begin{itemize}
			\item La integral superior (de Darboux) de $f$ como $\overline{\int_{a}^{b}} f(x) dx \coloneq \displaystyle{\inf_{\Pi}} \overline{S}(f;\Pi)$.
			
			\item La integral inferior (de Darboux) de $f$ como $\underline{\int_{a}^{b}} f(x) dx \coloneq \displaystyle{\sup_{\Pi}} \underline{S}(f;\Pi)$.
		\end{itemize}
	\end{definition}

	\begin{theorem}
		Sea $f:[a,b]\to\R$ acotada. Entonces,
		\[
		\underline{\int_{a}^{b}} f(x) dx = \displaystyle{\lim_{\|\Pi\|\to 0}} \underline{S}(f;\Pi)\quad \text{y} \quad \overline{\int_{a}^{b}} f(x) dx = \displaystyle{\lim_{\|\Pi\|\to 0}}\overline{S}(f;\Pi).
		\]
	\end{theorem}

	\begin{remark}
		Equivalentemente, para cualquier sucesión $(\Pi_n)_{n\in\N}$ de partición de $[a,b]$ tal que $\|\Pi_n\|\xrightarrow{n\to\infty} 0$, se tiene que 

		\[
		\underline{\int_{a}^{b}} f(x) dx = \lim_{n\to\infty} \underline{S}(f;\Pi_n)\quad\text{y}\quad \overline{\int_{a}^{b}} f(x) dx = \lim_{n\to \infty} \overline{S}(f;\Pi_n).
		\]
	\end{remark}

	\begin{theorem}
		Dada $f:[a,b]\to\R$ acotada, son equivalentes:

		\begin{enumerate}
			\item $\underline{\int_{a}^{b}} f(x) dx = \overline{\int_{a}^{b}} f(x) dx$ (i.e., $f$ es Darboux integrable).

			\item $f$ es Riemann integrable.

			\item $\lim_{\|\Pi\|\to 0} \overline{S}(f;\Pi)-\underline{S}(f;\Pi)=0$.

			\item $\forall (\Pi_n)_{n\in\N}$ sucesión de particiones de $[a,b]$ tal que $\|\Pi_n\|\to 0$, 

			\[
			\lim_{n \to \infty} \overline{S}(f;\Pi_n)-\underline{S}(f;\Pi_n)=0.
			\]

			\item $\exists (\Pi_n)_{n\in\N}$ sucesión de particiones de $[a,b]$ tal que 

			\[
			\lim_{n \to \infty}\overline{S}(f;\Pi_n)-\underline{S}(f;\Pi_n)=0.
			\]
		\end{enumerate}
	\end{theorem}

	\section{Clase 3 (07/08)}

	\begin{note}
		Las integrales en el sentido de Darboux y el de Riemann coinciden.
	\end{note}

	\begin{prop}
		Si $f:[a,b]\to\R$ es monótona, entonces es Riemann integrable.
	\end{prop}

	\begin{remark}
		Una función monótona tiene discontinuidades numerables.
	\end{remark}

	\begin{prop}
		Si $f:[a,b]\to\R$ es continua, entonces es Riemann integrable.
	\end{prop}

	En particular, existen funciones Riemann integrables con numerables discontinuiodades. De hecho, hay ejemplos con $c$ (cardinal del continuo) discontinuidades. No obstante, si $f$ es integral de Riemann, su conjunto de discontinuidades tiene que ser "pequeño".

	\begin{theorem}
		Sea $f:[a,b]\to\R$ acotada. Entonces, $f$ es integral de Riemann si y sólo si su conjunto de discontinuidades tiene medida nula.
	\end{theorem}

	\begin{definition}[intervalo]
		Decimos que un conjunto $I\subseteq\overline{\R}\coloneq\R\cup \{-\infty,\infty\}$ es un intervalo si satisface

		\[
		x,y\in I \Rightarrow z\in I \text{ para todo } \min x,y\leq z\leq\max x,y.
		\]
	\end{definition}

	\begin{eg}
		(y propiedades)
		\begin{itemize}
			\item Dados $a\leq b$ ($a,b\in\R$), los conjuntos $(a,b),(a,b],[a,b],[a,b)$ son intervalos;

			\item El conjunto vacío es un intervalo ($\varnothing = (a,a)$);

			\item Los puntos son intervalos. $I = [\lambda,\lambda]$;

			\item La intersección son intervalos de intervalos.
		\end{itemize}
	\end{eg}

	\begin{definition}[intervalo generalizado]
		Decimos que un conjunto $I\subseteq\R^d$ es un intervalo si puede escribirse como

		\[
		I=\prod_{k=1}^{d} I_k
		\]

		donde cada $I_r$ es un intervalo en $\R$. La medida de un intervalo $I\subseteq\R^d$ se define como
		
		\[
		|I|\coloneq\prod_{k=1}^{d} |I_k|.
		\]
	\end{definition}

	\begin{note}
		Los intervalos en $\R^d$ heredan las mismas pripiedades en $\R$:

		\begin{itemize}
			\item Intersección de intervalos en $\R^d$ es intervalo.

			\item Si $I\subseteq J\subseteq\R^d$ son intervalos, entonces $|I|\leq|J|$.
		\end{itemize}
	\end{note}

	\begin{definition}[medida nula]
		Un conjunto $E\subseteq\R^d$ se dice de medida nula si, dado $\varepsilon>0$, existe una sucesión $(I_n)_{n\in\N}$ de intervalos de $\R^d$ tal que

		\[
		E\subseteq\bigcup_{n\in\N} I_n\quad\text{ y }\quad \sum_{n\in\N}^{}|I_n|<\varepsilon.
		\]
	\end{definition}

	\begin{eg}
		(y propiedades)
		\begin{enumerate}
			\item Todo conjunto unitario $\{x\}, (x\in\R^d)$ tiene medida nula;

			\item Toda unión numerable de conjuntos de medida nula tiene medida nula;

			\item Cualquier conjunto numerable tiene medida nula;
			
			\item Cualquier subconjunto de un conjunto de medida nula tiene medida nula;

			\item Existen conjuntos no numerables de medida nula:

			\begin{itemize}
				\item En $\R^d$ con $d\geq 2$, los ejes $\{x:x_1=0\}, i=1,\dots,d$ tiene medida nula.

				\item En $\R$, el conjunto de cantor tiene medida nula.
			\end{itemize}

			\item $E\subseteq\R^d$ es de medida nula, entonces $\alpha\dot E$ tiene medida nula $\forall\alpha\in\R$.
			
			\item $E\subseteq\R^d$ es de medida nula, entonces $E + v$ tiene medida nula $\forall v\in\R^d$.

			\item Si $E$ contiene un intervalo no unitario, entonces no tiene medida nula. Notar que:

			\begin{itemize}
				\item La vuelta no es válida: $\R\textbackslash\Q$ no contiene untervalos no unitarios pero no puede tener medida nula.

				\item De esto se deduce que si $E\subseteq\R^d$ tiene medida nula. Entonces $E^c$ es denso (no vale la vuelta: $E^c=\Q$).
			\end{itemize}

			\item $E\subseteq\R^d$ tiene medida nula si y sólo si

			\[
			|E|_e\coloneq\inf \{ \sum_{n\in\N} |I_n| : E \subseteq \bigcup_{n\in\N} I_n \} = 0, \quad I_n \text{ intervalo } \forall n \in \N.
			\]
		\end{enumerate}
	\end{eg}

	\section{Clase 4 (08/08)}

	\begin{theorem}
		Sea $f:[a,b]\to\R$ acotada. Entonces
		\begin{align*}
			f \text{ Riemann integrable}\Longleftrightarrow & D_f=\{x\in[a,b]:f\text{ discontinua en } x\} \\ & \text{ tiene medida nula.}
		\end{align*}
	\end{theorem}

	\subsection{Limitaciones de la integral de Riemann}

	\begin{enumerate}
		\item Sólo está definida para $f$ acotada y sobre intervalos $[a,b]$ acotados. La teoría de integrales impropias resuelve esto.

		\item Propiedades del espacio $\mathcal{R}([a,b])=\{f:[a,b]\to\R:f\text{ Riemann integrable}\}$: Nos gustaría poder definir una noción de convergencia en $\mathcal{R}([a,b])$ tal que
		\[
		f_n\to f \text { en } \mathcal{R}([a,b]) \Rightarrow \int_{a}^{b} f_n\to \int_{a}^{b} f \quad \left( \lim \int_{a}^{b} f_n = \int_{a}^{b} \lim f_n \right).
		\]
	\end{enumerate}

	\begin{remark}
		La convergencia puntal NO cumple esto (punto 2).
	\end{remark}

	\begin{eg}[1]~
		\begin{itemize}
			\item $f_n \coloneq n \chi_{(0,\frac{1}{n}]}$ es Riemann integrable en $[0,1],\ \forall n \in \N$;

			\item $f_n \to f \cong 0$ puntualmente en $[0,1]$;

			\item $\int_{0}^{1} f_n = 1 \not\to 0 = \int_{0}^{1} f$.
		\end{itemize}
	\end{eg}

	\begin{eg}[2]~
		\begin{itemize}
			\item Sea $(Q_n)_{n \in \N}$ una enumeración de $\Q \cap [0,1]$;

			\item $f_n \coloneq \chi_{\{ Q_1,\dots,Q_n \}}$ es Riemann integrable en $[0,1],\ \forall n \in \N$;

			\item $f_n \to f \coloneq \chi_{\Q \cap [0,1]}$ puntualmente en $[0,1]$;

			\item $f$ no es Riemann integrable. $\underline{\int_{0}^{1}} f = 0 \neq 1 = \overline{\int_{0}^{1}} f$.
		\end{itemize}
	\end{eg}

	\begin{remark}
		La convergencia uniforme SÍ cumple esto, pero es demasiado fuerte.
	\end{remark}

	\begin{ex}[Guía 1]
		Sean $(f_n)_{n \in \N} \subset \mathcal{R}([a,b])$ tales que $f_n \to f$ uniformemente en $[a,b]$. Entonces, $f \in \mathcal{R}([a,b])$ y $\lim_{n \to \infty} \int_{a}^{b} f_n = \int_{a}^{b} f$.
	\end{ex}

	\begin{eg}[3]~
		\begin{itemize}
			\item $f_n (x) \coloneq x^n$ en $[0,1],\ f_n \in \mathcal{R}([a,b]),\ \forall n \in \N,\ f_n \to \chi = f$ puntualmente;

			\item $f \in \mathcal{R}([a,b])$ y $\int_{0}^{1} f_n (x) dx = \frac{1}{n+1} \to 0 = \int_{0}^{1}$;

			\item $f_n$ no converge uniformemente a $f$.
		\end{itemize}
	\end{eg}

	Resulta que la noción de convergencia "óptima" (la más "débil" que cumple lo que queremos) es la de convergencia en $L'$:

	\[
	f_n \xrightarrow{L'} f \text{ si } \lim_{n \to \infty} \int_{a}^{b} | f_n - f | = 0.
	\]

	Esta noción de convergencia viene dada por una "norma":

	\begin{itemize}
		\item $\| f \|_{L'} \coloneq \int_{a}^{b} |f|$ (recordar que $f \in \mathcal{R}([a,b]) \implies |f| \in \mathcal{R}([a,b])$);

		\item $d_{L'} (f,g) \coloneq \| f - g \|_{L'} = \int_{a}^{b} |f-g|$.
	\end{itemize}

	\begin{remark}
		$\| \cdot \|_{L'}$ no es una norma porque $\| f \|_{L'} = 0 \nRightarrow f = 0$. Decimos que es una \textit{pseudo-norma} y $d$ una \textit{pseudo-métrica}.
	\end{remark}

	Para arreglar esto, dadas $f,g : [a,b] \to \R$, decimos que son \textit{equivalentes} y lo notamos $f \sim g$ si $\{ x \in [a,b] \ : \ f(x) \neq g(x) \}$ tiene medida nula. Resulta que $\sim$ es una relación de equivalencia y, además,

	\[
	f,g \in \mathcal{R} ([a,b]),\ f \sim g \implies \int_{a}^{b} f = \int_{a}^{b} g.
	\]

	Sea $\overline{\mathcal{R}}([a,b])$ el conjunto de clases de equivalencia de $\mathcal{R}([a,b])$, y denotamos por $\overline{f}$ a la clase de equivalencia de $f \in \mathcal{R}([a,b])$. Con esto, $\| \overline{f} \|_{L'} \coloneq \int_{a}^{b} |f| dx$ define una norma en $\overline{\mathcal{R}} ([a,b])$ que se llama la \textbf{norma $L'$}.

	\begin{remark}
		Hay un problema: $(\overline{\mathcal{R}} ([a,b]), \| \cdot \|_{L'})$ NO ES COMPLETO!
	\end{remark}

	\begin{enumerate}
		\item[3.] \textbf{TFC:} Si $f \in \mathcal{R} ([a,b])$ es continua en $x_0 \in [a,b]$, entonces $F(x) \coloneq \int_{a}^{x} f(t) dt$ es derivable en $x_0$ y $F'(x_0) = f(x_0)$. En particular, $F$ es derivable en $x$ y $F'(x)=f(x)$ para todo $x$ salvo un conjunto de medida nula.
	\end{enumerate}


	\subsection{Clase 5 (18/08)}

	\noindent \textbf{Teorema Fundamental del Cálculo: } Si $f \in \mathcal{R}([a,b])$ es continua en $x_0 \in [a,b]$, entonces $F: [a,b] \to \R$ dada por $F(x) \coloneq \int_{a}^{x} f(t) dt$ es derivable en $x=x_0$ y vale $F'(x_0) = f(x_0)$. En particular, $F'(x) = f(x)$ salvo quizás por un conjunto de $x \in [a,b]$ de medida nula. O sea, podemos integrar y luego derivar y esto es "casi" como no hacer nada. Pero, tenemos problemas:

	\begin{enumerate}
		\item \textbf{Este "casi" no puede removerse}
		\begin{theorem}[Hankel, 1871]
			Dado $[a,b] \subseteq \R$, existe $f\in\mathcal{R}([a,b])$ tal que $F(x) \coloneq \int_{a}^{x} f(t) dt$ no es derivable para ningún $x$ en un subconjunto denso en $[a,b]$ (y, en particular, infinito).
		\end{theorem}

		\item \textbf{A veces no podemos componer en el orden inverso}
		\begin{theorem}[Volterra, 1881]
			Dado $[a,b]\subseteq\R$, existe $f: [a,b]\to\R$ derivable en $[a,b]$, tal que $f'$ es acotada en $[a,b]$ pero $f' \not\in \mathcal{R} ([a,b])$.
		\end{theorem}
	\end{enumerate}

	\noindent \textbf{Extendiendo la integral de Riemann}

	Sean $f:[a,b]\to\R$ acotada y $\Pi = \{ x_0,\dots,x_n \}$ una partición de $[a,b]$. Definimos:
	\begin{align*}
		\Phi_{f,\Pi}(x) & \coloneq m_{I_1} \chi_{[x_0,x_i]} (x) + \sum_{i=2}^{n} m_{I_i} \chi_{(x_{i-1},x_i]}(x), \quad m_{I_i} = \inf_{t \in I_i} f(t) \\
		& = m_{I_1} \chi_{\{x_0\}}(x) + \sum_{i=1}^{n} m_{I_i} \chi_{(x_{i-1},x_i]}(x) \\
		\psi_{f,\Pi} & \coloneq M_{I_1} \chi_{\{x_0\}}(x) + \sum_{i=1}^{n} M_{I_i} \chi_{(x_{i-1},x_i]}(x), \quad M_{I_i} = \sup_{t\in I_i} f(t)
	.\end{align*}

	Observemos que $\Phi_{f,\Pi}(x) \leq f(x) \leq \psi_{f,\Pi}(x) \quad \forall x \in [a,b]$. Además, 
	\[
	\int_{a}^{b} \Phi_{f,\Pi}(x) dx = \underline{S}(f,\Pi) \\
	\int_{a}^{b} \psi_{f,\Pi}(x) dx = \overline{S}(f,\Pi).
	\]
	\noindent En particular, si $f$ es Riemann integrable,
	\begin{align*}
		\int_{a}^{b} f(x) dx & = \overline{\int_{a}^{b}} f(x) dx = \inf \left\{ \int_{a}^{b} \psi_{f,\Pi} \ : \ \Pi \text{ partición} \right\} \\
		& = \underline{ \int_{a}^{b} } f(x) dx = \sup \left\{ \int_{a}^{b} \Phi_{f,\Pi} \ : \ \Pi \text{ partición} \right\}
	.\end{align*}

	\begin{definition}[función escalonada]
		Una función $\Phi : [a,b] \to \R$ se dice escalonada si existen $\Pi = \{ x_0,\dots,x_n \}$ partición de $[a,b]$ y $c_1,\dots,c_n \in \R$ tales que
		\[
		\Phi |_{(x_{i-1},x_i)} \equiv c_i \quad \forall i = 1,\dots,n
		\]
	\end{definition}

	Notemos que podemos escribir a cualquier función $\Phi$ escalonada como
	\begin{align*}
		\Phi (x) & \coloneq \sum_{i=1}^{n} c_i \cdot \chi_{(x_{i-1},x_i}(x) + \sum_{i=0}^{n} \Phi(x_i) \cdot \chi_{\{x_i\}}(x) \\
		& = \sum_{i=1}^{k} c_j \cdot \chi_{A_j}(x).
	.\end{align*}
	\noindent donde los $A_j$ son intervalos disjuntos tales que $\displaystyle\bigcup_{j=1}^{k} A_j = [a,b]$ (se pone una "D" dentro de la "U" de unión para denotar que estamos haciendo una unión disjunta).

	Si tomamos $\Phi$ de la forma $\Phi = \sum_{j=1}^{k} c_j \cdot \chi_{A_j}$ con $(A_j)_{j=1,\dots,k}$ disjuntos, $\displaystyle\bigcup_{j=1}^{k} A_j = [a,b]$ (la U tiene la D en medio) pero $A_j$  no son necesariamente intervalos, diremos que $\Phi$ es una función escalonada generalizada. Como para funciones escalonadas "normales", tenemos
	\[
	\int_{a}^{b} \Phi (x) dx = \sum_{j=1}^{k} c_j \cdot |A_j| \left( = \sum_{i=1}^{n} c_i \cdot |I_i| \right)
	\]

	$\bigcupcolon$

	\noindent \textbf{La función longitud}
	Sea $\mathcal{I}$ la colección de los intervalos en $\R$. Definimos la función longitud $\lambda : \mathcal{I} \to [0,\infty]$ como $\lambda (I) \coloneq |I|$.
	
	\noindent \textbf{Propiedades:}
	\begin{enumerate}
		\item $\lambda (\varnothing) = 0$;

		\item $I_1,I_2 \in \mathcal{I},\ I_1\subseteq I_2 \implies \lambda (I_1) \leq \lambda (I_2) (\text{Monotonía de } \lambda)$;

		\item (Aditividad finita de $\lambda$) Si $I \in \mathcal{I}$ es tal que $I = \displaystyle\bigcup_{i=1}^{n} J_i$ (la union tiene la D en medio) con $J_i \in \mathcal{I},\ \forall i = 1,\dots,n,\ J_i \cap J_j = \varnothing$ sin $i\neq j$, entonces
		\[
		\lambda (I) = \sum_{i=1}^{n} \lambda (J_i)
		\]

		\item ($r$-aditividad de $\lambda$) Si $I \in \mathcal{I}$ es tal que $I = \displaystyle\bigcup_{i=1}^{\infty} I_i $, con $(I_i)_{i} \in \N \subseteq \mathcal{I}$ disjuntos, entonces
		\[
		\lambda(I) = \sum_{i=1}^{\infty} \lambda (I_i)
		\]

		\item ($r$-subaditividad de $\lambda$) Si $I \in \mathcal{I}$ verifica $I \subseteq \displaystyle\bigcup_{i=1}^{\infty} I_i, \ (I_1)_{i \in \N})$ intervalos (no necesariamente disjuntos), entonces $\lambda (I) \leq \sum_{i=1}^{\infty} \lambda (I_i)$

		\item $\lambda (I + x) = \lambda (I), \ \forall x \in \R, \ I+x \coloneq \{a + x \ : \ a \in I \}  $
	\end{enumerate}



% end lessons
\end{document}
