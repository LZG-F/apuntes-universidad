\documentclass[a4paper]{report}
\usepackage[utf8]{inputenc}
\usepackage[T1]{fontenc}
\usepackage{textcomp}

\usepackage{url}

% \usepackage{hyperref}
% \hypersetup{
%     colorlinks,
%     linkcolor={black},
%     citecolor={black},
%     urlcolor={blue!80!black}
% }

\usepackage{graphicx}
\usepackage{float}
\usepackage[usenames,dvipsnames]{xcolor}

% \usepackage{cmbright}

\usepackage{amsmath, amsfonts, mathtools, amsthm, amssymb}
\usepackage{mathrsfs}
\usepackage{cancel}

\newcommand\N{\ensuremath{\mathbb{N}}}
\newcommand\R{\ensuremath{\mathbb{R}}}
\newcommand\Z{\ensuremath{\mathbb{Z}}}
\renewcommand\O{\ensuremath{\emptyset}}
\newcommand\Q{\ensuremath{\mathbb{Q}}}
\newcommand\C{\ensuremath{\mathbb{C}}}
\let\implies\Rightarrow
\let\impliedby\Leftarrow
\let\iff\Leftrightarrow
\let\epsilon\varepsilon

%MIS AGREGADOS

\newcommand{\bigcupd}{\mathop{\ensurestackMath{\stackinset{l}{}{c}{+.25ex}{\hspace{3pt}{\text{\tiny{D}}}}{\displaystyle{\bigcup}}}}}

\newcommand{\textbigcupd}{\mathop{\ensurestackMath{\stackinset{l}{}{c}{+.15ex}{\hspace{1.7pt}{\text{\tiny{D}}}}{\bigcup}}}}

\newcommand{\cupd}{\mathop{\ensurestackMath{\stackinset{l}{}{c}{+.25ex}{\hspace{1.4pt}{\text{\tiny{d}}}}{\cup}}}}

\usepackage{stackengine,scalerel}

\usepackage[normalem]{ulem}

\usepackage{dsfont}

% demostraciones bidireccionales

\newcommand{\Onlyifstep}{%
	\begingroup
	\fboxsep=1pt
	\raisebox{1.2ex}{\fbox{\raisebox{-1.2ex}{$\Rightarrow$\hspace{-0.05em}}}}%
	\endgroup
	\hspace{0.5em}%
	}
\newcommand{\Ifstep}{%
	\begingroup
	\fboxsep=1pt
	\raisebox{1.2ex}{\fbox{\raisebox{-1.2ex}{\hspace{-0.05ex}$\Leftarrow$}}}%
	\endgroup
	\hspace{0.5em}%
	}

\usepackage{tikz}
\usepackage{tikz-cd}
\usetikzlibrary{calc, tikzmark}

%MIS AGREGADOS

% horizontal rule
\newcommand\hr{
    \noindent\rule[0.5ex]{\linewidth}{0.5pt}
}

%\usepackage{tikz}
%\usepackage{tikz-cd}

% theorems
\usepackage{thmtools}
\usepackage[framemethod=TikZ]{mdframed}
\mdfsetup{skipabove=1em,skipbelow=0em}
%, innertopmargin=5pt, innerbottommargin=6pt}

\theoremstyle{definition}

\makeatletter

%\declaretheoremstyle[
%	headfont=\bfseries\sffamily\color{ForestGreen!70!black}, bodyfont=\normalfont, 
%	mdframed={ 
%		linewidth=2pt,
%		rightline=false, topline=false, bottomline=false,
%		linecolor=ForestGreen, backgroundcolor=ForestGreen!5
%	}
%]{thmgreenbox}
%
%\declaretheoremstyle[
%	headfont=\bfseries\sffamily\color{NavyBlue!70!black}, bodyfont=\normalfont, 
%	mdframed={ 
%		linewidth=2pt,
%		rightline=false, topline=false, bottomline=false,
%		linecolor=NavyBlue, backgroundcolor=NavyBlue!5
%	}
%]{thmbluebox}
%
%\declaretheoremstyle[
%	headfont=\bfseries\sffamily\color{NavyBlue!70!black}, bodyfont=\normalfont, 
%	mdframed={ 
%		linewidth=2pt,
%		rightline=false, topline=false, bottomline=false,
%		linecolor=NavyBlue
%	}
%]{thmblueline}
%
%\declaretheoremstyle[
%	headfont=\bfseries\sffamily\color{RawSienna!70!black}, bodyfont=\normalfont, 
%	mdframed={ 
%		linewidth=2pt,
%		rightline=false, topline=false, bottomline=false,
%		linecolor=RawSienna, backgroundcolor=RawSienna!5
%	}
%]{thmredbox}
%
%\declaretheoremstyle[
%	headfont=\bfseries\sffamily\color{RawSienna!70!black}, bodyfont=\normalfont, numbered=no,
%	mdframed={ 
%		linewidth=2pt,
%		rightline=false, topline=false, bottomline=false,
%		linecolor=RawSienna, backgroundcolor=RawSienna!1
%	},
%	qed=\qedsymbol
%]{thmproofbox}
%
%\declaretheoremstyle[
%	headfont=\bfseries\sffamily\color{NavyBlue!70!black}, bodyfont=\normalfont, numbered=no,
%	mdframed={ 
%		linewidth=2pt,
%		rightline=false, topline=false, bottomline=false,
%		linecolor=NavyBlue, backgroundcolor=NavyBlue!1
%	}
%]{thmexplanationbox}





\declaretheoremstyle[headfont=\bfseries\sffamily, bodyfont=\normalfont, mdframed={ nobreak } ]{thmgreenbox}
\declaretheoremstyle[headfont=\bfseries\sffamily, bodyfont=\normalfont, mdframed={ nobreak } ]{thmredbox}
\declaretheoremstyle[headfont=\bfseries\sffamily, bodyfont=\normalfont]{thmbluebox}
\declaretheoremstyle[headfont=\bfseries\sffamily, bodyfont=\normalfont]{thmblueline}
\declaretheoremstyle[headfont=\bfseries\sffamily, bodyfont=\normalfont, numbered=no, mdframed={ rightline=false, topline=false, bottomline=false, }, qed=\qedsymbol ]{thmproofbox}
\declaretheoremstyle[headfont=\bfseries\sffamily, bodyfont=\normalfont, numbered=no, mdframed={ nobreak, rightline=false, topline=false, bottomline=false } ]{thmexplanationbox}


\declaretheorem[numberwithin=chapter, style=thmgreenbox, name=Definición]{definition}
\declaretheorem[sibling=definition, style=thmredbox, name=Corolario]{corollary}
\declaretheorem[sibling=definition, style=thmredbox, name=Proposición]{prop}
\declaretheorem[sibling=definition, style=thmredbox, name=Teorema]{theorem}
\declaretheorem[sibling=definition, style=thmredbox, name=Lema]{lemma}
%AGREGADO
\declaretheorem[sibling=definition, style=thmredbox, name=Criterio]{criterio}
%AGREGADO



\declaretheorem[numbered=no, style=thmexplanationbox, name=Demostración]{explanation}
\declaretheorem[numbered=no, style=thmproofbox, name=Demostración]{replacementproof}
\declaretheorem[style=thmbluebox,  numbered=no, name=Ejercicio]{ex}
\declaretheorem[style=thmbluebox,  numbered=no, name=Ejemplo]{eg}
\declaretheorem[style=thmblueline, numbered=no, name=Observación]{remark}
\declaretheorem[style=thmblueline, numbered=no, name=Nota]{note}

\renewenvironment{proof}[1][\proofname]{\begin{replacementproof}}{\end{replacementproof}}

%\AtEndEnvironment{eg}{\null\hfill$\diamond$}%

\newtheorem*{uovt}{UOVT}
\newtheorem*{notation}{Notación}
\newtheorem*{previouslyseen}{As previously seen}
\newtheorem*{problem}{Problema}
\newtheorem*{observe}{Observar}
\newtheorem*{property}{Propiedad}
\newtheorem*{intuition}{Intuición}


\usepackage{etoolbox}
\AtEndEnvironment{vb}{\null\hfill$\diamond$}%
\AtEndEnvironment{intermezzo}{\null\hfill$\diamond$}%




% http://tex.stackexchange.com/questions/22119/how-can-i-change-the-spacing-before-theorems-with-amsthm
% \def\thm@space@setup{%
%   \thm@preskip=\parskip \thm@postskip=0pt
% }

\usepackage{xifthen}

\def\testdateparts#1{\dateparts#1\relax}
\def\dateparts#1 #2 #3\relax{
    \marginpar{\small\textsf{\mbox{#1 #2 #3}}}
}

\def\@lesson{}%
\newcommand{\clase}[3]{
    \ifthenelse{\isempty{#3}}{%
        \def\@lesson{Clase #1}%
    }{%
        \def\@lesson{Clase #1: #3}%
    }%
    \subsection*{\@lesson}
    \testdateparts{#2}
}

% fancy headers
\usepackage{fancyhdr}
\pagestyle{fancy}

% \fancyhead[LE,RO]{Gilles Castel}
\fancyhead[RO,LE]{\@lesson}
\fancyhead[RE,LO]{}
\fancyfoot[LE,RO]{\thepage}
\fancyfoot[C]{\leftmark}

\makeatother

% figure support (https://castel.dev/post/lecture-notes-2)
\usepackage{import}
\usepackage{xifthen}
\pdfminorversion=7
\usepackage{pdfpages}
\usepackage{transparent}
\newcommand{\incfig}[1]{%
    \def\svgwidth{\columnwidth}
    \import{./figures/}{#1.pdf_tex}
}

% %http://tex.stackexchange.com/questions/76273/multiple-pdfs-with-page-group-included-in-a-single-page-warning
\pdfsuppresswarningpagegroup=1

\author{Gilles Castel}

\DeclareMathOperator{\supp}{supp}
\DeclareMathOperator{\spann}{span}
\DeclareMathOperator{\Id}{Id}
\DeclareMathOperator{\Ker}{Ker}
\DeclareMathOperator{\im}{Im}
\DeclareMathOperator{\GL}{GL}
\DeclareMathOperator{\SL}{SL}
\DeclareMathOperator{\Mat}{Mat}

\title{Ayudantías Ecuaciones Diferenciales Ordinarias}
\author{}
\begin{document}
    \maketitle
    \tableofcontents
    % start lessons

    \chapter{}
    \setcounter{section}{0}
	\section{Ayudantía 2 (18/08)}

	\begin{enumerate}
		\item Considere la ecuación canónica
		\[
		y^{n} = f(t,y,y',\dots,y^{(n-1)}) \tag{1}
		\]
		\begin{proof}[Proof ]
			Como $y$ es solución de $(1)$, en particular $y \in C^n(I)$. Por lo tanto la función $t \mapsto (t,y(t),y'(t),\dots,y^{(n-1)}(t)) \in C^1 (I,\R^n)$, y luego  $f(t,y(t),\dots, y^{(n-1)}(t) \in C^1(I)$. Por lo tanto $y \in C^{n+1}(I)$. \\
			\noindent \fbox{Por inducción en $k \in \N$} \textbf{Caso base:} Facil! \\
			\noindent \textbf{Paso inductivo:} Suponer que $\exists\ k \in \N$ tal que se cumple lo anterior y que $f \in C^{k+1}(\R^n)$. Por hipótesis, $y \in C^{n+k}(I)$, y por lo tanto $g(t) = (t,y,y',\dots,y^{(n-1)}) \in C^{n+k-(n-1)}(I) = C^{k+1}(I;\R^n)$. Luego , la composición $(y^{(n)}(t) =)\ f \circ g (t) \in C(I)^{k+1}$, por lo que $y \in C^{n+k+1}(I)$. Concluimos que si $f \in C^{\infty}(\R^n)$, entonces $y \in C^{\infty}(I)$.
		\end{proof}

		\item Considere el sistema
		\[
		(*) \begin{cases}
			y''(t) + y(t)^2 = 0 \quad \forall\ t \in [0,1]; \\
			y(0) = y(1) = 0.
		\end{cases}
		\]
		Demuestre que si $y$ es una solución no trivial de $(*)$, entonces $y(t) > 0 \ \forall t \in (0,1)$.
		\begin{proof}[Proof ]
			Notemos que $y''(t) = -y(t)^2 \leq 0$. Entonces, $y$ es convava; es decir, $\forall a,b\in [0,1]$, y $\forall s \in [0,1]$ se tiene la desigualdad
			\[
			y(a+(b-a)s) \geq y(a) + (y(b)-y(a))s. \tag{$C$}
			\]
			\noindent Con $a=0,\ b=1$, tenemos que $\forall \ t \in [0,1]$,
			\[
			y(t) \geq 0 + t(0 - 0) = 0.
			\]
			\noindent Ahora, como $y \not\equiv 0,\ \exists t_0 \in (0,1)$ tal que $y(t_0) > 0$. Sea $t_1 \in (0,t_0)$, y $s = \frac{t_1}{t_1} \in (0,1)$. Por $(C)$, con $a = 0,\ b=t_0$, tenemos que
			\begin{align*}
				y(t_1) = y(0 + t_0\cdot s) & \stackrel{(C)}{\geq} y(0) + s(y(t_0)-y(0)) \\
				& = 0 + s \cdot y(t_0) > 0
			.\end{align*}
			\noindent Por lo tanto, $y > 0$ en $(0,t_0)$. Por otro lado, para $t_1 \in (t_0, 1)$, definimos $s = \frac{t_1 - t_0}{1-t_0}$, y ocupando $(C)$ con $a = t_0$ y $b = 1$ obtenemos
			\begin{align*}
				y(t_1) = y(t_0 + s(1-t_0)) & \stackrel{(C)}{\geq} y(t_0) + s(y(1)-y(t_0)) \\
				& = (1-s)y(t_0) > 0
			.\end{align*}
		\end{proof}

		\item Considere
		\[
		(+) \begin{cases}
			y' = \sqrt{|y|}; \\
			y(t_0) = y_0.
		\end{cases}
		\]
		\begin{enumerate}
			\item Resuelva el sistema para $y_0 \neq 0$ y determine el invervalo maximal de la solución.
		\end{enumerate}
		\begin{proof}[Proof ]~
			\begin{enumerate}
				\item Primero asumamos que $y_0 > 0$. Dividiendo la EDO por $\sqrt{y}$ e integrando alrededor de $t_0$, obtenemos
				\begin{align*}
					(t-t_0) & = \int_{t_0}^{t} 1 ds = \int_{t_0}^{t} \frac{y'(s)}{\sqrt{y(s)}} ds \stackrel{(c.v.)}{=} \int_{y_0}^{y(t)} \frac{1}{\sqrt{u}} du \\
					& = 2 (\sqrt{y(t)} - \sqrt{y_0}) \\
					\implies y(t) & = \left( \frac{t-t_0}{2} + \sqrt{y_0} \right)^2
				.\end{align*}
				\noindent Para $y_0 < 0$
				\[
				(t-t_0) = \int_{|y_0|}^{|y(t)|} \frac{-1}{\sqrt{v}} dv = -2 (|\sqrt{y(t)|} - \sqrt{|y_0|})
				\]
				\noindent Por lo tanto, $y(t) = \left( \sqrt{|y_0|} - \frac{t-t_0}{2} \right)^2$. Sea 
				\[
				F_1(r) = \int_{y_0}^{r} \frac{dx}{\sqrt{x}} = 2 (\sqrt{r} - \sqrt{y_0}).
				\]
				\noindent Esta función es monótona (estricta) en el intervalo $(0,\infty)$ y por lo tanto biyectiva. $F_1(y(t)) = t-t_0$. Calculamos
				\begin{align*}
					T_{+} & = \lim_{r \to \infty} F_1(r) = \infty \\
					T_{-} & = \lim_{r \to 0} F_1(r) = -2 \sqrt{y_0}
				.\end{align*}
				\noindent Entonces $t-t_0 \in (-2 \sqrt{y_0},\infty),\ t \in (t_0-2 \sqrt{y_0}, \infty)$.
				\[
				F_2(r) = 2 ( \sqrt{|y_0|} - \sqrt{|r|}, \quad r \in (-\infty, 0)
				\]
				\begin{align*}
					T_{-} & = \lim_{r \to -\infty} F_2(r) = \infty \\
					T_{+} & = \lim_{r \to \infty} 
				.\end{align*}
			\end{enumerate}
		\end{proof}







	\end{enumerate}






% end lessons
\end{document}
