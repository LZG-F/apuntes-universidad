\documentclass[a4paper]{report}
\usepackage[utf8]{inputenc}
\usepackage[T1]{fontenc}
\usepackage{textcomp}

\usepackage{url}

% \usepackage{hyperref}
% \hypersetup{
%     colorlinks,
%     linkcolor={black},
%     citecolor={black},
%     urlcolor={blue!80!black}
% }

\usepackage{graphicx}
\usepackage{float}
\usepackage[usenames,dvipsnames]{xcolor}

% \usepackage{cmbright}

\usepackage{amsmath, amsfonts, mathtools, amsthm, amssymb}
\usepackage{mathrsfs}
\usepackage{cancel}

\newcommand\N{\ensuremath{\mathbb{N}}}
\newcommand\R{\ensuremath{\mathbb{R}}}
\newcommand\Z{\ensuremath{\mathbb{Z}}}
\renewcommand\O{\ensuremath{\emptyset}}
\newcommand\Q{\ensuremath{\mathbb{Q}}}
\newcommand\C{\ensuremath{\mathbb{C}}}
\let\implies\Rightarrow
\let\impliedby\Leftarrow
\let\iff\Leftrightarrow
\let\epsilon\varepsilon

%MIS AGREGADOS

\newcommand{\bigcupd}{\mathop{\ensurestackMath{\stackinset{l}{}{c}{+.25ex}{\hspace{3pt}{\text{\tiny{D}}}}{\displaystyle{\bigcup}}}}}

\newcommand{\textbigcupd}{\mathop{\ensurestackMath{\stackinset{l}{}{c}{+.15ex}{\hspace{1.7pt}{\text{\tiny{D}}}}{\bigcup}}}}

\newcommand{\cupd}{\mathop{\ensurestackMath{\stackinset{l}{}{c}{+.25ex}{\hspace{1.4pt}{\text{\tiny{d}}}}{\cup}}}}

\usepackage{stackengine,scalerel}

\usepackage[normalem]{ulem}

\usepackage{dsfont}

% demostraciones bidireccionales

\newcommand{\Onlyifstep}{%
	\begingroup
	\fboxsep=1pt
	\raisebox{1.2ex}{\fbox{\raisebox{-1.2ex}{$\Rightarrow$\hspace{-0.05em}}}}%
	\endgroup
	\hspace{0.5em}%
	}
\newcommand{\Ifstep}{%
	\begingroup
	\fboxsep=1pt
	\raisebox{1.2ex}{\fbox{\raisebox{-1.2ex}{\hspace{-0.05ex}$\Leftarrow$}}}%
	\endgroup
	\hspace{0.5em}%
	}

\usepackage{tikz}
\usepackage{tikz-cd}
\usetikzlibrary{calc, tikzmark}

%MIS AGREGADOS

% horizontal rule
\newcommand\hr{
    \noindent\rule[0.5ex]{\linewidth}{0.5pt}
}

%\usepackage{tikz}
%\usepackage{tikz-cd}

% theorems
\usepackage{thmtools}
\usepackage[framemethod=TikZ]{mdframed}
\mdfsetup{skipabove=1em,skipbelow=0em}
%, innertopmargin=5pt, innerbottommargin=6pt}

\theoremstyle{definition}

\makeatletter

%\declaretheoremstyle[
%	headfont=\bfseries\sffamily\color{ForestGreen!70!black}, bodyfont=\normalfont, 
%	mdframed={ 
%		linewidth=2pt,
%		rightline=false, topline=false, bottomline=false,
%		linecolor=ForestGreen, backgroundcolor=ForestGreen!5
%	}
%]{thmgreenbox}
%
%\declaretheoremstyle[
%	headfont=\bfseries\sffamily\color{NavyBlue!70!black}, bodyfont=\normalfont, 
%	mdframed={ 
%		linewidth=2pt,
%		rightline=false, topline=false, bottomline=false,
%		linecolor=NavyBlue, backgroundcolor=NavyBlue!5
%	}
%]{thmbluebox}
%
%\declaretheoremstyle[
%	headfont=\bfseries\sffamily\color{NavyBlue!70!black}, bodyfont=\normalfont, 
%	mdframed={ 
%		linewidth=2pt,
%		rightline=false, topline=false, bottomline=false,
%		linecolor=NavyBlue
%	}
%]{thmblueline}
%
%\declaretheoremstyle[
%	headfont=\bfseries\sffamily\color{RawSienna!70!black}, bodyfont=\normalfont, 
%	mdframed={ 
%		linewidth=2pt,
%		rightline=false, topline=false, bottomline=false,
%		linecolor=RawSienna, backgroundcolor=RawSienna!5
%	}
%]{thmredbox}
%
%\declaretheoremstyle[
%	headfont=\bfseries\sffamily\color{RawSienna!70!black}, bodyfont=\normalfont, numbered=no,
%	mdframed={ 
%		linewidth=2pt,
%		rightline=false, topline=false, bottomline=false,
%		linecolor=RawSienna, backgroundcolor=RawSienna!1
%	},
%	qed=\qedsymbol
%]{thmproofbox}
%
%\declaretheoremstyle[
%	headfont=\bfseries\sffamily\color{NavyBlue!70!black}, bodyfont=\normalfont, numbered=no,
%	mdframed={ 
%		linewidth=2pt,
%		rightline=false, topline=false, bottomline=false,
%		linecolor=NavyBlue, backgroundcolor=NavyBlue!1
%	}
%]{thmexplanationbox}





\declaretheoremstyle[headfont=\bfseries\sffamily, bodyfont=\normalfont, mdframed={ nobreak } ]{thmgreenbox}
\declaretheoremstyle[headfont=\bfseries\sffamily, bodyfont=\normalfont, mdframed={ nobreak } ]{thmredbox}
\declaretheoremstyle[headfont=\bfseries\sffamily, bodyfont=\normalfont]{thmbluebox}
\declaretheoremstyle[headfont=\bfseries\sffamily, bodyfont=\normalfont]{thmblueline}
\declaretheoremstyle[headfont=\bfseries\sffamily, bodyfont=\normalfont, numbered=no, mdframed={ rightline=false, topline=false, bottomline=false, }, qed=\qedsymbol ]{thmproofbox}
\declaretheoremstyle[headfont=\bfseries\sffamily, bodyfont=\normalfont, numbered=no, mdframed={ nobreak, rightline=false, topline=false, bottomline=false } ]{thmexplanationbox}


\declaretheorem[numberwithin=chapter, style=thmgreenbox, name=Definición]{definition}
\declaretheorem[sibling=definition, style=thmredbox, name=Corolario]{corollary}
\declaretheorem[sibling=definition, style=thmredbox, name=Proposición]{prop}
\declaretheorem[sibling=definition, style=thmredbox, name=Teorema]{theorem}
\declaretheorem[sibling=definition, style=thmredbox, name=Lema]{lemma}
%AGREGADO
\declaretheorem[sibling=definition, style=thmredbox, name=Criterio]{criterio}
%AGREGADO



\declaretheorem[numbered=no, style=thmexplanationbox, name=Demostración]{explanation}
\declaretheorem[numbered=no, style=thmproofbox, name=Demostración]{replacementproof}
\declaretheorem[style=thmbluebox,  numbered=no, name=Ejercicio]{ex}
\declaretheorem[style=thmbluebox,  numbered=no, name=Ejemplo]{eg}
\declaretheorem[style=thmblueline, numbered=no, name=Observación]{remark}
\declaretheorem[style=thmblueline, numbered=no, name=Nota]{note}

\renewenvironment{proof}[1][\proofname]{\begin{replacementproof}}{\end{replacementproof}}

%\AtEndEnvironment{eg}{\null\hfill$\diamond$}%

\newtheorem*{uovt}{UOVT}
\newtheorem*{notation}{Notación}
\newtheorem*{previouslyseen}{As previously seen}
\newtheorem*{problem}{Problema}
\newtheorem*{observe}{Observar}
\newtheorem*{property}{Propiedad}
\newtheorem*{intuition}{Intuición}


\usepackage{etoolbox}
\AtEndEnvironment{vb}{\null\hfill$\diamond$}%
\AtEndEnvironment{intermezzo}{\null\hfill$\diamond$}%




% http://tex.stackexchange.com/questions/22119/how-can-i-change-the-spacing-before-theorems-with-amsthm
% \def\thm@space@setup{%
%   \thm@preskip=\parskip \thm@postskip=0pt
% }

\usepackage{xifthen}

\def\testdateparts#1{\dateparts#1\relax}
\def\dateparts#1 #2 #3\relax{
    \marginpar{\small\textsf{\mbox{#1 #2 #3}}}
}

\def\@lesson{}%
\newcommand{\clase}[3]{
    \ifthenelse{\isempty{#3}}{%
        \def\@lesson{Clase #1}%
    }{%
        \def\@lesson{Clase #1: #3}%
    }%
    \subsection*{\@lesson}
    \testdateparts{#2}
}

% fancy headers
\usepackage{fancyhdr}
\pagestyle{fancy}

% \fancyhead[LE,RO]{Gilles Castel}
\fancyhead[RO,LE]{\@lesson}
\fancyhead[RE,LO]{}
\fancyfoot[LE,RO]{\thepage}
\fancyfoot[C]{\leftmark}

\makeatother

% figure support (https://castel.dev/post/lecture-notes-2)
\usepackage{import}
\usepackage{xifthen}
\pdfminorversion=7
\usepackage{pdfpages}
\usepackage{transparent}
\newcommand{\incfig}[1]{%
    \def\svgwidth{\columnwidth}
    \import{./figures/}{#1.pdf_tex}
}

% %http://tex.stackexchange.com/questions/76273/multiple-pdfs-with-page-group-included-in-a-single-page-warning
\pdfsuppresswarningpagegroup=1

\author{Gilles Castel}

\DeclareMathOperator{\supp}{supp}
\DeclareMathOperator{\spann}{span}
\DeclareMathOperator{\Id}{Id}
\DeclareMathOperator{\Ker}{Ker}
\DeclareMathOperator{\im}{Im}
\DeclareMathOperator{\GL}{GL}
\DeclareMathOperator{\SL}{SL}
\DeclareMathOperator{\Mat}{Mat}

\title{Topología}
\author{}
\date{Basado en las clases impartidas por Eduardo Oregón en el segundo semeste del 2025}
\begin{document}
    \maketitle
    \tableofcontents
    % start lessons

    \chapter{}
    \setcounter{section}{0}
	\section{Clase 1 (04/08): Espacios Topológicos [12]}

	\begin{definition}[sistema de vecindades]
		X conjunto no vacío. Si $x\in X$, consideramos $\mathcal{V}_x \subset 2^{X}$, tal que:

		\begin{enumerate}
			\item $\forall x \in X$, $\forall V \in \mathcal{V}_x$, $x \in \mathcal{V}_x$;

			\item $\forall x \in X$, $\forall V \in \mathcal{V}$, si $V' \supset V \Rightarrow V' \in \mathcal{V}_x$;

			\item Si $V_1,V_2 \in \mathcal{V} \Rightarrow V_1 \cap V_2 \in \mathcal{V}_x$.
		\end{enumerate}

		\noindent El sistema de vecindades es $\{ \mathcal{V}_x \}_{x \in X}$. Si $V \in \mathcal{V}_x$, $V$ es vecindad de $x$.  
	\end{definition}

	\begin{eg}
		\begin{enumerate}
			\item $(X,d)$ espacio métrico $\mathcal{V}_x \coloneq \{ V \subset X | \exists \varepsilon > 0 \text{ tal que } B_{\varepsilon} (x) \subset V \}$. Verificamos que sea sistema de vecindad.

			\begin{proof}[Proof]
				Verificamos $1),2)$ y $3)$:
				\begin{enumerate}
					\item[1)] $x \in X$, $V \in \mathcal{V}_x \Rightarrow x \in B_{\varepsilon} (x) \subset V$;

					\item[2)] $X$ $x \in X$, $V \in \mathcal{V}_x$, $V' \supset V \Rightarrow x \in B_{\varepsilon} (x) \subset V \subset V' \Rightarrow V' \in \mathcal{V}_x$;

					\item[3)] $\begin{aligned}[t]
						x \in V_1 \cap V_2, V_1,V_2 \in \mathcal{V}_x & \Rightarrow B_{\varepsilon_1} (x) \subset V_1, B_{\varepsilon_2} (x) \subset V_2  \\ & \Rightarrow B_{\min \{ \varepsilon_1, \varepsilon_2 \}} (x) \subset V_1 \cap V_2 \\ & \Rightarrow V_1 \cap V_2 \in \mathcal{V}_x.
					\end{aligned}$
				\end{enumerate}
			\end{proof}

			\item $X$ arbitrario, $\forall x \in X$, sea $\mathcal{V}_x = \{ X \}$ es sistema de vecindades (vacuidad).

			\item $X$ arbitrario $\mathcal{V}_x = \{ V \subset X \text{ } | \text{ } x \in V \text{ y } X \textbackslash V \text{ sea finito} \}$ (queda como ejercicio chequear que esto define un sistema de vecindades).  
		\end{enumerate}
	\end{eg}

	\begin{definition}[topología desde sistema de vecindades]
		Tenemos $X$, $\{ \mathcal{V}_x \}_{x \in X}$ sistema de vecindades. Definimos, $\tau = \{ U \subset X \text{ } | x \in U \Rightarrow U \in \mathcal{V}_x \}$.    
	\end{definition}

	\begin{lemma}
		$\tau$ cumple lo siguiente:

		\begin{enumerate}
			\item $\varnothing, X \in \tau$;

			\item $U_{\alpha} \in \tau, \alpha \in A \Rightarrow \bigcup_{\alpha \in A} U_{\alpha} \in \tau$;

			\item $U_1,\dots,U_n \in \tau \Rightarrow U_1 \cap \cdots \cap U_n \in \tau$.
		\end{enumerate}
	\end{lemma}

	$\tau$ es la topología inducida por $\{ \mathcal{V}_x \}$. Elementos de $\tau$ (subconjuntos de $X$) se llamarán abiertos.


	\section{Clase 2 (06/08): Topología, Base [12, 13]}

	\begin{proof}[Proof ] (último lema de la clase anterior)
		\begin{enumerate}
			\item $\varnothing \in \tau$ por vacuidad.
			\begin{align*}
				X \in \tau : x \in X & \Rightarrow \exists V \in \mathcal{V}_x \quad (1) x \in V; (2) x \in V \subset X \\ & \Rightarrow X \in \mathcal{V}_x. \quad \forall x : X \in \tau 
			\end{align*}

			\item Tomar $\{ U_{\alpha} \}_{\alpha \in A}, \text{ } U_{\alpha} \in \tau, \text{ } \mathcal{U} = \bigcup_{\alpha \in A} U_{\alpha}$. Si $x \in \mathcal{U} \Rightarrow x \in U_{\alpha} \in \mathcal{V}_x$ para algún $\alpha$. Como $U_{\alpha} \in \tau \Rightarrow U_{\alpha} \in \mathcal{V}_x$. Luego, si $x \in U_{\alpha} \subset \mathcal{U} \Rightarrow \mathcal{U} \in \mathcal{V}_x$, $\forall x \in \mathcal{U}$. Por lo tanto, $\mathcal{U} \in \tau$.

			\item Tomamos $U_1,\dots,U_n \in \tau$, $\mathcal{U} = U_1 \cap \cdots \cap U_n$ y $x \in \mathcal{U}$. Luego, $x \in U_i \quad \forall i$. Como $U_i \in \tau \Rightarrow U_i \in \mathcal{V}_x, \quad \forall i$. Por inducción (con las intersecciones), podemos afirmar que $\mathcal{U} \in \mathcal{V}_x, \text{ } \forall x \in \mathcal{U}$. Por lo tanto, $\mathcal{U} \in \tau$. 
		\end{enumerate}
	\end{proof}

	\subsection{Topología}

	\begin{definition}[topología]
		$X$ conjunto no vacío, $\tau \subset 2^X$ es una topología si cumple:

		\begin{enumerate}
			\item $\varnothing, X \in \tau$;

			\item $U_{\alpha} \in \tau, \text{ } \alpha \in A \Rightarrow \bigcup_{\alpha \in A} U_{\alpha} \in \tau$;

			\item $U_1,\dots,U_n \in \tau \Rightarrow U_1 \cap \cdots \cap U_n \in \tau$.
		\end{enumerate}
	\end{definition}

	\begin{remark}
		Se utilizará la siguiente notación:
		\begin{itemize}
			\item $(X,\tau)$ se llama espacio topológico.

			\item $U \in \tau \Rightarrow U$ se llama abierto (con respecto a la topología).
		\end{itemize}
	\end{remark}

	\begin{lemma}
		$\tau$ topología en $X \Rightarrow$ Inducida por un único sistema de vecindades.
	\end{lemma}

	\begin{proof}[Proof ]
		Para $x \in X$, definir $\mathcal{V}_x = \{ V \subset X \text{ } | \text{ } \exists U \in \tau \text{ con } x \in U \subset V \}$. Verificamos que $\{ \mathcal{V}_x \}_x$ es sistema de vecindades:

		\begin{enumerate}
			\item La definición implica $V \in \mathcal{V}_x \Rightarrow x \text{ } (\in U \subset) \text{ } \in V$;

			\item $\begin{aligned}[t]
				\text{Si } V \in \mathcal{V}_x \text{ y } V' \supset V & \Rightarrow (V \in \mathcal{V}_x) \text{ } x \in U \subset \text{ } (U \in \tau) \\
				& \Rightarrow x \in U \subset V' \Rightarrow V' \in \mathcal{V}_x;
			\end{aligned}$

			\item $\begin{aligned}[t]
				\text{Tomar } V_1,V_2 \in \mathcal{V}_x & \Rightarrow x \in U_1 \subset V_1, \quad x \in U_2 \subset V_2 \text{ con } U_1,U_2 \in \tau \\
				& \Rightarrow x \in \underbrace{U_1 \cap U_2}_{\in \text{ } \tau} \subset V_1 \cap V_2 \Rightarrow V_1 \cap V_2 \in \mathcal{V}_x;
			\end{aligned}$
		\end{enumerate}

		(falta demostrar unicidad).
	\end{proof}

	\begin{eg}[de espacios topológicos] \text{}
		\begin{enumerate}
			\item (Topología métrica): $(X,d)$ espacio métrico. Abierto es $U \in X$ tal que $\forall x \in U, \exists \varepsilon > 0$ tal que $x \in B_{\varepsilon} (x) \subset U$.
			\begin{enumerate}
				\item $X = \R^n, \text{ } d((x_i),(y_i)) = \sqrt{ \\sum_{i=1}^{n} (x_i - y_i)^2 }$. Así, se obtiene la topología estándar.

				\item $X$ arbitrario, $d$ métrica discreta $d(x,y) = \begin{cases}
					1 \quad x \neq y \\
					0 \quad x = y.
				\end{cases}$ Así, se obtiene la topología discreta: $\tau = 2^X$.
			\end{enumerate}

			\item (Topología indiscreta): $X$ arbitrario, $\tau = \{ \varnothing , X \}$;

			\item (Topología cofinita): $X$ arbitrario, $\tau_{cof} = \{ U \subset X \text{ | } X \textbackslash U \text{ es finito} \} \cap \{ \varnothing \}$ (queda como ejercicio verificar que es topología).
		\end{enumerate}
	\end{eg}

	\subsection{Base de una topología}

	Una base es un subconjunto "manejable" de $\tau$ que la describe completamente!

	\begin{definition}[base]
		$X$ es conjunto. $\mathcal{B} \subset 2^X$ es base para alguna topología si:
		\begin{enumerate}
			\item $\forall x \in X, \exists B \in \mathcal{B}$ tal que $x \in B \text{ } \left( \bigcup_{B \in \mathcal{B}} B = X \right)$.

			\item $\forall B_1, B_2 \in \mathcal{B}, \forall x \in B_1 \cap B_2, \exists B_3 \in \mathcal{B}$ tal que $x \in B_3 \subset B_1 \cap B_2$.
		\end{enumerate}
	\end{definition}

	\begin{definition}[topología inducida]
		La topología inducida por la base $\mathcal{B}$ en $X$ es:

		\[
		\tau = \{ U \subset X \text{ | } \forall x \in U, \exists B \in \mathcal{B} \text{ tal que } x \in B \subset U \}.
		\]
	\end{definition}

	\begin{note}
		$\mathcal{B} \subset \tau$.
	\end{note}

	\begin{lemma}
		$\tau$, definido arriba, es una topología.
	\end{lemma}

	\begin{eg}
		$(X,d)$ espacio métrico $\Rightarrow \mathcal{B} = \{ B_{\varepsilon} (x) \text{ | } x \in X, \varepsilon > 0 \}$ es base de la topología métrica.
	\end{eg}

	\section{Clase 3 (08/08): Bases, Topología producto [13,15]}

	\begin{proof}[Proof ] (lema 1.8)
		\begin{enumerate}
			\item $\varnothing, X \in \tau \text{ : } \varnothing \in \tau$ por vacuidad y $X \in \tau$ por propiedad $(1)$ de $\mathcal{B}$.

			\item $\tau$ cerrado bajo unión: $\{ U_{\alpha} \}_{\alpha \in A}$ colección con $U_{\alpha} \in \tau, \text{ } \mathcal{U} = \bigcup_{\alpha} U_{\alpha}$.
			\begin{align*}
				\text{Si } x \in \mathcal{U} & \Rightarrow x \in U_{\alpha} \text{ para algún } \alpha \\
				& \Rightarrow x \in B \subset U_{\alpha} \text{ para algún } B \in \mathcal{B} \\
				& \Rightarrow x \in B \subset \mathcal{U}.
			\end{align*}
			Por lo tanto, $\mathcal{U} \in \tau$.

			\item $\tau$ cerrado bajo intersección finita: $U_1,\dots,U_n \in \tau, \mathcal{U} = U_1 \cap \cdots \cap U_n$. Sea $x \in \mathcal{U} \Rightarrow x \in U_i \text{ } \forall i \text{ } (U_i \in \tau) \Rightarrow x \in B_i \subset U_i \text{ } \forall \text{ } i, B_i \in \mathcal{B}$. Propiedad $(2)$ implica $x \in B \subset B_1 \cap \cdots \cap B_n \subset U_1 \cap \cdots \cap U_n = \mathcal{U}$. Por lo tanto, $\mathcal{U} \in \tau$.
		\end{enumerate}
	\end{proof}

	\begin{note}
		Si $B$ base genera $\tau \Rightarrow B \subset \tau$.
	\end{note}

	\begin{definition}[topología generada]
		$\tau$ topología está generada por una base $B$ sin $B$ es base, y $\tau$ es topología generada por $B$.
	\end{definition}

	Utilidad: Dada $\tau$ topología a estudiar, queremos encontrar base $B$ que la describa.

	\begin{eg}
		$(X,d)$ espacio métrico, $\mathcal{B} = \{ B_{\varepsilon} (x) \text{ | } \varepsilon > 0, x \in X \}$ es base para la topología métrica.  
	\end{eg}

	\begin{proof}[Proof ] Probamos que $B$ es base.
		\begin{enumerate}
			\item Notar $X = \bigcup_{x \in X} B_1 (x)$. Por lo tanto, $\bigcup_{B \in \mathcal{B}} B = X$.

			\item $B_1,B_2 \in \mathcal{B}, B_1 = B_{\varepsilon_1} (x_1), B_2 = B_{\varepsilon_2} (x_2)$. Sea $x \in B_1 \cap B_2$. Queremos encontrar $\varepsilon > 0$ tal que $B_{\varepsilon} (x) \subset B_1 \cap B_2$. Por desigualdad triangular, tenemos que $\varepsilon = \min \{ \varepsilon_1 - d(x,x_1), \varepsilon_2 - d(x,x_2) \}$ sirve.  
		\end{enumerate}
	\end{proof}

	\begin{note}
		\begin{enumerate}
			\item Una base no es necesariamente una topología ($(1)$ y $(2)$ pueden fallar).

			\item Si $B$ es base y $\tau$ topología, $B \subset \tau \nRightarrow \tau$ es generada por $B$.
		\end{enumerate}
	\end{note}

	\begin{eg}
		Topología del límite inferior en $\R \text{ : } B_l = \{ [a,b) \text{ | } a,b \in \R , a < b \}$ (se deja como ejercicio demostrar que $B_l$ es base).
	\end{eg}

	\begin{definition}[topología del límite inferior]
		$B_l$ genera la topología del límite inferior $\tau_l$.
	\end{definition}

	\begin{remark} \text{}
		\begin{enumerate}
			\item $\tau_l$ no es $\tau_{std}$ ($[a,b)$ abierto en $\tau_l$ pero no en $\tau_{std}$

			\item $\tau_{std} \subset \tau_l$ (la demostración de esto queda como ejercicio).

			\item (Intuición): Si $0 \in \R, y \in \R$ (para $\tau_{std}$, $y$ cerda de $0$ si $|y| < \varepsilon$). Para $\tau_l$, $y$ cerca de $0$, si $y \in [0,\varepsilon)$ ($0\leq y < \varepsilon$ para $\varepsilon > 0$ chiquito).
		\end{enumerate}
	\end{remark}

	\subsection{Comparación de topologías}

	\begin{definition}[topologías finas]
		$\tau,\tau'$ topologías en $X$, decimos que $\tau'$ es más fina que $\tau$ si $\tau' \supset \tau$. Decimos que $\tau$ y $\tau'$ son comparables si $\tau' \supset \tau$ o $\tau \supset \tau'$. 
	\end{definition}

	\begin{eg}
		$\tau_l$ es más fina que $\tau'$.
	\end{eg}

	\begin{eg}
		$\forall \tau$ topología en $X$, $\{ \varnothing, X \} \subset \tau \subset 2^X$. Donde $\{ \varnothing, X \}$ es llamada la topología indiscreta (todos cercanos entre sí) y $2^X$ la topología discreta (todos lejanos entre sí).    
	\end{eg}

	En conclusión, si $\tau'$ es más fina que $\tau$, los puntos están más lejanos respecto a $\tau'$ que a $\tau$

	\section{Clase 4 (11/08): Topología producto [15] e inducida [16]}

	\begin{lemma}
		$\mathcal{B},\mathcal{B}'$ bases en $X$ que generan la topología $\tau,\tau'$ respectivamente. Entonces
		\begin{align*}
			\tau' \supset \tau & \iff (\text{todo elemento de } \mathcal{B} \text{ está en } \tau'); \\
			& \iff \forall B \in \mathcal{B}, B \text{ es unión de elementos de } \mathcal{B}'; \\
			& \iff \forall B \in \mathcal{B}, \forall x \in B, \exists B' \in \mathcal{B}' \text{ tal que } x \in B' \subset B
		.\end{align*}
	\end{lemma}

	\begin{lemma}
		$\mathcal{B}_{X \times Y} \coloneq \{ U \times U' \text{ | } U \text{ abierto en } X, U' \text{ abierto en } Y \}$ es una base para una topología.
	\end{lemma}

	\begin{definition}[topología producto]
		Topología producto en $X \times Y$ es la generada por $\mathcal{B}_{X \times Y}$.
	\end{definition}

	\begin{proof}[Proof ] (lemma 1.13.)
		\begin{enumerate}
			\item Como $X \times Y \in \mathcal{B}_{X \times Y} \Rightarrow \bigcup_{B \in \mathcal{B}_{X \times Y}} B = X \times Y$.

			\item Tomar $B_1 = U_1 \times U_1' \in \mathcal{B}_{X \times Y}, B_2 = U_2 \times U_2' \in \mathcal{B}_{X \times Y}, (x,y) \in B_1 \cap B_2 \text{ } (U_1,U_2 \text{ abiertos en } X \text{ y } U_1',U_2' \text{ abiertos en } Y)$. Notar que:
			\[
			B_1 \cap B_2 = (U_1 \times U_1') \cap (U_2 \times U_2') = \underbrace{(U_1 \cap U_2)}_{\text{abto. en } X} \times \underbrace{(U_1' \cap U_2')}_{\text{abto. en } Y} \in \mathcal{B}_{X \times Y}.
			\]
		\end{enumerate}
	\end{proof}

	\begin{note}
		Misma demostración (salvo modificaciones esperables) implica que si $\mathcal{B}_X$ es base de $X$, $\mathcal{B}_Y$ base de $Y$, $\mathcal{B}_{X \times Y}' \coloneq \{ B \times B' \text{ | } B \in \mathcal{B}_X, B' \in \mathcal{B}_Y \}$ es base y genera la misma topología generada por $\mathcal{B}_{X \times Y}$.  
	\end{note}

	\begin{eg}[importante]
		$\R^2 = \R \times \R$. Propiedad: topología estándar de $\R^2$ (métrica euclidiana) es igual a la topología producto en $\R \times \R$ (cada uno con su topología estándar).
		\begin{itemize}
			\item Topología estándar en $\R^2$: generada por base $\mathcal{B} = \{ B_{\varepsilon} (x,y) \text{ | } (x,y) \in \R^2, \varepsilon > 0 \}$.

			\item Topología producto en $\R^2$: generada por base $\mathcal{B}' = \{ (a,b) \times (c,d) \text{ | } a < b, c < d \}$.  
		\end{itemize}
	\end{eg}

	\begin{ex}
		Verificar para $\R^n$.
	\end{ex}

	\begin{definition}[topología inducida]
		$\tau|_Y \coloneq \{ Y \cap U \text{ | } U \in \tau \}$ es topología en $Y$. La llamamos topología en $Y$ inducida por $X$.  
	\end{definition}

	\begin{proof}[Proof ] (topología inducida es topología)
		\begin{enumerate}
			\item $\varnothing = \varnothing \cap Y, Y = X \cap Y$.

			\item Si $U_{\alpha} \in \tau|_Y, \alpha \in A \Rightarrow U_{\alpha} = U_{\alpha}' \cap Y$ con $U_{\alpha}' \in \tau \Rightarrow \bigcup_{\alpha \in A} U_{\alpha} = \bigcup_{\alpha \in A} (U_{\alpha \in A} \cap Y) = \left[ \bigcup_{\alpha \in A} U_{\alpha} \right] \times Y \in \tau|_Y$.

			\item $U_1,\dots,U_n \in \tau|_Y, U_i = U_i' \cap Y \Rightarrow U_1 \cap \cdots \cap U_n = (U_1' \cap Y) \cap \cdots \cap (U_n' \cap Y) = (U_1' \cap \cdots \cap U_n') \cap Y \in \tau|_Y$.
		\end{enumerate}
	\end{proof}

	\begin{lemma}
		$\mathcal{B}|_Y \coloneq \{ Y \cap B \text{ | } B \in \mathcal{B} \}$ es base para la topología en $Y$ inducida por $X$.  
	\end{lemma}

	\begin{remark}
		Cuidado: La noción de abierto depende de la topología a especificar.
	\end{remark}

	\begin{eg}
		En $X = \R, Y = [0,1] \cup (2,3) \cup \{ 4 \}$. Notar que:
		\begin{itemize}
			\item $Y$ es abierto en $Y$, pero no es abierto en $X$.

			\item $[0,1]$ también abierto en $Y : [0,1] = Y \cap (-1,2)$.

			\item $\{ 4 \}$ también abierto en $Y : \{ 4 \} = Y \cap (3,5)$.
		\end{itemize}
	\end{eg}

	\begin{note}
		Si $U \subset Y$ es abierto en $X \Rightarrow$ abierto en $Y$.
	\end{note}

	\begin{lemma}
		$Y \subset X, \tau|_Y \subset \tau \iff Y \text{ es abierto en } X$.
	\end{lemma}

	\begin{prop}
		$X,Y$ espacios topológicos, $A \subset X, B \subset Y$.
		\begin{align*}
			\text{En } A \times B \rightarrow & \text{ topología inducida desde } X \times Y \text{ (con topología producto) } \\
			\rightarrow & \text{ topología producto desde } A \text{ y } B \text{ (con topología inducida} \\
			& \text{ por } X,Y \text{ respectivamente)}
		.\end{align*}
		\noindent Estas topologías son la misma.
	\end{prop}

	\begin{proof}[Proof ]
		Elemento de topología primera: $U = U' \cap A \times B$ \\
		Elemento de topología segunda: $U$ es unión de productos $V \times V'$ con $V$ abierto en $A$, $V'$ abierto en $B$. Notar que $V \times V' = (W \cap A) \times (W' \cap B) = (W \times W') \cap A \times B$.
	\end{proof}

	\section{Clase 5 (13/08): Cerrados, clausura, puntos límites [17]}

	\begin{definition}[conjunto cerrado]
		$X$ espacio topológico, $C \subset X$ es cerrado si $X \textbackslash C$ es abierto. 
	\end{definition}

	\begin{lemma}
		\text{}
		\begin{enumerate}
			\item $X, \varnothing$ son cerrados;

			\item Si $C_{\alpha} \subset X$ cerrados, $\alpha \in A \Rightarrow \bigcap_{\alpha} C_{\alpha}$ es cerrado;

			\item Si $C_1, \dots, C_n$ cerrados, entonces $C_1 \cup \cdots \cup C_n$ es cerrado.
		\end{enumerate}
	\end{lemma}

	\begin{proof}[Proof ]
		\text{}
		\begin{enumerate}
			\item $X = X \textbackslash \varnothing, \text{ } \varnothing = X \textbackslash X$;

			\item $C_{\alpha} = \displaystyle\bigcap_{\alpha \in A} C_{\alpha} \Rightarrow X \textbackslash C = X \textbackslash \displaystyle\bigcap_{\alpha \in A} C_{\alpha} = \underbrace{\displaystyle\bigcup_{\alpha \in A} (\overbrace{X \textbackslash C_{\alpha}}^{\text{abto}})}_{\text{abto}}$;

			\item $C = C_1 \cup \cdots \cup C_n \Rightarrow X \textbackslash C = X \textbackslash (C_1 \cup \cdots \cup C_n) = \underbrace{(\overbrace{X \textbackslash C_1}^{\text{abto}}) \cap \cdots \cap (\overbrace{X \textbackslash C_n}^{\textrm{abto}})}_{\text{abto}}$.
		\end{enumerate}
	\end{proof}

	\begin{eg}
		\text{}
		\begin{enumerate}
			\item $X = \R, [a,b]$ es cerrado ($\R \textbackslash [a,b] = (-\infty, a) \cup (b, \infty)$);

			\item $(X,d)$ espacio métrico (+ topología métrica) $\Rightarrow \overline{B_{\varepsilon}} (x)$ es cerrado. Luego, $X \textbackslash \overline{B_{\varepsilon}} (x) = \bigcup_{y \in X \textbackslash \displaystyle\overline{B_{\varepsilon}} (x)} B_{d(x,y)-\varepsilon} (y)$ (abierto en topología métrica);

			\item $X$ con la topología discreta $\Rightarrow$ todo subconjunto de $X$ es abierto y cerrado!
		\end{enumerate}
	\end{eg}

	\begin{definition}[cerrado topología inducida]
		$X$ espacio topológico, $Y \subset X$ (con la topología inducida), $C \subset Y$ es cerrado en $Y$ si es cerrado en la topología inducida.
	\end{definition}

	\begin{lemma}
		$C$ es cerrado en $Y$ si y solo si $C = C' \cap Y$ con $C'$ cerrado en $X$.
	\end{lemma}

	\begin{proof}[Proof ]
		$\begin{aligned}[t]
			C \subset Y \text{ es cerrado en } Y \iff \ & Y \textbackslash C \text{ es abierto en } Y \\
			\iff \ & Y \textbackslash C = U \cap C \text{ con } U \subset X \text{ abierto} \\
			\iff \ & C = (X \textbackslash U) \cap Y = C' \cap Y \text{, con } \\
			& C' = X \textbackslash U \text{ cerrado}
		.\end{aligned}$
	\end{proof}

	\begin{definition}[clausura e interior]
		$X$ espacio topológico, $A \subset X$:
		\begin{enumerate}
			\item El interior de $A$ es $\mathring{A} =$ unión de todos los abiertos contenidos en $A$;

			\item La clausura de $A$ es $\overline{A} =$ intersección de todos los cerrados que contienen $A$.
		\end{enumerate}
	\end{definition}

	\begin{remark}
		\text{ }
		\begin{enumerate}
			\item $\mathring{A}$ es abierto, $\overline{A}$ es cerrada, $\mathring{A} \subset A \subset \overline{A}$;

			\item $A$ es abierto si y solo si $\mathring{A} = A$. $A$ es cerrado si y solo si $\overline{A} = A$;

			\item $\overline{\overline{A}} = \overline{A}, \ \mathring{\mathring{A}} = \mathring{A}$;

			\item El interior $\mathring{A}$ es el abierto mas grande contenido en $A$ y la clausura $\overline{A}$ es el cerrado mas pequeño que contiene a $A$.
		\end{enumerate}
	\end{remark}

	\begin{prop}
		$X$ espacio topológico, $A \subset X$ cualquiera, $x \in X$.
		\begin{align*}
			x \in \overline{A} & \iff \forall U \text{ abierto conteniendo a } X \text{, se tiene } A \cap U \neq \varnothing \tag*{($*$)} \\
			& \iff \text{ toda vecindad de } x \text{ interseca a } A \\
			& \iff A \text{ contiene puntos arbitrariamente cercanos a } X \text{ (según la topología)}
		.\end{align*}
	\end{prop}

	\begin{corollary}
		$C \subset X$ es cerrado si y solo si $\forall x \in X$, si toda vecindad de $x$ contiene un punto de $C$, entonces $x \in X$.
	\end{corollary}

	\begin{proof}[Proof ] (proposición 1.24) \\
		\Ifstep Suponer que $x \not\in \overline{A}$. Entonces $\exists C$ cerrado con $A \subset C$ y $x \not\in C$. Luego, tomar $U \coloneq C \textbackslash C$ abierto. Entonces, $A \cap U = \varnothing$ y $x \in U$. Es decir, negamos $(*)$.

		\noindent \Onlyifstep Negamos $(*) \implies \exists U$ abierto con $x \in U$ y $U \cap A = \varnothing$. Luego, $C = X \textbackslash U$ cerrado con $A \subset C$ y $x \not\in C$. Entonces, $x \not\in A$.
	\end{proof}

	\begin{definition}[puntos de acumulación]
		$A \subset X$. Decimos que $x \in X$ es punto límite/de acumulación de $A$ si $\forall \ U$ abierto conteniendo a $x$, se tiene que $U \cap ( A \textbackslash \{ x \} ) \neq \varnothing$. Escribimos $A' \coloneq \{ \text{puntos límite de } A \}$.
	\end{definition}

	\begin{eg}
		En $\R$, tenemos lo siguiente:
		\begin{center}
		\begin{tabular}{ | c | c | c | c | }
			\hline
			$A$ & $\mathring{A}$ & $\overline{A}$ & $A'$ \\
			\hline
			$(a,b)$ & $(a,b)$ & $[a,b]$ & $[a,b]$ \\
			$[a,b)$ & $(a,b)$ & $[a,b]$ & $[a,b]$ \\
			$[a,b]$ & $(a,b)$ & $[a,b]$ & $[a,b]$ \\
			$[0,1] \cup \{2\}$ & $(0,1)$ & $[0,1] \cup \{2\}$ & $(0,1)$ \\
			\hline
		\end{tabular}
		\end{center}
		\noindent Notar que $2$ no es punto de acumulación.
	\end{eg}


	\section{Clase 6 (18/08): Espacios Hausdorff, convergencia [17]}

	\begin{remark}
		$x \in A' \iff x \in \overline{A \setminus \{x\}}$. 
	\end{remark}

	\begin{lemma}
		$\forall \ A \subset X,\ \overline{A} = A \cup A'$.
	\end{lemma}

	\begin{proof}[Proof ]
		\fbox{$\supset$} Notar que $A \subset \overline{A}$. Si $x \in A' \implies x \in \overline{A \setminus \{x\}} \subset \overline{A} \ (*)$. Notar que $(*) A \subset B \implies \overline{A} \subset \overline{B}$. Por lo tanto $A' \subset \overline{A}$. Entonces, $A \cup A' \subset \overline{A}$.

		\noindent \fbox{$\subset$} $\ (\overline{A} \subset A \cup A', \text{ equiv: } \overline{A}\setminus A \subset A')$ Si $x \in \overline{A} \setminus A$. Entonces, $x \not\in A$ y $\forall \ U \ni x$ abierto se tiene $A \cap U \neq \varnothing$. Como $x \not\in A \implies (A \setminus \{x\}) \cap U \neq \varnothing$. Entonces, $x \in A'$.  
	\end{proof}

	\begin{remark}
		$A'$ no es necesariamente cerrado.
	\end{remark}

	\begin{eg}
		$X = \{ a,b \};\ \tau = \{ \varnothing, X \}$ ($a,b$ indistinguibles desde el punto de vista de $\tau$). $A = \{ b \} \implies A' = \{b\} $ (no es cerrado). $a \not\in A' \iff a \not\in \overline{A \setminus \{a\} } = \overline{\varnothing} = \varnothing$. $b \in A \iff b \in \overline{A \setminus \{b\} } = \overline{\{a\}} = \{a,b\}$.      
	\end{eg}

	\noindent \textbf{Problemas:}
	\begin{itemize}
		\item Subconjuntos finitos no tienen topología discreta;

		\item Subconjuntos finitos no son cerrados.
	\end{itemize}

	\begin{lemma}
		Si $X$ es espacio topológico arbitrario. Son equivalentes:
		\begin{enumerate}
			\item Todos los subconjuntos finitos de $X$ tienen la topología discreta.

			\item Todos los subconjuntos finitos de $X$ son cerrados.
		\end{enumerate}
	\end{lemma}

	\begin{definition}[espacios $T_1$]
		Un espacio topológico $X$ es $T_1$ (cumple el axioma $T_1$) si sus subconjuntos finitos son cerrados.
	\end{definition}

	\begin{eg}
		$X$ con la topología indiscreta NO es $T_1$ si $\# X \geq 2$.
	\end{eg}

	\begin{eg}
		$X$ con topología cofinita es $T_1$. En la topología
		\[
		\{ \text{subconjuntos cerrados} \} = \{ \text{conjuntos finitos} \}
		\]
	\end{eg}

	\begin{lemma}
		$X$ es $T_1,\ A \subset X \implies A'$ es cerrado.
	\end{lemma}

	\begin{proof}[Proof ]
		(Queremos $\overline{A'} = A'$, i.e. $\overline{A'} \setminus A' = \varnothing$) Suponer que $x \in \overline{A'},\ x \not\in A'$. Si $x \not\in A'$, entonces $\exists\ U$ abierto con $x\in U$ y $U\cap A \subset \{x\}$. Si $x \in \overline{A'}$, entonces $A'\cap U \neq \varnothing$. Luego, $\exists\ y \in U \cap A' \ (y\neq x)$. Como $X$ es $T_1$, entonces $\{x\}$ es cerrado. Luego, $X \setminus \{x\}$ es abierto, y con ello tenemos que $U \setminus \{x\}$ es abierto. Si $V = U \setminus \{x\}$ abierto que contiene a $y \ (y\in A')$, entonces $V$ contiene puntos de $A$, distintos de $y$. Luego, $\exists\ z \in A \cap V$. Así, $z \in A \cap U$ y $z \neq x$. Contradicción! \textreferencemark
	\end{proof}

	\begin{definition}[espacios $T_2$ o Haussdorff]
		Un espacio topológico $X$ es $T_2$ (o Hausdorff), si $\forall \ x \neq y$ en $X$ existen $U,U' \subset X$ abiertos \underline{disjuntos} con $x \in U,\ y \in U'$.
	\end{definition}

	\begin{eg}
		$X$ con la topología cofinita, con $\# X = \infty$ es $T_1$ pero no es Hausdorff. Veamos que esto es así. Si $x \neq y \in X,\ x \in U,\ y \in U'$ abiertos ($X \setminus U,\ X \setminus U'$ finitos), entonces $(X \setminus U) \cup (X \setminus U')$ finito. Luego, $X \setminus (U \cap U')$ finito. Así, $U \cap U'$ infinito, por lo que $U \cap U'$ no puede ser disjunto.  
	\end{eg}

	\begin{lemma}
		$X$ Hausdorff $\implies X$ es $T_1$.
	\end{lemma}

	\begin{proof}[Proof ]
		($X$ es $T_1 \iff$ subconjuntos finitos son cerrados $\iff$ singletons son cerrados) $\rightarrow$ (veremos el último si y solo si) Sea $x \in X$, queremos que $X \setminus \{x\}$ sea abierto. Si $y \neq x$, dado que $X$ es Hausdorff, $\exists\ U_y, U_y'$ abiertos disjuntos con $y \in U_y,\ x \in U_y'$. Luego, $x \not\in U_y$. Por lo tanto, $X \setminus \{ x \} = \displaystyle\bigcup_{y \neq x} U_y$ es abierto.
	\end{proof}

	\begin{eg}
		$(X,d)$ espacio métrico, $X$ es Hausdorff con la topología métrica.
	\end{eg}

	\begin{corollary}[secreto]
		Existen topologías que no vienen de métricas.
	\end{corollary}

	\begin{proof}[Proof ][del ejemplo]
		Para la topología métrica, bolas abiertas son abiertos. Si $x \neq y$, entonces $U = B_{\frac{d(x,y)}{2}}(x),\ U' = B_{\frac{d(x,y)}{2}}(y)$.
	\end{proof}

	En $X$ con la topología cofinita, $X = \{ x_1, x_2, x_3,\dots \}$ infinito contable. Definimos $y_n = x_n$ con $n \geq 1$ (cada elemento de $X$ aparece exactamente una vez). Cada abierto $\varnothing \neq U \subset X$ contiene a $y_n \ \forall n \geq \N$ ($N$ depende de $U$). (próxima clase: $y_n \to x \ \forall x \in X$). 
	


	







% end lessons
\end{document}
