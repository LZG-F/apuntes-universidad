\section{Clase (03/09)}

\begin{eg}[N°2]
	\[ (PC) \ \begin{cases}
		y'(t) = \frac{t+2}{t^2+(y(t))^2} \quad \forall t \in I \subset \R, \\
		y(0) = 1
	\end{cases} \]
	Sea $\Omega = \R \setminus \{(0,0)\}$ y $f(x,y) = \frac{x+2}{x^2 + y^2} \quad \forall (x,y) \in \Omega$. Como $f \in C^{\infty}(\Omega) \implies f \in \text{Lip}_{loc}(y;\Omega)$, y así, existe $\delta > 0$ tal que (PC) admite una única solución en $I_{\delta} = [-\delta,\delta]$, que denotamos por $\varphi \in C^{1}(I_{\delta};\R)$. De hecho, $\varphi \in C^{\infty}((\delta,\delta);\R)$ y podemos escribir un polinomio de MacLaurin de orden $2$ con resto de Peano:
	\[ \varphi(t) = \varphi(0) + \varphi'(0)t + \frac{\varphi''(0)}{2}t^2 + o(t^2) \]
	para $t \to 0$. Sabemos que $\varphi(0) = 1$ y $\varphi'(0) = \frac{0+2}{0+1}=2$. Además, derivando $(PC)_{(1)}$ se obtiene:
	\[ \varphi''(t) = \frac{t^2 + (\varphi(t))^2 - (t+2)(2t+2\varphi(t)\varphi'(t))}{(t^2 + (\varphi(t))^2)^2} \quad \forall t \in \mathring{I}_{\delta}, \]
	con lo que: $\varphi''(0) = -7$. Luego:
	\[ \varphi(t) = 1 + 2t - \frac{7}{2}t^2 + o(t^2) \]
	para $t \to 0$.
\end{eg}

\subsection*{IV. Unicidad global y solución global del problema de Cauchy}

\textbf{4.1 Unicidad global de solución del (PC).}

\begin{lemma}[Grönwall]
	Sean $x_0 < x_1$ números reales, $h \in \R$ y $u,k \in C([x_0,x_1];\R)$ funciones tales que $k(x) \geq 0 \quad \forall x \in [x_0,x_1]$. Entonces,
	\begin{enumerate}
		\item[(A)] Si $u(x) \leq h + \int_{x_{0}}^{x} k(s)u(s) ds \quad \forall x \in [x_0,x_1]$, entonces
		\[ u(x) \leq h\cdot \exp\left( \int_{x_{0}}^{x} h(s)ds \right) \quad \forall x \in [x_{0},x_{1}] \]

		\item [(B)] Si $u(x) \leq h + \int_{x}^{x_{1}}h(s)u(s)ds \quad \forall x \in [x_{0},x_{1}]$, entonces:
		\[ u(x) \leq h \cdot \exp \left( \int_{x}^{x_{1}}h(s)ds \right) \quad \forall x \in [x_{0},x_{1}] \]
	\end{enumerate}
\end{lemma}
\begin{proof}[Proof][sólo apartado (A)]
	Definamos la función 
	\[ v(x) = \int_{x_{0}}^{x} k(s)u(s)ds \quad \forall x \in [x_{0},x_{1}], \]
	y así, $v \in C^1([x_{0},x_{1}];\R),\ v(x_{0}) = 0$ y $v'(x) = k(x)u(x) \quad \forall x \in (x_{0},x_{1})$. Como $u(x) \leq h + v(x) \quad \forall x \in [x_{0},x_{1}]$, entonces
	\begin{align*}
		& h(x)u(x) \leq h\cdot k(x) + v(x)k(x) \quad \forall x \in [x_{0},x_{1}] \\
		\iff \ & v'(x) \leq h\cdot k(x) + k(x)v(x) \quad \forall x \in [x_{0},x_{1}]
	.\end{align*}
	Luego:
	\begin{align*}
		& \exp \left( - \int_{x_{0}}^{x} h(s)ds \right) v'(x) - \exp \left(- \int_{x_{0}}^{x}h(s)ds \right)k(x)v(x) \\
		& \leq h\cdot k(x) \exp \left( - \int_{x_{0}}^{x} k(s)ds \right) \\
		\iff \ & \frac{d}{dx} \left[ \exp \left(-\int_{x_{0}}^{x}k(s)ds\right)v(x) \right] \leq -h \frac{d}{dx} \left[ \exp \left( - \int_{x_{0}}^{x} h(s)ds \right) \right]
	.\end{align*}
	Integrando en $[x_{0},x]$, con $x \in (x_{0},x_{1}]$, se obtiene:
	\begin{align*}
		& \exp \left( -\int_{x_{0}}^{x}k(s)ds \right) v(x) \leq h \left[ 1 - \exp \left( -\int_{x_{0}}^{x} k(s)ds \right) \right] \\
		\implies \ & v(x) \leq h \cdot \exp \left( \int_{x_{0}}^{x} k(s)ds \right) -h \\
		\iff \ & u(x) \leq h + v(x) \leq h \cdot \exp \left( \int_{x_{0}}^{x}k(s)ds \right) \quad \forall x \in [x_{0},x_{1}]
	\end{align*}
\end{proof}

\begin{observe}~
	\begin{enumerate}
		\item Si $h=0$, el lema de Grönwall asefura que $u(x) \leq 0 \quad \forall x \in [x_{0},x_{1}]$;

		\item Es posible utilizar una única formulación que agrupe las dos condiciones del lema:
		\begin{align*}
			& u(x) \leq h + \left| \int_{x_{0}}^{x}k(s)u(s)ds \right| \quad \forall x,x_{0}\in I \\
			\implies \ & u(x) \leq h \cdot \exp \left( \left| \int_{x_{0}}^{x} k(s)ds \right| \right)
		.\end{align*}
	\end{enumerate}
\end{observe}

\begin{theorem}[unicidad global]
	Sea $\Omega \subset \R^{n+1}$ un conjunto abierto, $(x_{0},y_{0}) \in \Omega$ y $f \in C(\Omega;\R^n) \cap \text{Lip}_{loc}(y;\Omega)$. Si $(I_1,\varphi_1)$ y $(I_2,\varphi_{2})$ son soluciones locales del problema de Cauchy
	\[ (PC) \ \begin{cases}
		y'(x) = f(x,y(x)) \quad \forall x \in I \subset \R, \\
		y(x_{0}) = y_{0}
	\end{cases} \]
	entonces $\varphi_{1}(x) = \varphi_{2}(x)\quad \forall x \in I_1 \cap I_2$ (acá, $I_1,I_{2} \subset \R$ son intervalos que conienen a $x_{0}$).
\end{theorem}
\begin{proof}[Proof ]
	Sabemos que $(I_1 \cap I_{2}) \setminus \{x_{0}\} \neq \varnothing$, y así, tomemos $x_{1} \in (I_{1} \cap I_{2}) \setminus \{x_{0}\}$, con $x_{1} > x_{0}$. Definamos el conjunto:
	\[ K = \{ (s,\varphi_{1}(s)) \ \big| \ s \in [x_{0},x_{1}] \} \cup \{ (s,\varphi_{2}(s) \ \big| \ s \in [x_{0},x_{1}] \} \] 
	Luego, $K \subset \Omega$ es compacto. Como $f$ es continua en $\Omega$, resulta que: $\sup_{(x,y) \in K} |f(x,y)| < \infty \implies f \in \text{Lip}(y;K)$, y sea $L_k > 0$ una constante de Lipschitz (global) para $f$, respecto a su segunda variable en $K$. Sabemos que $\varphi_{i}(x) = y_0 + \int_{x_{0}}^{x} f(s,\varphi(s))ds \quad \forall x \in [x_0,x_{1}],\ \forall i \in \{1,2\}$. Entonces,
	\begin{align*}
		|\varphi_{1}(x) - \varphi_{2}(x)| &= \left| \int_{x_{0}}^{x} [f(s,\varphi_{1}(s)) - f(s,\varphi_{2}(s))]ds \right| \\
		& \leq \int_{x_{0}}^{x} \big|f(s,\varphi_{1}(s)) - f(s,\varphi(s))\big| ds \\
		& \leq L_K \int_{x_{0}}^{x} \big| \varphi_{1}(s) - \varphi_{2}(s) \big| ds \quad \forall x \in [x_{0},x_{1}]
	.\end{align*}
	Así (gracias al lema de Grönwall), $\big| \varphi_{1}(x) - \varphi_{2}(x) \big| = 0 \quad \forall x \in [x_{0},x_{1}] \implies \varphi_{1}(x) = \varphi_{2}(x) \quad \forall x \in [x_{0},x_{1}]$.
\end{proof}
