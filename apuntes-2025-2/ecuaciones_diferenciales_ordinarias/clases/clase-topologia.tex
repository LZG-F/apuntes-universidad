\documentclass[a4paper]{report}
\usepackage[utf8]{inputenc}
\usepackage[T1]{fontenc}
\usepackage{textcomp}

\usepackage{url}

% \usepackage{hyperref}
% \hypersetup{
%     colorlinks,
%     linkcolor={black},
%     citecolor={black},
%     urlcolor={blue!80!black}
% }

\usepackage{graphicx}
\usepackage{float}
\usepackage[usenames,dvipsnames]{xcolor}

% \usepackage{cmbright}

\usepackage{amsmath, amsfonts, mathtools, amsthm, amssymb}
\usepackage{mathrsfs}
\usepackage{cancel}

\newcommand\N{\ensuremath{\mathbb{N}}}
\newcommand\R{\ensuremath{\mathbb{R}}}
\newcommand\Z{\ensuremath{\mathbb{Z}}}
\renewcommand\O{\ensuremath{\emptyset}}
\newcommand\Q{\ensuremath{\mathbb{Q}}}
\newcommand\C{\ensuremath{\mathbb{C}}}
\let\implies\Rightarrow
\let\impliedby\Leftarrow
\let\iff\Leftrightarrow
\let\epsilon\varepsilon

% demostraciones bidireccionales

\newcommand{\Onlyifstep}{%
	\begingroup
	\fboxsep=1pt
	\raisebox{1.2ex}{\fbox{\raisebox{-1.2ex}{$\Rightarrow$\hspace{-0.05em}}}}%
	\endgroup
	\hspace{0.5em}%
	}
\newcommand{\Ifstep}{%
	\begingroup
	\fboxsep=1pt
	\raisebox{1.2ex}{\fbox{\raisebox{-1.2ex}{\hspace{-0.05ex}$\Leftarrow$}}}%
	\endgroup
	\hspace{0.5em}%
	}

% horizontal rule
\newcommand\hr{
    \noindent\rule[0.5ex]{\linewidth}{0.5pt}
}

\usepackage{tikz}
\usepackage{tikz-cd}

% theorems
\usepackage{thmtools}
\usepackage[framemethod=TikZ]{mdframed}
\mdfsetup{skipabove=1em,skipbelow=0em, innertopmargin=5pt, innerbottommargin=6pt}

\theoremstyle{definition}

\makeatletter

\declaretheoremstyle[headfont=\bfseries\sffamily, bodyfont=\normalfont, mdframed={ nobreak } ]{thmgreenbox}
\declaretheoremstyle[headfont=\bfseries\sffamily, bodyfont=\normalfont, mdframed={ nobreak } ]{thmredbox}
\declaretheoremstyle[headfont=\bfseries\sffamily, bodyfont=\normalfont]{thmbluebox}
\declaretheoremstyle[headfont=\bfseries\sffamily, bodyfont=\normalfont]{thmblueline}
\declaretheoremstyle[headfont=\bfseries\sffamily, bodyfont=\normalfont, numbered=no, mdframed={ rightline=false, topline=false, bottomline=false, }, qed=\qedsymbol ]{thmproofbox}
\declaretheoremstyle[headfont=\bfseries\sffamily, bodyfont=\normalfont, numbered=no, mdframed={ nobreak, rightline=false, topline=false, bottomline=false } ]{thmexplanationbox}


\declaretheorem[numberwithin=chapter, style=thmgreenbox, name=Definition]{definition}
\declaretheorem[sibling=definition, style=thmredbox, name=Corollary]{corollary}
\declaretheorem[sibling=definition, style=thmredbox, name=Proposition]{prop}
\declaretheorem[sibling=definition, style=thmredbox, name=Theorem]{theorem}
\declaretheorem[sibling=definition, style=thmredbox, name=Lemma]{lemma}



\declaretheorem[numbered=no, style=thmexplanationbox, name=Proof]{explanation}
\declaretheorem[numbered=no, style=thmproofbox, name=Proof]{replacementproof}
\declaretheorem[style=thmbluebox,  numbered=no, name=Exercise]{ex}
\declaretheorem[style=thmbluebox,  numbered=no, name=Example]{eg}
\declaretheorem[style=thmblueline, numbered=no, name=Remark]{remark}
\declaretheorem[style=thmblueline, numbered=no, name=Note]{note}

\renewenvironment{proof}[1][\proofname]{\begin{replacementproof}}{\end{replacementproof}}

\AtEndEnvironment{eg}{\null\hfill$\diamond$}%

\newtheorem*{uovt}{UOVT}
\newtheorem*{notation}{Notation}
\newtheorem*{previouslyseen}{As previously seen}
\newtheorem*{problem}{Problem}
\newtheorem*{observe}{Observe}
\newtheorem*{property}{Property}
\newtheorem*{intuition}{Intuition}


\usepackage{etoolbox}
\AtEndEnvironment{vb}{\null\hfill$\diamond$}%
\AtEndEnvironment{intermezzo}{\null\hfill$\diamond$}%




% http://tex.stackexchange.com/questions/22119/how-can-i-change-the-spacing-before-theorems-with-amsthm
% \def\thm@space@setup{%
%   \thm@preskip=\parskip \thm@postskip=0pt
% }

\usepackage{xifthen}

\def\testdateparts#1{\dateparts#1\relax}
\def\dateparts#1 #2 #3 #4 #5\relax{
    \marginpar{\small\textsf{\mbox{#1 #2 #3 #5}}}
}

\def\@lesson{}%
\newcommand{\lesson}[3]{
    \ifthenelse{\isempty{#3}}{%
        \def\@lesson{Lecture #1}%
    }{%
        \def\@lesson{Lecture #1: #3}%
    }%
    \subsection*{\@lesson}
    \testdateparts{#2}
}

% fancy headers
\usepackage{fancyhdr}
\pagestyle{fancy}

% \fancyhead[LE,RO]{Gilles Castel}
\fancyhead[RO,LE]{\@lesson}
\fancyhead[RE,LO]{}
\fancyfoot[LE,RO]{\thepage}
\fancyfoot[C]{\leftmark}
\renewcommand{\headrulewidth}{0pt}

\makeatother

% figure support (https://castel.dev/post/lecture-notes-2)
\usepackage{import}
\usepackage{xifthen}
\pdfminorversion=7
\usepackage{pdfpages}
\usepackage{transparent}
\newcommand{\incfig}[1]{%
    \def\svgwidth{\columnwidth}
    \import{./figures/}{#1.pdf_tex}
}

% %http://tex.stackexchange.com/questions/76273/multiple-pdfs-with-page-group-included-in-a-single-page-warning
\pdfsuppresswarningpagegroup=1

\author{Gilles Castel}

\DeclareMathOperator{\supp}{supp}
\DeclareMathOperator{\spann}{span}
\DeclareMathOperator{\Id}{Id}
\DeclareMathOperator{\Ker}{Ker}
\DeclareMathOperator{\im}{Im}
\DeclareMathOperator{\GL}{GL}
\DeclareMathOperator{\SL}{SL}
\DeclareMathOperator{\Mat}{Mat}

\title{Ecuaciones Diferenciales Ordinarias}
\author{}
\date{Basado en las clases impartidas por - en el segundo semeste del 2025}
\begin{document}

\maketitle
\chapter{}
\setcounter{section}{0}
\section{Clase (20/08)}
\subsection{Algunas ecuaciones no-lineales}

\begin{eg}
	Considere la EDO:
	\[
	y' = -\frac{t^2 + y^2}{t^2 - ty} = -\frac{M(t,y)}{N(t,y)}, \quad t > 0
	\]
	Sean $M(t,y) =  t^2 + y^2$ y $N(t,y) = t^2 - ty, \quad \forall (t,y) \in \R^2$, y así:
	\[
	M(st,sy) = s^2 M(t,y) \text{ y } N(st,sy) = s^2 N(t,y), \quad s,t,y \in \R.
	\]
	En tal caso, conviene introductir el cambio de variable $y = ty$, y así:
	\[
	u+tu' = -\frac{t^2+t^2u^2}{t^2-t^2u} = - \frac{1+u^2}{1-u},
	\]
	y así
	\begin{align*}
		u' = -\frac{1}{t}\cdot \frac{1+u}{1-u} &\iff \frac{du}{dt} = -\frac{1}{t}\cdot \frac{1+u}{1-u} \\
		& \iff \frac{1-u}{1+u} du = - \frac{1}{t} dt \\
		& \iff log ((1+u)^2)-u = -log (t) + C \\
		& \iff (1+u)^2 = \frac{C}{t} e^u \\
		& \iff (t + y(t))^2 = Ct e^{y(t)/t} \quad \text{(solución definida implícitamente)}
	.\end{align*}
\end{eg}
	\noindent En general, si $M,N : \R^2 \to \R$ son dos funciones tales que
	\[
	M(st,sy)= s^{\alpha} M(t,y) \text{ y } N(st,sy) = s^{\alpha} N(t,y),\quad \forall s,t,y\in \R,
	\]
	\noindent para cierto $\alpha> 0$, se sugiere utilizar el cambio de variable $y = tu$.\\

	\noindent \textbf{Ecuación de Bernoulli:} tiene la forma
	\[
	y'(t) + P(t)y(t) = f(t)(y(t))^n \quad \text{con } n \in \N
	\]
	\begin{note}
		los casos $n=0$ y $n=1$ ya han sido estudiados.
	\end{note}

	\noindent Para $n\geq 2$ conviene utilizar el cambio de variable $u = y^{1-n}$. Luego, $u' = (1-n)y^{-n}y'$, es decir: $y' = \frac{1}{1-n} y^n u'$, y así:
	\begin{align*}
		\frac{y^n}{1-n} u' + P(t)y(t) = f(t)(y(t))^n & \iff \frac{1}{1-n} u' + P(t) y^{1-n} = f(t) \\
		& \iff u'(t) + (1-n) P(t) u(t) = (1-n) f(t) \\ 
		& \qquad \text{(se resuelve con factor integrante)}
	.\end{align*}

	\noindent \textbf{Ecuación de Ricatti:} es de la forma
	\[ y' = P(t) + Q(t)y(t) + R(t)(y(t))^2 \]
	\noindent Supongamos que se conoce una solución $y_1(t)$ de esta EDO, es decir:
	\[ y_1'(t) - P(t) - Q(t)y_1(t) - R(t)(y_1(t))^2 = 0\]
	\noindent Luego, definimos $z(t) = y(t) - y_1(t)$, y así:
	\[ y'(t) = z'(t) + y_1'(t) = P(t) + Q(t)(z(t) + y_1(t)) + R(t)(z(t) + y_1(t))^2 \\
	\iff z'(t) - [Q(t) + 2y_1(t) R(t)] z(t) = R(t)(z(t))^2 \ (\text{ecuación de bernoulli con } n=2) \]

	\begin{eg}
		Resolver la EDO:
		\[ y'(t) = 6 + 5y(t) + (y(t))^2 \]
		\noindent Corresponde a una ecuación de Ricatti con:
		\[ P(t)=6, \quad Q(t)=5,\quad R(t)=1,\quad \forall t \in \R \]
		\noindent Nótese que $y_1(t) = -2,\ \forall t \in \R$ es solución. Luego, definimos:
		\begin{align*}
			& z(t) = y(t) - y_1(t) = y(t) + 2 \quad \forall t \in \R \\
			\implies & z'(t) - z(t) = (z(t))^2
		.\end{align*}

		\noindent Luego,
		\begin{align*}
			u'(t) + u(t) = -1 & \implies u(t) = C e^{-t} - 1 \quad \forall t \in \R \ (C \in \R) \\
			& \implies z(t) = \frac{1}{u(t)} = \frac{1}{C e^{-t} -1} \\
			& \implies y(t) = z(t) - 2 = \frac{1}{C e^{-t} -1 }-2
		.\end{align*}
	\end{eg}
		\begin{note}
			Si $C>0$, la solución "explota" cuando $ t = \log (C)$. 
		\end{note}

		\subsection{III. Problema de Cauchy: existencia y unicidad}

		Sea $\Omega \subset \R^{n+1}$ un conjunto abierto, que por lo general será de la forma $ \Omega = I \times \widetilde{\Omega}$, con $I \subset \R$ un intervalo abierto y $\widetilde{\Omega}\subset \R^{n}$ un conjunto abierto. Dada una función
		\[ f \in C (\Omega;\R^n) \]
		\noindent consideramos nuevamente el problema de Cauchy
		\begin{align*} (PC) \ \begin{cases}
			 y'(x) = f(x,y(x)), \quad \forall x \in I \text{ con } & x_0 \in I \text{ y} \\
			 y(x_0) = y_0, & y \in \R^n
		\end{cases} \end{align*}

		\noindent El (PC) puede ser formulado de manera equivalente, pero "relativamente" más débil:

		\begin{lemma}
			Sea $\Omega \subset \R^{n+1}$ un conjunto abierto de la forma $\Omega = I \times \widetilde{\Omega},\ I \subset \R$ intervalo abierto y $\widetilde{\Omega} \subset \R^n$ conjunto abierto. Dados $x_0 \in I,\ y_0 \in \R^n$ y $f \in C(\Omega;\R^n)$, una función $\varphi : I \to \R^n$ es solución de (PC) si y sólo si:
			\begin{enumerate}
				\item[(i)] $\varphi \in C(I;\R^n)$;

				\item[(ii)] $(x,\varphi(x)) \in \Omega \quad \forall x \in I$;

				\item[(iii)] $\varphi (x) = y_0 + \int_{x_0}^{x} f(s,\varphi(s)) ds \quad \forall x \in I$.
			\end{enumerate}
			(i, ii y iii es formulación integral del (PC)).
		\end{lemma}

		\begin{remark}
			La formulación integral nos permite estudiar el (PC) desde una perspectiva más abstracta. Supongamos por ahora que
			\[ \Omega = \R^{n+1},\ I = \R,\ \widetilde{\Omega} = \R^n \text{ y } f \in C(\R^{n+1};\R^n). \]
		\end{remark}

		\noindent Dados $y_0 \in \R^n$ y $x_0 \in \R$, consideramos la aplicación $T: C(\R;\R^n) \to C(\R;\R^n)$.
		\[ T(\varphi)(x) = y_0 + \int_{x_0}^{x} f(s,\varphi(s)) ds \quad \forall x \in \R \]
		
		\noindent Por el lema precedente, es evidente que $\varphi \in C(\R;\R^n)$ es solución de (PC) si y sólo si $T(\varphi) \equiv \varphi$ (i.e., $\varphi$ es punto fijo de $T$).

%############################################################################################%
%############################################################################################%
%############################################################################################%

\section{Clase (22/08)}

$\Omega \subset \R^{n+1},\ \Omega = I \times \widetilde{\Omega}$ con $I \subset \R$ intervalo abierto y $\widetilde{\Omega} \subset \R^n$ conjunto abierto.
\[ (PC) \ \begin{cases}
	y'(x) = f(x, y(x))\quad \forall x \in I, \\
	y(x_0) = y_0 \quad (x_0 \in I, y_0 \in \R^n)
\end{cases} \]

\begin{enumerate}
	\item[(i)] Resolver \underline{localmente} el problema (PC) corresponde a encontrar un intervalo $J \subset I$ y una función $\varphi \in C^1 (J;\R^n)$ tales que:
	\[ x_0 \in J,\ (x,\varphi(x)) \in \Omega \ \forall x \in J,\ \varphi'(x) = f(x,\varphi(x))\ \forall x \in J; \]

	\item[(ii)] Si $J_1,\ J_2 \subset I$ son intervalos abiertos que contienen a $x_0$, y $\phi_1 \in C^1(J_1;\R^n)$ y $\phi_2 \in C^1(J_2; \R^n)$ son soluciones locales de (PC), decimos que $\phi_2$ \underline{extiende} a $\phi_{1}$ es una restricción de $\phi_{2}$;

	\item[(iii)] Una solución es \underline{maximal} cuando no admite extensiones;

	\item[(iv)] Una solución local, definida sobre $J \subset I$, es \underline{global} si $J = I$.
\end{enumerate}

\noindent \textbf{Recuerdo.} Dados $a<b$ números reales, el espaacio vectorial
\[ C([a,b];\R^n) \]
\noindent dotado de la norma $\| \varphi \|_{\infty} = \max_{x \in [a,b]} |\varphi(x)| \quad \forall \varphi \in C([a,b];\R^n)$, es un espacio de Banach.

\begin{itemize}
	\item Todo sub-conjunto cerradeo de $(C([a,b];\R^n);\| \cdot \|_{\infty})$ es un espacio métrico completo.

	\item Todo sub-conjunto vectorial cerrado de $(C([a,b];\R^n);\| \cdot \|_{\infty})$ es un espacio de Banach.
\end{itemize}


\subsection{Aplicaciones contractivas: teorema del punto fijo de Baanch}

\begin{definition}[aplicación contractiva]
	Sea $(X,d)$ un espacio métrico completo. Decimos que una aplicación $T: X \to X$ es \underline{contractiva} si existe $\alpha \in [0,1)$ tal que
	\[ d(T(x),T(y)) \leq \alpha d(x,y) \quad \forall x,y \in X \]
\end{definition}

\begin{theorem}[punto fijo de Banach]
	Sea $(X,d)$ un espacio métrico completo y $T: X \to X$ una aplicación contractiva. Entonces, existe un único $\hat{x} \in X$ tal que $T(\hat{x}) = \hat{x}$.
\end{theorem}


\subsection{Funciones Lipschitz}

\begin{note}
	$\Omega \subset \R^{n+1},\ (x,y) \in \Omega,\ x \in \R$, y $y \in \R^n$
\end{note}

\begin{definition}[función globalmente Lipschitz]
	Sea $\Omega \subset \R^{n+1}$ abierto, y $f : \Omega \to \R^n$ una función. Decimos que $f$ es \underline{globalmente} Lipschitz respecto a la variable $y$ en $\Omega$ si existe una constante $L>0$ tal que
	\[ |f(x,y_1) - f(x,y_2)| \leq L | y_1 - y_2 | \quad \forall (x,y_1), (x,y_2) \in \Omega \]
\end{definition}

\begin{note}
	Lip$(y;\Omega)$ denota el espacio vectorial de todas las funciones $f:\Omega\to\R^n$ que son globalmente Lipschitz respecto a $y$ en $\Omega$.
\end{note}

\begin{definition}[función localmente Lipschitz]
	Sea $\Omega \subset \R^n$ abierto. Se dice que una función $f: \Omega \to \R^n$ es \underline{localmente} Lipschitz respecto a la variable $y$ en $\Omega$ si: para cualquier punto $(\overline{x}, \overline{y})\in\Omega$, existe $\varepsilon>0$ y una constante $L>0$ tales que $B((\overline{x},\overline{y}),\varepsilon) \subset \Omega$, y 
	\[ |f(x,y_1) - f(x,y_2)| \leq L |y_1 - y_2| \quad \forall (x,y_1),(x,y_2) \in B((\overline{x},\overline{y},\varepsilon). \]
\end{definition}

\begin{note}
	$\text{Lip}_{\textit{loc}}(y;\Omega)$.
\end{note}

\begin{prop}~
	\begin{enumerate}
		\item[(A)] Si $f \in \text{ Lip}(y;\Omega)$, entonces $f$ es uniformemente continua respecto a $y$ en $\Omega$: $(\forall \varepsilon > 0)(\exists \delta > 0)(\forall (x,y_1),(x,y_2) \in \Omega,$
		\[ |y_1-y_2|\leq\delta \implies |f(x,y_1)-f(x,y_2)| \leq \varepsilon \]

		\item[(B)] Si $f\in \text{ Lip}_{\textit{loc}}(y;\Omega)$, entonces $f$ es continua respecto a la variable $y$ en $\Omega$: para todo $(\overline{x},\overline{y})\in\Omega$:
		\[ (\forall \varepsilon > 0)(\exists \delta > 0)\ |y-\overline{y}|\leq \delta \implies (\overline{x},\overline{y})\in\Omega \text{ y } |f(\overline{x},\overline{y}) - f(\overline{x},y)| \leq \varepsilon \]
	\end{enumerate}
\end{prop}

\begin{remark}
	En general, Lip$(y;\Omega) \not\subseteq C(\Omega;\R^n)$. Por ejemplo,
	\[ f(x,y) = \begin{cases}
		0 \text{ si } x \leq 0 \\
		y \text{ si } x > 0
	\end{cases} \]
	pertenece a Lip$(y;\R^2)$, pero es discontinua en el conjunto $\{ (x,y) \in \R^2 \ | \ x = 0 \}$. Recíprocamente, la continuidad ("en pareja") de una función no implica ningún tipo de Lipschitzianidad (local o global).
\end{remark}

\begin{eg}
	$\hat{f}(x,y) = \sqrt{|y|}\quad \forall (x,y) \in \R^2$. Vemos que no es localmente Lipszhitz: $\hat{f} \not\in \text{ Lip}_{\textit{loc}}(y;\R^2)$: por contradicción, debiese existir $\varepsilon > 0$ tal que $f \in \text{ Lip}(y; B((0,0),\varepsilon))$. Por ende, existe una constante $L>0$ tal que
	\begin{align*}
		\left|f(0,0)-f \left(0,\frac{\varepsilon}{n}\right)\right| & \leq L \cdot \frac{\varepsilon}{n} \quad \forall n \geq 2, \\
		& \iff \sqrt{\frac{\varepsilon}{n}} \leq \frac{L \varepsilon}{n} \\
		& \iff \sqrt{n} \leq L \sqrt{\varepsilon} \quad \forall n \geq 2,
	\end{align*}
	\noindent lo cual es absurdo!
\end{eg}

\begin{theorem}
	Sea $\Omega\subset\R^{n+1}$ un conjunto abierto y
	\[ f = (f_1,f_2,\dots,f_n):\ \Omega \to \R^n \]
	\noindent una función tal que las derivadas parciales $\frac{\partial f_i}{\partial y_j}\ \forall i,j \in \{1,\dots,n\}$ existen y son continuas en $\Omega$. Entonces:
	\begin{enumerate}
		\item[(A)] $f \in \text{Lip}_{\textit{loc}}(y;\Omega)$;

		\item[(B)] Si, además, $\Omega$ es convexo, $f \in \text{Lip}(y;\Omega)$ si y sólo si
		\[ \sup_{(x,y)\in\Omega} \left| \frac{\partial f_i}{\partial y_j}(x,y) \right| < \infty \quad \forall i,j \in \{ 1,\dots, n \} \]
	\end{enumerate}
\end{theorem}





% end lessons
\end{document}
